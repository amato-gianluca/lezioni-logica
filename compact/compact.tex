\documentclass[aspectratio=169,10pt]{beamer}
\usepackage[T1]{fontenc}
\usepackage{lmodern} % Latin Modern fonts provide bold sans shapes in T1
\usepackage[italian]{babel}
\usepackage[utf8]{inputenc}
\usepackage{pifont}
\usepackage{qrcode}
\usepackage[
    type={CC},
    modifier={by-nc-sa},
    version={4.0}
]{doclicense}
\usepackage{tikz}
\usetikzlibrary{backgrounds, fit, calc, positioning, arrows.meta, overlay-beamer-styles}
\PassOptionsToPackage{height=1cm}{beamerouterthemesidebar}
\usetheme{Berkeley}
\deftranslation[to=italian]{Definition}{Definizione}
\deftranslation[to=italian]{Example}{Esempio}

% !TeX spellcheck = it_IT

\title{Logica}
\author{Gianluca Amato}

\newcommand{\xmark}{{\color{red}{\ding{55}}}}
\newcommand{\ra}{\rightarrow}
\newcommand{\conn}[1]{\textcolor{blue}{#1}}
\newcommand{\quant}[1]{\textcolor{blue}{#1}}
\newenvironment{inference}{\begin{tabular}{c}}{\end{tabular}}

\AtBeginSection[]
{
  \begin{frame}
  \frametitle{Contenuti}
   \tableofcontents[currentsection]
  \end{frame}
}

\begin{document}

\begin{frame}
    \titlepage
\end{frame}

\begin{frame}
    \centering
    \begin{minipage}{12cm}
        \doclicenseThis

        \medskip
        \href{\doclicenseURL}{\doclicenseURL}
    \end{minipage}
\end{frame}

\begin{frame}
    Queste slide sono basate sul libro:
    \bigskip

    \hrule
    \medskip
    Dario Palladino\\
    \textcolor{blue}{\href{https://www.carocci.it/prodotto/corso-di-logica-3}{Corso di logica: introduzione elementare al calcolo dei predicati (terza edizione)}}\\
    Carocci editore\\[0.2cm]

    \hrule

    \bigskip
    \centering
    \qrcode[height=4cm]{https://www.carocci.it/prodotto/corso-di-logica-3}
\end{frame}

\section{Introduzione}

\begin{frame}{Cos'è la logica}
    \begin{definition}[Logica]
        La \alert{logica} (dal greco \textit{logos}, ovvero ``parola'', ``pensiero'') è lo studio del ragionamento.
    \end{definition}
    \begin{example}[Esempi di ragionamento]
        \begin{columns}
            \column{0.5\textwidth}
            \begin{center}
                Tutti gli uomini sono mortali.\\
                Socrate è un uomo.\\
                Dunque Socrate è mortale.\\[0.6cm]
                Ogni numero intero è o pari o dispari\\
                Il numero intero 3 non è pari\\
                Dunque 3 è dispari.
            \end{center}
            \column{0.5\textwidth}
            \centering
            \includegraphics[width=2cm,keepaspectratio]{logica.png}

            \alert{Sono ragionamenti corretti?}
        \end{columns}
    \end{example}
    \begin{block}{Nota}
        Esistono varie forme di ragionamento, qui siamo interessati al \alert{ragionamento deduttivo}.
    \end{block}
\end{frame}

\subsection{Inferenze, proposizioni e connettivi}

\begin{frame}[label=inferenza]{Ragionamento deduttivo}
    La forma principale del ragionamento deduttivo è quella dell'\alert{inferenza}.
    \begin{definition}[Inferenza]
        Un'\alert{inferenza} è una sequenza di \alert{proposizioni} di cui l'ultima è ottenuta come \alert{conclusione} delle rimanenti, che si assumono come \alert{premesse}.
    \end{definition}
    \begin{example}
        \[
            \begin{tikzpicture}
                \tikzset{
                    fillstyle/.style={rounded corners, draw=black},
                    greenstyle/.style={fillstyle,fill=green!50},
                    redstyle/.style={fillstyle,fill=red!50},
                    bluestyle/.style={fillstyle,fill=blue!50},
                    yellowstyle/.style={fillstyle,fill=yellow!50}
                }
                \clip (-6cm, -3cm) rectangle (6cm, 0.5cm);
                \node (p1) {Tutti gli uomini sono mortali.};
                \node (p2) [below of=p1] {Socrate è un uomo.};
                \node (c) [below of=p2] {Socrate è mortale.};
                \begin{scope}[on background layer]
                    \only<2->{\node (inf) [greenstyle, fit={(p1) (p2) (c)}, inner sep=0.5cm] {};}
                    \only<4->{\node (premises) [redstyle, fit={(p1) (p2)},  inner sep=0.2cm] {};}
                    \only<5->{\node (conclusion) [yellowstyle, fit={(c)},  inner sep=0.2cm] {};}
                    \only<3->{\node (prop1) [bluestyle, fit={(p1)},  inner sep=-0.05cm] {};}
                    \only<3->{\node (prop2) [bluestyle, fit={(p2)},  inner sep=-0.05cm] {};}
                    \only<3->{\node (prop3) [bluestyle, fit={(c)},  inner sep=-0.05cm] {};}
                \end{scope}
                \draw ($(c.west) + (-1cm,0.5cm)$) -- ($(c.east) + (1cm,0.5cm)$);
                \only<2->{
                    \draw (inf) ++ (-4.5cm,0cm)  node (inflabel) [greenstyle] {Inferenza};
                    \draw [thick, ->] (inflabel) -- (inf.west);
                }
                \only<4->{
                    \draw (premises) ++ (-4.5cm,0cm)  node (premiseslabel) [redstyle] {Premesse};
                    \draw [thick, ->] (premiseslabel) -- (premises.west);
                }
                \only<5->{
                    \draw (conclusion) ++ (-4.5cm,0cm)  node (conclusionlabel) [yellowstyle] {Conclusione};
                    \draw [thick, ->] (conclusionlabel) -- (conclusion.west);
                }
                \only<3->{
                    \draw (prop2) ++ (+4.5cm,0cm)  node (proplabel) [bluestyle] {Proposizioni};
                    \draw [thick, ->] (proplabel) -- (prop1.east);
                    \draw [thick, ->] (proplabel) -- (prop2.east);
                    \draw [thick, ->] (proplabel) -- (prop3.east);
                }
            \end{tikzpicture}
            \vspace{-0.4cm}
        \]
    \end{example}
\end{frame}

\begin{frame}{Proposizioni}
    \begin{definition}[Proposizione]
        Una \alert{proposizione} è una qualsiasi espressione linguistica per la quale ha senso chiedersi se sia vera o falsa.
    \end{definition}
    % \begin{columns}
    % \column{0.25\textwidth}
    %     \begin{center}
    %         \includegraphics[width=3cm,keepaspectratio]{zelda.jpg}
    %     \end{center}
    % \column{0.7\textwidth}
    \begin{example}[Proposizioni \checkmark o no? \xmark]
        \begin{itemize}
            \item Link ama Zelda \only<1-4>{\makebox(0,0)[l]{\hspace{6cm}\vspace{-2.5cm}\raisebox{-1\totalheight}{\includegraphics[width=4cm,keepaspectratio]{zelda.jpg}}}} \pause \checkmark
                  \pause
            \item Link ama Zelda? \pause \xmark\ (è una \emph{domanda})
                  \pause
            \item $4 + 2 = 6$
                  \only<5-10>{\makebox(0,0)[l]{\hspace{7cm}\vspace{-1.5cm}\raisebox{-1\totalheight}{\includegraphics[width=3cm,keepaspectratio]{calcolatrice2.png}}}} \pause  \checkmark
                  \pause
            \item $4 + 2 = 5$ \pause \ \checkmark \ (è falsa, ma è comunque una proposizione)
                  \pause
            \item $3 + 5^2$ \pause \ \xmark \ (è una \emph{descrizione})
                  \pause
            \item Tutti gli uomini sono mortali
                  \only<11-12>{\makebox(0,0)[l]{\hspace{4cm}\vspace{1.5cm}\raisebox{-1\totalheight}{\includegraphics[width=4cm,keepaspectratio]{mortal-kombat-11.jpg}}}}
                  \pause \checkmark
                  \pause
            \item Lazzaro, alzati e cammina!
                  \only<13-14>{\makebox(0,0)[l]{\hspace{5cm}\vspace{2.5cm}\raisebox{-1\totalheight}{\includegraphics[width=2.5cm,keepaspectratio]{lazzaro2.jpg}}}}
                  \pause \xmark \ (è un \emph{ordine})
                  \pause
            \item Quel ramo del lago di Como che volge a mezzogiorno
                  \only<15-16>{\makebox(0,0)[l]{\hspace{1.3cm}\vspace{4.5cm}\raisebox{-1\totalheight}{\includegraphics[width=2.3cm,keepaspectratio]{LagoDiComo.jpg}}}}
                  \pause \xmark \ (è una \emph{descrizione})
        \end{itemize}
    \end{example}
    % \end{columns}
\end{frame}

\begin{frame}{Connettivi}
    \begin{columns}
        \column{0.65\textwidth}
        \begin{definition}[Connettivo]
            Un \alert{connettivo} è un elemento grammaticale che collega un certo numero di proposizioni tra di loro per formare una nuova proposizione.
        \end{definition}
        \column{0.3\textwidth}
        \begin{center}
            \includegraphics[width=3cm,keepaspectratio]{puzzle.png}
        \end{center}
    \end{columns}
    \begin{example}
        \begin{itemize}
            \item Roma è la capitale d'Italia \conn{e} Lione è la capitale della Francia
            \item Roma è la capitale d'Italia \conn{oppure} Lione è la capitale della Francia
            \item \conn{Se} Roma è la capitale d'Italia \conn{allora} Lione è la capitale della Francia
            \item \conn{Non è vero} che Roma è la capitale d'Italia.
        \end{itemize}
    \end{example}
\end{frame}

\againframe<6>{inferenza}

\begin{frame}{Inferenze corrette}
    \begin{definition}[Inferenza corretta]
        Un'inferenza è corretta se, ogni qualvolta le premesse sono vere, allora è necessariamente vera anche la conclusione.
    \end{definition}
    \begin{example}[un'inferenza corretta]
        \begin{center}
            \begin{inference}
                Carlo è ligure \conn{o} Stefano è piemontese\\
                Stefano \conn{non} è piemontese\\
                \hline
                Carlo è ligure
            \end{inference}
        \end{center}
    \end{example}
\end{frame}

\begin{frame}{Inferenze e forma logica}
    La correttezza di un'inferenza dipende solo dalla \alert{forma logica} delle proposizioni.
    \begin{example}[inferenze con la stessa forma logica]
        \begin{center}
            \begin{inference}
                Carlo è ligure \conn{o} Stefano è piemontese\\
                Stefano \conn{non} è piemontese\\
                \hline
                Carlo è ligure
            \end{inference}
        \end{center}
        \medskip

        \begin{center}
            \begin{inference}
                Il maggiordomo è colpevole, \conn{oppure} la cameriera è colpevole.\\
                La cameriera \conn{non} è colpevole\\
                \hline
                Il maggiordomo è colpevole
            \end{inference}
        \end{center}
        \medskip

        \begin{center}
            \begin{inference}
                \conn{O} due buchi neri si sono scontrati \conn{oppure} il rivelatore di onde gravitazionali è rotto.\\
                Il rivelatore di onde gravitazionali \conn{non} è rotto.\\
                \hline
                Due buchi neri si sono scontrati.
            \end{inference}
        \end{center}
    \end{example}
\end{frame}

\begin{frame}{Regole di inferenza}
    Mettiamo in evidenza la \alert{forma logica} dell'inferenza.\\
    \medskip
    Usiamo delle \alert{lettere proposizionali}, al posto delle proposizioni:
    \begin{center}
        A  = ``Carlo è ligure'' \hspace{1cm} B = ``Stefano è piemontese''
    \end{center}
    otteniamo
    \[
        \begin{tikzpicture}[remember picture]
            \node (carlo) {\begin{inference}
                    Carlo è ligure \conn{o} Stefano è piemontese\\
                    Stefano \conn{non} è piemontese\\
                    \hline
                    Carlo è ligure
                \end{inference}
            };
            \node (disj) [right=of carlo] {
                \begin{inference}
                    A o B\\
                    non B\\
                    \hline
                    A
                \end{inference}
            };
            \draw [-{Latex[width=3mm]},very thick] (carlo) -- (disj);
        \end{tikzpicture}
    \]
    \pause
    \begin{definition}[Regola di inferenza]
        Un'inferenza in cui usiamo \alert{lettere proposizionali}  invece di proposizioni vere e proprie prende il nome di \alert{regola di inferenza}.
        \begin{tikzpicture}[overlay, remember picture]
            \draw [red, thick] (disj) circle (1cm);
            \path (disj) ++ (0,-1.3cm) node [red] {Regola di inferenza};
        \end{tikzpicture}
    \end{definition}
\end{frame}

\subsection{Regole di inferenza corrette}

\begin{frame}{Regole di inferenza corrette}
    Se al contrario rimpiazziamo le lettere con delle proposizioni, troviamo un'\alert{istanza} della regola.
    Se una regola è corretta, tutte le istanze lo sono.

    \begin{example}[Sillogismo disgiuntivo]
        La regola di inferenza\\
        \begin{center}
            \begin{inference}
                A o B\\
                non B\\
                \hline
                A
            \end{inference}
        \end{center}
        chiamata \alert{regola del sillogismo disgiuntivo} è corretta, quindi\\
        \begin{center}
            \begin{inference}
                Il maggiordomo è colpevole, \conn{oppure} la cameriera è colpevole.\\
                La cameriera \conn{non} è colpevole\\
                \hline
                Il maggiordomo è colpevole
            \end{inference}\\
        \end{center}
        è corretta.
    \end{example}
\end{frame}

\begin{frame}{Un'altra regola corretta: modus ponens}
    Consideriamo l'inferenza
    \begin{center}
        \begin{inference}
            \conn{Se} Fabio è genovese, \conn{allora} Fabio è ligure\\
            Fabio è genovese\\
            \hline
            Fabio è ligure
        \end{inference}
    \end{center}
    \only<1-1>{\makebox(0,0)[l]{\hspace{3cm}\vspace{-2.5cm}\raisebox{-1\totalheight}{\includegraphics[width=8.5cm,keepaspectratio]{cartina_liguria.png}}}}
    \pause
    Si generalizza nella regola:
    \begin{center}
        \begin{inference}
            Se A allora B\\
            A\\
            \hline
            B
        \end{inference}
    \end{center}
    detta \alert{regola del modus ponens}. Vedi anche:
    \begin{center}
        \begin{inference}
            \conn{Se} sono colpevole \conn{allora} devo essere punito\\
            Sono colpevole\\
            \hline
            Devo essere punito
        \end{inference}
    \end{center}
\end{frame}

\begin{frame}{Un'altra regola corretta: modus tollens}
    Consideriamo l'inferenza
    \begin{center}
        \begin{inference}
            \conn{Se} Fabio è genovese, \conn{allora} Fabio è ligure\\
            Fabio \conn{non} è ligure\\
            \hline
            Fabio \conn{non} è genovese
        \end{inference}
    \end{center}
    \pause
    Si generalizza nella regola:
    \begin{center}
        \begin{inference}
            Se A allora B\\
            non B\\
            \hline
            non A
        \end{inference}
    \end{center}
    detta \alert{regola del modus tollens}. Vedi anche:
    \begin{center}
        \begin{inference}
            \conn{Se} sono colpevole \conn{allora} devo essere punito\\
            \conn{Non} devo essere punito\\
            \hline
            \conn{Non} sono colpevole
        \end{inference}
    \end{center}
\end{frame}

\subsection{Regole di inferenza non corrette}

\begin{frame}{Fallacia della negazione dell'antecedente (1)}
    Consideriamo l'inferenza
    \begin{center}
        \begin{inference}
            \conn{Se} sono un ladro \conn{allora} devo essere punito\\
            \conn{Non} sono un ladro\\
            \hline
            \conn{Non} devo essere punito
        \end{inference}
    \end{center}
    \only<1-1>{\makebox(0,0)[l]{\hspace{5cm}\vspace{-3.5cm}\raisebox{-1\totalheight}{\includegraphics[width=4.5cm,keepaspectratio]{lupin.png}}}}
    \pause
    che si generalizza nella regola
    \begin{center}
        \begin{inference}
            Se A allora B\\
            non A\\
            \hline
            non B
        \end{inference}
    \end{center}
    È corretta?
\end{frame}

\begin{frame}{Fallacia della negazione dell'antecedente (2)}
    All'apparenza lo sembra, ma consideriamo un'altra istanza della regola.
    \begin{center}
        \begin{inference}
            \conn{Se} Fabio è pescarese, \conn{allora} Fabio è abruzzese \only<3->{\checkmark}\\
            Fabio \conn{non} è pescarese \only<4->{\checkmark}\\
            \hline
            Fabio \conn{non} è abruzzese \only<5->{\xmark}
        \end{inference}
    \end{center}
    È corretta?
    \only<1-1>{\makebox(0,0)[l]{\hspace{5cm}\vspace{-3.5cm}\raisebox{-1\totalheight}{\includegraphics[width=4.5cm,keepaspectratio]{abruzzo.png}}}}

    \pause
    \medskip
    Questa regola \textbf{non è} corretta: pensate al caso in cui Fabio sia nato a Chieti.

    \onslide<6->{\medskip
        Abbiamo trovato un \alert{controesempio}. L'inferenza non è corretta.}

    \onslide<7->{\medskip
        Ed infatti, se ci pensiamo bene, non è corretta neanche la prima regola di inferenza: magari non sono un ladro, ma devo essere punito per qualche altro motivo.}
\end{frame}

\begin{frame}{Fallacia dell'affermazione del conseguente}

    Consideriamo questa inferenza\ldots
    \begin{center}
        \begin{inference}
            \conn{Se} sono un ladro \conn{allora} devo essere punito\\
            Devo essere punito\\
            \hline
            Sono un ladro
        \end{inference}
    \end{center}
    \pause
    che si generalizza nella regola
    \begin{center}
        \begin{inference}
            Se A allora B\\
            B\\
            \hline
            A
        \end{inference}
    \end{center}
    e questa\ldots
    \begin{center}
        \begin{inference}
            \conn{Se} Fabio è pescarese, \conn{allora} Fabio è abruzzese\\
            Fabio è abruzzese\\
            \hline
            Fabio è pescarese
        \end{inference}
    \end{center}
    \pause
    Anche se la prima può sembrarci corretta (ma non lo è, potrei dover essere punito per qualche altro motivo), la regola di inferenza
    \alert{non è corretta}.
\end{frame}

\begin{frame}{Correttezza delle inferenze e delle conclusioni}
    \begin{block}{Attenzione!}
        Se le premesse di un'inferenza sono false, anche la conclusione può essere falsa, sebbene l'inferenza sia corretta.
    \end{block}

    Abbiamo detto che il \alert{modus ponens} è corretto:
    \begin{center}
        \begin{inference}
            Se A allora B\\
            A\\
            \hline
            B
        \end{inference}
    \end{center}
    Ma guardiamo questa istanza:
    \begin{center}
        \begin{inference}
            \conn{Se} Napoleone è francese \conn{allora} Napoleone è abruzzese\\
            Napoleone è francese\\
            \hline
            Napoleone è abruzzese
        \end{inference}
    \end{center}
\end{frame}

% \begin{frame}{Premesse incosistenti}
%     \begin{center}
%         \begin{inference}
%         A\\
%         non A\\
%         \hline
%         B
%         \end{inference}
%     \end{center}
%     Questa inferenza vuol dire che se ``A'' è vero, ma nello stesso tempo ``non A'' è vero (quindi ``A'' è falso) allora possiamo inferire una qualunque proposizione B.

%     \medskip
%     In latino si usa talvolta dire \emph{ex falso quodlibet}, dal falso segue qualsiasi cosa.
% \end{frame}


\begin{frame}<presentation:0>{Regole di inferenza a livello predicativo}
    Negli esempi visti prima, le regole di inferenza sono a livello \alert{proposizionale}: la loro verità dipende dai legami tra le proposizioni.

    \medskip
    Queste inferenze sono più complesse:
    \begin{center}
        \begin{inference}
            2 è minore di 5\\
            Se un numero è minore di un altro, allora il secondo è maggiore del primo\\
            \hline
            5 è maggiore di 2
        \end{inference}

        \medskip\medskip
        \begin{inference}
            La retta $r$ è perpendicolare alla retta $s$\\
            Se una retta è perpendicolare a un'altra, la seconda interseca la prima\\
            \hline
            $s$ interseca $r$
        \end{inference}

        \medskip\medskip
        \begin{inference}
            Maria è moglie di Aldo\\
            Se una persona è moglie di un'altra, allora quest'ultima è marito della prima\\
            \hline
            Aldo è marito di Maria
        \end{inference}
    \end{center}
    Hanno la stessa \alert{forma logica}.
\end{frame}

\newcommand{\myfbox}[2]{\tikz[baseline=(n.base)]\node(n)[alt=<1>{fill=#1!50}]{#2};}
\newcommand{\myfboxbis}[2]{\tikz[baseline=(n.base)]\node(n)[alt=<2>{fill=#1!50}]{#2};}

\begin{frame}<presentation:0>{Evidenziare la forma logica}
    \tikzstyle{every node} = [rounded corners, outer sep=0, inner sep=0.1cm]
    \begin{center}
        \begin{inference}
            \myfbox{red}{2}\myfbox{green}{è minore di}\myfbox{blue}{5}\\
            Se\myfbox{orange}{un numero}\myfbox{green}{è minore di}\myfbox{pink}{un altro}, allora\myfbox{pink}{il secondo}\myfbox{gray}{è maggiore del} \myfbox{orange}{primo}\\
            \hline
            \myfbox{blue}{5}\myfbox{gray}{è maggiore di}\myfbox{red}{2}
        \end{inference}\\[0.2cm]
        $\Downarrow$\\[0.2cm]
        \begin{inference}
            \myfbox{green}{R}\myfbox{red}{a}\myfbox{blue}{b}\\
            \myfboxbis{green}{per ogni }\myfbox{orange}{$x$}, \myfboxbis{green}{per ogni } \myfbox{pink}{$y$}, \myfboxbis{red}{se} \myfbox{green}{R}\myfbox{orange}{$x$}\myfbox{pink}{$y$}\myfboxbis{red}{allora}\myfbox{gray}{S}\myfbox{pink}{$y$}\myfbox{orange}{$x$}\\
            \hline
            \myfbox{gray}{S}\myfbox{blue}{b}\myfbox{red}{a}\\
        \end{inference}\\[0.2cm]
        \only<2->{
            $\Downarrow$\\[0.2cm]
            \begin{inference}
                Rab\\
                $\myfboxbis{green}{$\forall x \forall y$} (Rxy \myfboxbis{red}{$\ra$} Syx)$\\
                \hline
                Sba
            \end{inference}
        }
    \end{center}
\end{frame}

\begin{frame}{Regole di inferenza con i quantificatori}
    \tikzstyle{every node} = [rounded corners, outer sep=0, inner sep=0.1cm]

    Negli esempi visti prima, le regole di inferenza sono a livello \alert{proposizionale}: la loro verità dipende dai legami tra le proposizioni.

    \medskip
    Queste inferenze sono più complesse:
    \begin{center}
        \begin{inference}
            \myfbox{red}{Napoleone} \myfbox{green}{è corso}\\
            Tutti \myfbox{green}{i corsi}  \myfbox{gray}{sono francesi}\\
            \hline
            \myfbox{red}{Napoleone} \myfbox{gray}{è francese}
        \end{inference}
        \qquad
        \begin{inference}
            \myfbox{red}{Socrate} \myfbox{green}{è un uomo}\\
            Tutti \myfbox{green}{gli uomini}  \myfbox{gray}{sono mortali}\\
            \hline
            \myfbox{red}{Socrate} \myfbox{gray}{è mortale}
        \end{inference}
    \end{center}
    ma hanno la stessa \alert{forma logica} e corrispondono alla regola
    \begin{center}
        \begin{inference}
            $\myfbox{green}{P} \myfbox{red}{(a)}$\\
            \conn{per ogni} $x$, \conn{se} $\myfbox{green}{P}(x)$ \conn{allora} $\myfbox{gray}{Q}(x)$\\
            \hline
            $\myfbox{gray}{Q}  \myfbox{red}{(a)}$
        \end{inference}
    \end{center}
\end{frame}

\section{Logica delle proposizioni}

\subsection{Negazione}

\begin{frame}{Negazione}
    La \alert{negazione} trasforma una proposizione vera in una falsa e viceversa.
    \medskip

    In italiano è di solito resa con:
    \begin{itemize}
        \item \conn{non è vero che} prima della proposizione da negare
        \item \conn{non} prima del verbo della proposizione da negare
    \end{itemize}
    \begin{example}
        Roma è la capitale d'Italia (\checkmark) \pause
        \begin{itemize}
            \item \conn{Non è vero che} Roma è la capitale d'Italia (\xmark)
            \item  Roma \conn{non} è la capitale d'Italia (\xmark)
        \end{itemize}
        \medskip

        \pause
        Dante Alighieri ha scritto ``I Promessi Sposi'' (\xmark) \pause
        \begin{itemize}
            \item \conn{Non è vero che} Dante Alighieri ha scritto ``I Promessi Sposi'' (\checkmark)
            \item  Dante Alighieri \conn{non} ha scritto ``I Promessi Sposi'' (\checkmark)
        \end{itemize}
    \end{example}
\end{frame}

\subsection{Congiunzione}

\begin{frame}{Congiunzione}
    La \alert{congiunzione} collega due proposizioni tra di loro. La nuova proposizione risultante è vera quando entrambe le proposizioni di partenza sono vere. \medskip

    In italiano è di solito resa con \conn{e} oppure \conn{ma}.
    \begin{example}
        Roma è la capitale d'Italia (\checkmark) \\
        Parigi è la capitale della Francia (\checkmark) \pause
        \begin{itemize}
            \item Roma è la capitale d'Italia \conn{e} Parigi è la capitale della Francia (\checkmark)
            \item Roma è la capitale d'Italia \conn{ma} Parigi è la capitale della Francia (\checkmark)
        \end{itemize}
        \medskip

        \pause
        $2+2 = 4$ (\checkmark) \\
        $3 \times 2= 5$  (\xmark) \pause
        \begin{itemize}
            \item $2+2 = 4$ \conn{e} $3 \times  2= 5$  (\xmark)
        \end{itemize}
    \end{example}
\end{frame}

\begin{frame}{Congiunzione nella lingua italiana}
    Talvolta in italiano si cerca di evitare le ripetizioni, e il connettivo \conn{e} non si trova più tra due proposizioni.

    \begin{example}
        Carlo \conn{e} Maria sono appassionati di baseball \medskip

        \hspace{2cm} è un'abbreviazione di\medskip

        Carlo è appassionato di baseball \conn{e} Maria è appassionata di baseball
    \end{example}

    \pause
    Ma attenzione!!! Non tutte le \conn{e} sono istanze della congiunzione
    \begin{example}
        Carlo e Maria sono amici\medskip

        \hspace{2cm} \textbf{non è} un'abbreviazione di\medskip

        Carlo è amico \conn{e} Maria è amica\medskip
    \end{example}
\end{frame}

\subsection{Disgiunzione}

\begin{frame}{Disgiunzione}
    La \alert{disgiunzione} collega due proposizioni tra di loro. La nuova proposizione risultante è vera quando \textbf{una} delle proposizioni di partenza è vera. \medskip

    In italiano è di solito resa con \conn{o} ed \conn{oppure}.
    \begin{example}
        Roma è la capitale d'Italia (\checkmark)\\
        Lione è la capitale della Francia (\xmark)\\
        \pause
        \begin{itemize}
            \item Roma è la capitale d'Italia \conn{o} Lione è la capitale della Francia (\checkmark)
            \item Roma è la capitale d'Italia \conn{oppure} Lione è la capitale della Francia (\checkmark)
            \item \conn{O} Roma è la capitale d'Italia \conn{oppure} Lione è la capitale della Francia (\checkmark)
        \end{itemize}
        \medskip

        \pause
        $2+2 = 5$ (\xmark) \\
        $3 \times 2= 7$  (\xmark) \pause
        \begin{itemize}
            \item  $2+2 = 5$ \conn{oppure} $3 \times 2= 7$  (\xmark)
        \end{itemize}
    \end{example}
\end{frame}

% \begin{frame}{Disgiunzione 2}
%     \begin{block}{Uso nella lingua italiana}
%     Riconsideriamo l'esempio di prima:\\
%     \begin{itemize}
%     \item Roma è la capitale d'Italia \alert{o} Parigi è la capitale della Francia
%     \end{itemize}
%     \medskip
%     È una frase piuttosto strana:
%     \begin{itemize}
%     \item difficilmente la sentiremmo dire nella vita di tutti i giorni;
%     \item se la pronunciassimo durante una interrogazione il prof. potrebbe anche esser tentato di correggerci
%     \begin{itemize}
%         \item \guillemotleft Carlo, dovresti dire ``Roma è la capitale d'Italia \alert{e} Parigi è la capitale della Francia'' \guillemotright
%         \item ma la nostra affermazione è vera, semmai non è molto precisa
%     \end{itemize}
%     \end{itemize}
%     \end{block}
% \end{frame}

\begin{frame}{Disgiunzione nella lingua italiana}
    Riconsideriamo l'esempio di prima:\\
    \begin{example}
        Roma è la capitale d'Italia \conn{o} Lione è la capitale della Francia
    \end{example}
    È una proposizione piuttosto strana. Nella lingua italiana, utilizziamo la disgiunzione quasi sempre quando le due proposizioni sono strettamente correlate tra di loro. A un'interrogazione, uno studente un po' impreparato potrebbe rispondere:
    \begin{example}
        Roma è la capitale d'Italia \conn{o} Milano è la capitale d'Italia
    \end{example}
    Spesso, in questi casi, abbreviamo la frase, come già visto per la congiunzione
    \begin{example}
        Roma \conn{o} Milano è la capitale d'Italia
    \end{example}
\end{frame}

\begin{frame}{Disgiunzione inclusiva ed esclusiva}
    Consideriamo ancora una proposizione un po' innaturale:
    \begin{example}
        Roma è la capitale d'Italia \conn{o} Parigi è la capitale della Francia
    \end{example}
    Cosa ne pensate? È vera o falsa?
    \pause
    \begin{itemize}
        \item se per voi la frase è \textbf{vera}, vuol dire che considerate la ``\conn{o}'' in senso \alert{inclusivo}
              \begin{itemize}
                  \item se entrambe le proposizioni di base sono vere, la disgiunzione è vera;
                  \item la disgiunzione inclusiva è vera quando \alert{una o più} delle proposizioni di base è vera;
              \end{itemize}
              \pause
        \item se per voi la frase è \textbf{falsa}, vuol dire che considerate la ``\conn{o}'' in  senso \alert{esclusivo}
              \begin{itemize}
                  \item se entrambe le proposizioni di base sono vere, la disgiunzione è falsa;
                  \item per voi la ``\conn{o}'' introduce un'alternativa tra due possibilità, una delle quali deve essere falsa;
                  \item la disgiunzione esclusiva è vera quando \alert{esattamente una} delle proposizioni di base è vera.
              \end{itemize}
    \end{itemize}
\end{frame}


\begin{frame}{Disgiunzione inclusiva ed esclusiva}
    Abbiamo detto prima
    \begin{definition}
        La \alert{disgiunzione} collega due proposizioni tra di loro. La nuova proposizione risultante è vera quando \textbf{una} delle proposizioni di partenza è vera.
    \end{definition}
    In realtà siamo stati imprecisi. Dovremmo dire:
    \begin{definition}[Disgiunzione inclusiva]
        La \alert{disgiunzione inclusiva} collega due proposizioni tra di loro. La nuova proposizione risultante è vera quando \textbf{almeno una} delle proposizioni di partenza è vera.
    \end{definition}
    e
    \begin{definition}[Disgiunzione esclusiva]
        La \alert{disgiunzione esclusiva} collega due proposizioni tra di loro. La nuova proposizione risultante è vera quando \textbf{esattamente una} delle proposizioni di partenza è vera.
    \end{definition}
\end{frame}

\begin{frame}{Disgiunzione inclusiva ed esclusiva nella lingua italiana 1}
    Ma chi ha ragione? La ``\conn{o}'' in italiano è inclusiva od esclusiva?
    \medskip

    Spesso il problema non si pone perché, dal contesto, sappiamo che le due proposizioni non possono essere entrambe vere.
    \pause

    \begin{example}
        Vi assicuro che la seguente proposizione è vera:
        \begin{itemize}
            \item La capitale della California è Sacramento \conn{o} Los Angeles.
        \end{itemize}
        Vuol dire che possono essere entrambe capitali ?
    \end{example}
    \pause
    Ovviamente no, perché sappiamo che uno stato non può avere due capitali, ma se questa  ``impossibilità'' dettata dal contesto non si verifica, l'italiano è molto ambiguo.
    \begin{example}
        Vi assicuro che la seguente proposizione è vera:
        \begin{itemize}
            \item La bandiera della California contiene una stella \conn{o} un orso.
        \end{itemize}
        Vuol dire che la bandiera può avere entrambe ?
    \end{example}
\end{frame}

\begin{frame}{Disgiunzione inclusiva ed esclusiva nella lingua italiana 2}
    In generale, nella lingua di tutti i giorni, non è chiaro se ``\conn{o}''  e ``\conn{oppure}'' vadano intesi in senso inclusivo o esclusivo, ma
    \begin{block}{Osservazione finale}
        \begin{itemize}
            \item quando si studia logica (quindi anche negli esercizi) lo si intende sempre inclusivo;
            \item in matematica è sempre inclusivo:
                  \begin{itemize}
                      \item la proposizione ``$2 + 2 = 4$ \conn{oppure} $5$ è dispari'' è vera!
                  \end{itemize}
        \end{itemize}
    \end{block}
\end{frame}

\begin{frame}{Approfondimento: tavole di verità}
    Si può descrivere il comportamento dei connettivi con le \alert{tavole di verità}.

    \begin{example}[Tavole di verità dei connettivi]
        Indichiamo con $P$, $Q$, \ldots delle proposizioni qualunque. Allora:
        \begin{center}
            \begin{tabular}[t]{c|c}
                $P$ & non $P$ \\
                \hline
                V   & F       \\
                F   & V
            \end{tabular}
            \hspace{2cm}
            \begin{tabular}[t]{c|c|c}
                $P$ & $Q$ & $P$ e $Q$ \\
                \hline
                F   & F   & F         \\
                F   & V   & F         \\
                V   & F   & F         \\
                V   & V   & V         \\
            \end{tabular}
            \hspace{2cm}
            \begin{tabular}[t]{c|c|c}
                $P$ & $Q$ & $P$ o $Q$ \\
                \hline
                F   & F   & F         \\
                F   & V   & V         \\
                V   & F   & V         \\
                V   & V   & V         \\
            \end{tabular}
        \end{center}
    \end{example}
    \begin{block}{Nota}
        Venite dall'ITIS e avete studiato le \alert{porte logiche} ? Noterete allora che i connettivi sono praticamente la stessa cosa delle porte logiche \texttt{not}, \texttt{and} e \texttt{or}.
    \end{block}
\end{frame}

\begin{frame}{Approfondimento: simboli matematici per i connettivi}
    Ai matematici piacciono i simboli:
    \begin{definition}[Simboli dei connettivi]
        \begin{itemize}
            \item negazione (non $P$):  $\neg P$
            \item congiunzione ($P$ e $Q$):  $P \land Q$
            \item disgiunzione ($P$ o $Q$):  $P \lor Q$

        \end{itemize}
    \end{definition}
    \begin{example}
        $2 + 2 = 4$ \conn{e} $3 \times 2 = 6$\medskip

        \hspace{2cm}$\Downarrow$ usando i simboli visti sopra \medskip

        $2 + 2 = 4 \land 3 \times 2 = 6$
    \end{example}
\end{frame}

\begin{frame}{Approfondimento: forme proposizionali}
    Usando le lettere $P$, $Q$, $R$ etc... e i simboli $\land$, $\lor$ e $\neg$ è possibile costruire delle \alert{formule logiche}, chiamate anche \alert{forme proposizionali}.
    \begin{example}
        $P \lor \neg (Q \land R)$ è una forma proposizionale.
    \end{example}
    \pause
    Una forma proposizionale è quella che abbiamo precedentemente chiamato \alert{forma logica}:
    \begin{example}[dalle forme proposizionali alle proposizioni]
        Considerate la forma proposizionale  $P \land \neg (Q \land R)$. Allora se
        \begin{itemize}
            \item $P={}$ Il maggiordomo è l'assassino
            \item $Q={}$ Il delitto è stato commesso in casa
            \item $R={}$ Il coltello è l'arma del delitto
        \end{itemize}
        otteniamo
        \begin{itemize}
            \item Il maggiordomo è l'assassino \conn{o} \conn{non è vero che} il delitto è stato commesso in casa \conn{con} il coltello.
        \end{itemize}
    \end{example}
\end{frame}

\subsection{Implicazione}

\begin{frame}{Implicazione}
    L'\alert{implicazione} collega due proposizioni $P$ e $Q$ tra di loro, e viene di solito resa in italiano con  ``\conn{se} $P$ \conn{allora} $Q$''.    \begin{itemize}
        \item $P$ è chiamato \alert{antecedente};
        \item $Q$ è chiamato \alert{conseguente}.
    \end{itemize}
    L'implicazione è sempre vera, tranne quando l'antecedente è vero e il conseguente è falso.
    \begin{example}[Vero o falso?]
        \begin{itemize}
            \item \conn{se} Roma è la capitale d'Italia \conn{allora} 3+2=5 \pause (\checkmark)
                  \pause
            \item \conn{se} Roma è la capitale d'Italia \conn{allora} 3+2=0 \pause (\xmark)
                  \pause
            \item \conn{se} gli elefanti volano \conn{allora} 3+2=5 \pause (\checkmark)
                  \pause
            \item \conn{se} gli elefanti volano \conn{allora} 3+2=0 \pause (\checkmark)
        \end{itemize}
    \end{example}
\end{frame}

\begin{frame}[fragile]{Implicazione e linguaggio naturale}
    Gli esempi di prima vi sono sembrati strani?
    \begin{block}{Note}
        Quello che vi sto presentando è solo un tipo particolare di logica, chiamata \alert{logica classica proposizionale}. Il \conn{se \ldots allora} in italiano spesso non corrisponde alla implicazione di questa logica.
    \end{block}
    \begin{example}
        \begin{itemize}
            \item \conn{se} sarai promosso a scuola, \conn{allora} ti comprerò la PlayStation 5
                  \begin{itemize}
                      \item coinvolge azioni che avverranno in futuro, non ora: è il reame della \alert{logica temporale};
                      \item la versione classica proposizionale  sarebbe:\\
                            ``\conn{se} sei stato promosso a scuola, \conn{allora} ti ho comprato la PlayStation 5''  \raisebox{-0.2\totalheight}{\includegraphics[width=0.5cm,keepaspectratio]{vomitosa.png}}
                  \end{itemize}
                  \pause
                  % \item \conn{se} un oggetto viene lanciato da una torre, \conn{allora} cade a terra
                  % \begin{itemize}
                  %     \item ``un oggetto viene lanciato da una torre'' non è un proposizione, perché non sta parlando né di un oggetto né di un momento specifico, è una affermazione ipotetica;
                  %     \item quello che intendiamo dire è che in ogni situazione possibile in cui un oggetto viene lanciato, necessariamente questo cadrà per terra;
                  %     \item necessità e possibilità sono il reame della \alert{logica modale}.
                  % \end{itemize}
            \item \conn{se} $n$ è divisibile per $4$ \conn{allora} $n$ è pari
                  \begin{itemize}
                      \item $n$ è divisibile per $4$ non è una proposizione, non possiamo dire se è vera è falsa perché ciò dipende dal valore di $n$: siamo nel reame della \alert{logica dei predicati};
                      \item la versione proposizionale sarebbe ``\conn{se} 8 è divisibile per $4$ \conn{allora} $8$ è pari'' \raisebox{-0.2\totalheight}{\includegraphics[width=0.5cm,keepaspectratio]{vomitosa.png}}
                      \item si potrebbe trovare forse in un testo matematico, ma non è certo così comune.
                  \end{itemize}
        \end{itemize}
    \end{example}
\end{frame}

\subsection{Inferenze ed equivalenze logiche}

\begin{frame}{Equivalenze logiche}
    \begin{definition}
        Due proposizioni sono \alert{logicamente equivalenti} quando hanno sempre lo stesso valore di verità, indipendentemente dai valori di verità delle proposizioni semplici che la compongono.
    \end{definition}
    \begin{columns}
        \column{0.65\textwidth}
        \begin{example}[Commutatività della congiunzione]
            \emph{Pikachu è giallo \conn{e} Bulbasaur è verde} [\alert{$P \land Q$}]\medskip\\
            \hspace{1cm} è equivalente a  (si usa il simbolo $\equiv$)\medskip\\
            \emph{Bulbasaur è verde \conn{e} Pikachu è giallo} [\alert{$Q \land P$}]
        \end{example}
        \begin{example}[Idempotenza della congiunzione]
            \emph{Pikachu è giallo \conn{e} Pikachu è giallo}  [\alert{$P \land P$}]\medskip\\
            \hspace{1cm}$\equiv$\medskip\\
            \emph{Pikachu è giallo}  [\alert{$P$}].
        \end{example}
        \column{0.3\textwidth}
        \begin{center}
            \includegraphics[width=2cm,keepaspectratio]{Pikachu.png}

            \includegraphics[width=2cm,keepaspectratio]{Bulbasaur.png}
        \end{center}
    \end{columns}
\end{frame}

\begin{frame}{Equivalenze logiche notevoli 1}
    Alcune equivalenze logiche sono molto importanti. Tra di queste:
    \begin{example}[Doppia negazione]
        \conn{Non è vero che} Pikachu \conn{non} è giallo [\alert{$\neg \neg P$}]\\
        \medskip
        \hspace{1cm}$\equiv$\\
        \medskip
        Pikachu è giallo [\alert{$P$}]
    \end{example}
\end{frame}

\begin{frame}{Equivalenze logiche notevoli 2}
    \begin{example}[Legge di De Morgan]
        \conn{Non è vero} che Pikachu \conn{e} Bulbasaur sono entrambi gialli [\alert{$\neg (P \land Q)$}] \\
        \medskip
        \hspace{1cm}$\equiv$\\
        \medskip
        Pikachu \conn{non} è giallo \conn{o} Bulbasaur \conn{non} è giallo [\alert{$\neg P \lor \neg Q$}]
    \end{example}

    \begin{example}[Legge di De Morgan]
        \conn{Non è vero} che \conn{o} Pikachu \conn{o} Bulbasaur è rosso [\alert{$\neg (P \lor Q)$}] \\
        \medskip
        \hspace{1cm}$\equiv$\\
        \medskip
        Pikachu \conn{non} è rosso \conn{e} Bulbasaur \conn{non} è rosso [\alert{$\neg P \land \neg Q$}]
    \end{example}
\end{frame}

\begin{frame}{Equivalenze logiche notevoli 3}
    \begin{example}[Contronominale di un'implicazione]
        \conn{Se} $n$ è divisibile per $4$ \conn{allora} $n$ è pari [\alert{$P \to Q$}]\\
        \medskip
        \hspace{2cm}$\equiv$\\
        \medskip
        \conn{Se} $n$ \conn{non} è pari \conn{allora} $n$ \conn{non} è divisibile per 4 [\alert{$\neg Q \to \neg P$}]
    \end{example}
    Invece, le seguenti equivalenze sono false
    \begin{columns}
        \column{0.40\textwidth}
        \begin{example}[Implicazione inversa]
            \conn{Se} $n$ è divisibile per $4$ \conn{allora} $n$ è pari\\[0cm] [\alert{$P \to Q$}]\\
            \medskip
            \hspace{2cm}\alert{$\not\equiv$}\\
            \medskip
            \conn{Se} $n$ è pari \conn{allora} $n$ è divisibile per 4\\[0cm] [\alert{$Q \to P$}]
        \end{example}
        \column{0.55\textwidth}
        \begin{example}[Implicazione contraria]
            \conn{Se} $n$ è divisibile per $4$ \conn{allora} $n$ è pari\\[0cm] [\alert{$P \to Q$}]\\
            \medskip
            \hspace{2cm}\alert{$\not\equiv$}\\
            \medskip
            \conn{Se} $n$ \conn{non} è divisibile per 4 \conn{allora} $n$ \conn{non} è pari \\[0cm] [\alert{$\neg P \to \neg Q$}]
        \end{example}
    \end{columns}
\end{frame}

%\begin{frame}{Inferenze notevoli}
%\end{frame}

\section{Logica dei predicati}

\subsection{Quantificatori}

\begin{frame}{Funzioni proposizionali}
    Consideriamo ora queste frasi:
    \begin{itemize}
        \item $x$ è la capitale d'Italia
        \item $0 \cdot x = 3$\\
        \item $x^2 \geq 0$
    \end{itemize}
    Sono proposizioni?
    \medskip

    \pause
    No, perché il fatto che siano vere o false dipende da chi sono $x$ e $y$! Si chiamano \alert{funzioni proposizionali}. \medskip

    Se rimpiazziamo le variabili con dei \alert{termini}, torniamo ad avere delle proposizioni:
    \begin{itemize}
        \item Pescara è la capitale d'Italia
        \item $0 \cdot \mathbf{2} = 3$
        \item $\mathbf{3}^2 \geq 1$
    \end{itemize}

    Ma c'è un altro modo per ottenere delle proposizioni dalle forme proposizionali: si possono usare i \alert{quantificatori}.\medskip
\end{frame}

\begin{frame}{Quantificatore esistenziale}

    \begin{definition}[Quantificatore esistenziale]
        Il quantificatore esistenziale ``\quant{esiste $x$ tale che}'' ($\exists x$) trasforma una funzione proposizionale contenente $x$ in una proposizione, che è vera quando \textbf{c'è almeno un valore} che è possibile rimpiazzare al posto della $x$ che rende vera la funzione proposizionale.
    \end{definition}

    \begin{example}
        \quant{Esiste $x$ tale che} $x$ è la capitale d'Italia \pause (\checkmark)
        \begin{itemize}
            \item se pongo $x=\text{Roma}$, la proposizione $Roma$  è la capitale d'Italia diventa vera.
        \end{itemize}
        \pause
        \quant{Esiste $x$ tale che} $0 \cdot x = 3$ --- in simboli $\exists x, \, 0 \cdot x = 3$ \pause (\xmark)
        \begin{itemize}
            \item non c'è nessun valore che posso mettere al posto di $x$ che rende $0 \cdot x = 3$ vero.
        \end{itemize}
        \pause
        \quant{Esiste $x$ tale che} $x^2 \geq 0$ --- in simboli $\exists x, \, x^2 \geq 0$ \pause (\checkmark)
        \begin{itemize}
            \item se pongo $x=1$, abbiamo che $1^2 \geq 0$ è vera.
        \end{itemize}
    \end{example}
\end{frame}

\begin{frame}{Quantificatore universale}

    \begin{definition}[Quantificatore universale]
        Il quantificatore universale ``\quant{per ogni $x$}'' ($\forall x$) trasforma una funzione proposizionale contenente $x$ in una proposizione, che è vera quando \textbf{qualunque valore} che rimpiazzo al posto di $x$ rende vera la funzione proposizionale.
    \end{definition}

    \begin{example}
        \quant{Per ogni $x$,} $x$ è la capitale d'Italia \pause (\xmark)
        \begin{itemize}
            \item non è vero che qualunque cosa metto al posto di $x$ ottengo una proposizione vera, ad esempio, se $x=\text{Pescara}$ si ottiene ``Pescara è la capitale d'Italia'' che è falsa.
        \end{itemize}
        \pause
        \quant{Per ogni $x$,} $0 \cdot x = 3$ --- in simboli $\forall x, \, 0 \cdot x = 3$  \pause (\xmark)
        \begin{itemize}
            \item è falso
        \end{itemize}
        \pause
        \quant{Per ogni $x$,} $x^2 \geq 0$  in simboli $\forall x, \, x^2 \geq 0$ \pause (\checkmark)
        \begin{itemize}
            \item è vera perché il quadrato di un numero, qualunque esso sia, è sempre positivo (o tutt'al più nullo)
        \end{itemize}
    \end{example}
\end{frame}

\begin{frame}{Dominio di quantificazione}
    Nella proposizione ``\quant{Per ogni $x$,} $x$ è la capitale d'Italia'' quali sono i valori di $x$ sensati da provare?
    \begin{itemize}
        \item $x=\text{Roma}$ ha senso, e ``Roma è la capitale d'Italia'' è vera;
        \item $x=\text{Pescara}$ ha senso, anche se ``Pescara è la capitale d'Italia'' è falsa;
        \item $x=34$ non ha molto senso
    \end{itemize}
    Ogni quantificatore (universale o esistenziale) ha un \alert{dominio di quantificazione} implicito.
    \begin{itemize}
        \item \quant{Per ogni $x$}, $x$ è la capitale d’Italia \pause
              \begin{itemize}
                  \item dominio: insieme di tutte le città
              \end{itemize}
              \pause
        \item \quant{Per ogni $x$}, $3 + x = 4$ \pause
              \begin{itemize}
                  \item dominio: insieme dei numeri (ma quali: numeri interi, razionali, reali, \ldots?)
              \end{itemize}
    \end{itemize}
\end{frame}

\begin{frame}{Quantificatore universale limitato}
    Certe volte vogliamo essere espliciti sul dominio di quantificazione.
    \begin{example}
        \quant{Per ogni $x$}, $x + 2 > 0$ \pause
        \begin{itemize}
            \item è ambiguo... se $x$ può essere un numero qualunque è falso (pensiamo a $x=-10$)
            \item ma se $x$ sono solo i numeri naturali (interi positivi), allora è vero
        \end{itemize}
        \pause
        \quant{Per ogni} $x \in \mathbb{N}$, $x + 2 > 0$ \pause
        \begin{itemize}
            \item così specifichiamo chiaramente che $x$ lo prendiamo nei numeri naturali
            \item si chiama \alert{quantificatore limitato}
        \end{itemize}
    \end{example}
    In realtà, il quantificatore limitato è solo una notazione compatta per quello standard:
    \begin{example}
        \quant{Per ogni} $x \in \mathbb{N}$, $x + 2 > 0$ \\
        \medskip
        \hspace{2cm}$\equiv$\\
        \medskip
        \quant{Per ogni $x$}, \quant{se} $x \in \mathbb{N}$ \quant{allora} $x + 2 > 0$
    \end{example}
\end{frame}

\begin{frame}{Quantificatore esistenziale limitato}
    \begin{example}
        \quant{Esiste $x$ tale che} $2x = 1$ \pause (?)
        \begin{itemize}
            \item è ambiguo... se $x$ può essere un numero qualunque è vero (pensiamo ad esempio a $x=\frac{1}{2}$)
            \item ma se $x$ sono solo i numeri interi allora è falso
        \end{itemize}
        \quant{Esiste} $x \in \mathbb{Q}$ \quant{tale che} $2x = 1$
        \begin{itemize}
            \item chiariamo che $x$ può essere scelto tra i numeri razionali
        \end{itemize}
        \quant{Esiste $x$ tale che} $x \in \mathbb{Q}$ \conn{e} $2x=1$
        \begin{itemize}
            \item versione estesa
        \end{itemize}
    \end{example}
\end{frame}

\begin{frame}{Quantificatori e linguaggio naturale}
    Le proposizioni viste nei lucidi precedenti non sono propriamente in italiano.
    \begin{itemize}
        \item In italiano non si usano le variabili!! Al loro posto usiamo nomi e pronomi.
        \item L'uso di ``\quant{per ogni}'' in italiano è raro. Sono più comuni \quant{tutti}, \quant{qualunque}, ecc.
        \item Anche per ``\quant{esiste}'' in italiano ci sono molte alternative come \quant{c'è}, \quant{c'è almeno un}, ecc.
    \end{itemize}

    Proposizioni quantificate in italiano si possono comunque ``tradurre'' con ``\quant{per ogni}'' ed ``\quant{esiste}''.
    \begin{columns}
        \column{0.45\textwidth}
        \begin{example}
            \quant{Tutti} gli uomini sono mortali\\
            \medskip
            \hspace{1cm}$\equiv$\\
            \medskip
            \quant{per ogni} $x$, \conn{se} $x$ è un uomo \conn{allora} $x$ è mortale
        \end{example}
        \column{0.45\textwidth}
        \begin{example}
            \quant{Esiste} un elefante che vola\\
            \medskip
            \hspace{1cm}$\equiv$\\
            \medskip
            \quant{esiste $x$ tale che} $x$ è un elefante \conn{e} $x$ vola
        \end{example}
    \end{columns}
\end{frame}

\subsection{Equivalenze nella logica dei predicati}

\begin{frame}{Equivalenze logiche notevoli con i quantificatori 1}
    \begin{example}
        Tutti gli usignoli volano e trillano [\alert{$\forall x\in U , P(x) \land Q(x)$}]\\
        \medskip
        \hspace{1cm}$\equiv$\\
        \medskip
        Tutti gli usignoli volano e tutti gli usignoli trillano [\alert{$(\forall x \in U, P(x)) \land (\forall x \in U, Q(x))$}]
    \end{example}

    \begin{example}
        C'è un pokemon di colore giallo o verde\\[0cm] [\alert{$\exists x \in P, G(x) \lor V(x)$}]\\
        \medskip
        \hspace{1cm}$\equiv$\\
        \medskip
        C'è un pokemon di colore giallo o c'è un pokemon di colore verde\\[0cm] [\alert{$(\exists x \in P, G(x)) \lor (\exists x \in P, V(x))$}]
    \end{example}
\end{frame}

\begin{frame}{Equivalenze logiche notevoli con i quantificatori 2}
    \begin{example}
        Non tutti i gatti sono neri\\
        Non è vero che per ogni $x$, se $x$ è un gatto allora $x$ è nero\\[0cm]
        [\alert{$\neg \forall x, G(x) \to N(x)$}]\\
        \medskip
        \hspace{1cm}$\equiv$\\
        \medskip
        Esiste un gatto che non è nero\\
        Esiste $x$ tale che $x$ è un gatto ma $x$ non è nero\\[0cm]
        [\alert{$\exists x, G(x) \land \neg N(x)$}]
    \end{example}

    \begin{example}
        Non esiste un elefante volante\\[0cm]
        [\alert{$\neg \exists x, E(x) \land  V(x)$}]\\[0cm]
        \medskip
        \hspace{1cm}$\equiv$\\
        \medskip
        Tutti gli elefanti sono non volanti\\[0cm]
        [\alert{$\forall x, E(x) \to \neg V(x)$}]
    \end{example}
\end{frame}

\section{Esempi di quiz}

\begin{frame}{Esempio}
    Consideriamo le seguenti affermazioni:
    \begin{enumerate}
        \item Tutti i matematici sono distratti \only<2>{[\alert{$\forall x, M(x) \to D(x)$}]}
        \item Luigi è distratto \only<2>{[\alert{$D(\text{Luigi})$}]}
        \item Tutte le persone distratte amano nuotare \only<2>{[\alert{$\forall x, D(x) \to N(x)$}]}
    \end{enumerate}
    Se queste affermazioni sono vere, quale altra, tra le seguenti affermazioni, è necessariamente vera?
    \begin{enumerate}[A]
        \item Tutte le persone che amano nuotare sono distratte.
        \item Luigi è un matematico.
        \item Tutte le persone distratte sono matematici.
        \item Luigi ama nuotare.
        \item Tutti i matematici amano nuotare.
    \end{enumerate}
\end{frame}

\begin{frame}{Esempio}
    Consideriamo le seguenti affermazioni:
    \begin{enumerate}
        \item Tutti i matematici sono distratti [\alert{$\forall x, M(x) \to D(x)$}]
        \item Luigi è distratto [\alert{$D(\text{Luigi})$}]
        \item Tutte le persone distratte amano nuotare [\alert{$\forall x, D(x) \to N(x)$}]
    \end{enumerate}
    Se queste affermazioni sono vere, quale altra, tra le seguenti affermazioni, è necessariamente vera?
    \begin{enumerate}[A]
        \item Tutte le persone che amano nuotare sono distratte \only<2->{\alert{[$\forall x, N(x) \to D(x)$]}} \only<3>{(\xmark)}
    \end{enumerate}
    \pause
    \medskip
    Non c'è motivo per cui debba essere vera (la proposizione $3$ afferma l'inverso)
\end{frame}

\begin{frame}{Esempio}
    Consideriamo le seguenti affermazioni:
    \begin{enumerate}
        \item Tutti i matematici sono distratti [\alert{$\forall x, M(x) \to D(x)$}]
        \item Luigi è distratto [\alert{$D(\text{Luigi})$}]
        \item Tutte le persone distratte amano nuotare [\alert{$\forall x, D(x) \to N(x)$}]
    \end{enumerate}
    Se queste affermazioni sono vere, quale altra, tra le seguenti affermazioni, è necessariamente vera?
    \begin{enumerate}[A]\addtocounter{enumi}{2}
        \item Tutte le persone distratte sono matematici \only<2->{\alert{[$\forall x, D(x) \to M(x)$]}} \only<3>{(\xmark)}
    \end{enumerate}
    \pause
    \medskip
    Non c'è motivo per cui debba essere vera (la proposizione $1$ afferma l'inverso)
\end{frame}

\begin{frame}{Esempio}
    Consideriamo le seguenti affermazioni:
    \begin{enumerate}
        \item Tutti i matematici sono distratti [\alert{$\forall x, M(x) \to D(x)$}]
        \item Luigi è distratto [\alert{$D(\text{Luigi})$}]
        \item Tutte le persone distratte amano nuotare [\alert{$\forall x, D(x) \to N(x)$}]
    \end{enumerate}
    Se queste affermazioni sono vere, quale altra, tra le seguenti affermazioni, è necessariamente vera?
    \begin{enumerate}[A]\addtocounter{enumi}{1}
        \item Luigi è un matematico. \only<2->{\alert{[$M(\text{Luigi})$]}} \only<3>{(\xmark)}
    \end{enumerate}
    \pause
    \medskip
    Non c'è motivo per cui debba essere vera.
\end{frame}

\begin{frame}{Esempio}
    Consideriamo le seguenti affermazioni:
    \begin{enumerate}
        \item Tutti i matematici sono distratti [\alert{$\forall x, M(x) \to D(x)$}]
        \item Luigi è distratto [\alert{$D(\text{Luigi})$}]
        \item Tutte le persone distratte amano nuotare [\alert{$\forall x, D(x) \to N(x)$}]
    \end{enumerate}
    Se queste affermazioni sono vere, quale altra, tra le seguenti affermazioni, è necessariamente vera?
    \begin{enumerate}[A]\addtocounter{enumi}{3}
        \item Luigi ama nuotare. \only<2->{[\alert{$N(\text{Luigi})$}]} \only<3>{(\checkmark)}
    \end{enumerate}
    \pause
    \medskip
    L'inferenza è corretta. Infatti:
    \begin{itemize}
        \item da (3) sappiamo che tutte le persone distratte amano nuotare; ne inferiamo che:\\
              \hspace{1cm}\conn{se} Luigi è distratto, \conn{allora} Luigi ama nuotare [\alert{$D(\text{Luigi}) \to N(\text{Luigi})$}]
        \item da (2) e la proposizione di sopra, per la regola chiamata \alert{modus ponens}, sappiamo che:\\
              \hspace{1cm}Luigi ama nuotare [\alert{$N(\text{Luigi})$}]
    \end{itemize}
\end{frame}



\begin{frame}{Esempio}
    Consideriamo le seguenti affermazioni:
    \begin{enumerate}
        \item Tutti i matematici sono distratti [\alert{$\forall x, M(x) \to D(x)$}]
        \item Luigi è distratto [\alert{$D(\text{Luigi})$}]
        \item Tutte le persone distratte amano nuotare [\alert{$\forall x, D(x) \to N(x)$}]
    \end{enumerate}
    Se queste affermazioni sono vere, quale altra, tra le seguenti affermazioni, è necessariamente vera?
    \begin{enumerate}[A]\addtocounter{enumi}{4}
        \item Tutti i matematici amano nuotare. \only<2->{\alert{[$\forall x, M(x) \to N(x)$]}} \only<3>{(\checkmark)}
    \end{enumerate}
    \pause
    \medskip
    Sebbene questa proposizione non sia esplicitamente contenuta nelle premesse, è ovviamente inferibile dalle premesse.
    \begin{itemize}
        \item Preso una persona qualunque ($x$), se quella persona è un matematico ($M(x)$)\ldots
        \item Dalla (1) consegue che quella persona è distratta ($D(x)$)
        \item Dalla (3) consegue che quella persona ama nuotare ($N(x)$)
    \end{itemize}
\end{frame}

% \begin{frame}{Esempio 1}
%  Consideriamo le seguenti affermazioni:
%   \begin{enumerate}
% \item Tutti i matematici sono distratti [\alert{$\forall x, M(x) \to D(x)$}]
% \item Luigi è distratto [\alert{$D(\text{Luigi})$}]
% \item Tutte le persone distratte amano nuotare [\alert{$\forall x, D(x) \to N(x)$}]
% \end{enumerate}
% Se queste affermazioni sono vere, quale altra, tra le seguenti affermazioni, è necessariamente vera?
% \begin{enumerate}[A]\addtocounter{enumi}{4}
% \item Tutte le persone distratte sono matematici \only<3>{(\xmark)}
% \end{enumerate}
% \pause
% \end{frame}

% \begin{frame}{Esempio 2}
% Il regolamento edilizio del comune prevede che non possa essere data l’abitabilità ad un immobile destinato ad abitazione se esso non rispetta i parametri richiesti di risparmio idrico ed energetico. Questa legge viene sempre applicata e questo vuol dire che:
% \begin{enumerate}[A]
% \item tutte le abitazioni che rispettano i parametri richiesti di risparmio idrico rispettano anche quelli di risparmio energetico
% \item se un’abitazione rispetta i parametri richiesti di risparmio idrico ed energetico allora siamo sicuri che otterrà l’abitabilità
% \item rispettare i parametri richiesti di risparmio idrico ed energetico è una condizione necessaria per ottenere l’abitabilità
% \item rispettare i parametri richiesti di risparmio idrico ed energetico è una condizione sufficiente per ottenere l’abitabilità
% \item ci sono abitazioni che non rispettano i parametri richiesti di risparmio idrico ma che hanno l’abitabilità
% \end{enumerate}
% \end{frame}

{
\setlength{\voffset}{0pt}
\setlength{\headsep}{0pt}
% all template changes are local to this group.
%~ \setbeamertemplate{navigation symbols}{}
\usebackgroundtemplate{\noindent\includegraphics[width=\paperwidth,height=\paperheight]{thats_all.jpg}}
\frame[plain,label=END]{
}
}

\end{document}
