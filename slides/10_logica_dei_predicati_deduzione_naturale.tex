\documentclass[aspectratio=169,10pt,handout]{beamer}

\usepackage{bussproofs}
\EnableBpAbbreviations

%\setbeameroption{show notes}

\input{preamble.inc}

\title{Deduzione naturale nella logica dei predicati}

\begin{document}

\begin{frame}
    \titlepage
\end{frame}

\begin{frame}
    \copyrightpage
\end{frame}

\begin{frame}
    \bookpage{sulla sezioni 10.4}
\end{frame}

\begin{frame}{Deduzione naturale}
    \begin{itemize}
        \item Il calcolo logico della deduzione naturale che abbiamo introdotto per la logica proposizionale si può estendere alla logica dei predicati.
        \item In questo caso è ancora più importante, perché non c'è un metodo sistematico per capire se una regola di inferenza è corretta o no come quello delle tavole di verità.
        \item Le regole della deduzione naturale sono le stesse di prima, con l'aggiunta di quattro nuove regole:
              \begin{itemize}
                  \item introduzione del quantificatore universale;
                  \item eliminazione del quantificatore universale;
                  \item introduzione del quantificatore esistenziale;
                  \item eliminazione del quantificatore esistenziale.
              \end{itemize}
    \end{itemize}
\end{frame}

\begin{frame}{Eliminazione del quantificatore universale}
    \begin{prooftree}
        \AXC{$\forall x \phi(x)$}
        \RightLabel{($\elim\forall$)}
        \UIC{$\phi(k)$}
    \end{prooftree}
    dove
    \begin{itemize}
        \item $\phi(x)$ indica una qualunque formula potenzialmente contenente $x$;
        \item $t$ è un termine (costante individuale o variabile libera) qualunque;
        \item $\phi(k)$ è ottenuta rimpiazzando le occorrenze libere di $x$ con la costante $k$.
    \end{itemize}

    \medskip
    Informalmente, vuol dire che se so che $\phi(x)$ è vera per tutti gli individui, allora sarà sicuramente vera per $k$, chiunque sia $k$.
    \begin{example}[$\forall x (Px \to Qx), Pa \vdash Qa$]
        \begin{prooftree}
            \AXC{$\forall x (Px \to Qx)$}
            \RightLabel{($\elim\forall$)}
            \UIC{$Pa \to Qa$}
            \AXC{$Pa$}
            \RightLabel{($\elim\to$)}
            \BIC{$Qa$}
        \end{prooftree}
    \end{example}
\end{frame}

\begin{frame}{Introduzione del quantificatore esistenziale}
    \begin{prooftree}
        \AXC{$\phi(k)$}
        \RightLabel{($\intro\exists$)}
        \UIC{$\exists x \phi(x)$}
    \end{prooftree}
    \medskip
    Informalmente, vuol dire che se so che $\phi$ è vera per un certo individuo specifico $k$, posso concludere che esiste un individuo $x$ che rende vera $\phi$ (l'individuo in questione è proprio $k$!).
    \begin{example}[$\forall x Px \vdash \exists x Px$]
        \begin{prooftree}
            \AXC{$\forall x Px$}
            \RightLabel{($\elim\forall$)}
            \UIC{$Pa$}
            \RightLabel{($\intro\exists$)}
            \UIC{$\exists x Px$}
        \end{prooftree}
    \end{example}
\end{frame}

\begin{frame}{Introduzione del quantificatore universale}
    Come posso concludere una formula del tipo $\forall x Px$ ?

    \medskip
    Dimostrando che $Px$ è vera per un individuo $k$ qualsiasi, di cui non so nulla.

    \begin{center}
        \framebox{\begin{minipage}{0.8\textwidth}
                \begin{prooftree}
                    \AXC{$\phi(k)$}
                    \RightLabel{($\intro\forall$)}
                    \UIC{$\forall x \phi(x)$}
                \end{prooftree}
                \begin{center}
                    se $k$ non appare in $\phi(x)$ e in alcuna ipotesi da cui dipende $\phi(k)$
                \end{center}
            \end{minipage}}
    \end{center}

    \begin{example}[$\forall x (Px \to Qx), \forall x Px \vdash \forall x Qx$]
        \begin{prooftree}
            \AXC{$\forall x (Px \to Qx)$}
            \RightLabel{($\elim\forall$)}
            \UIC{$\forall a (Pa \to Qa)$}
            \AXC{$\forall x Px$}
            \RightLabel{($\elim\forall$)}
            \UIC{$Pa$}
            \RightLabel{($\elim\to$)}
            \BIC{$Qa$}
            \RightLabel{($\intro\forall$)}
            \UIC{$\forall x Qx$}
        \end{prooftree}
    \end{example}
\end{frame}

\begin{frame}{Cosa accade se dimentico la condizione a latere ?}
    La regola di inferenza è falsa. Ad esempio, si può dimostrare che $Pa \vdash \forall x Px$ con un singolo passo:
    \begin{prooftree}
        \AXC{$Pa$}
        \RightLabel{($\intro\forall$)}
        \UIC{$\forall x Px$}
    \end{prooftree}

    \medskip
    Come dire che
    \begin{center}
        \begin{inference}
            Carlo studia matematica\\
            \hline
            Tutti studiano matematica
        \end{inference}
    \end{center}
\end{frame}

\begin{frame}{Eliminazione del quantificatore esistenziale}
    Come posso usare il fatto di sapere che $\exists x Px$ ? Ricordiamo che $\exists$ è una specie di disgiunzione infinita.

    \begin{center}
        \framebox{\begin{minipage}{0.8\textwidth}
                \begin{prooftree}
                    \AXC{$\exists x \phi(x)$}
                    \AXC{$[\phi(k)]$}
                    \dottedLine
                    \UIC{$\psi$}
                    \RightLabel{($\elim\exists$)}
                    \BIC{$\psi$}
                \end{prooftree}
                \begin{center}
                    se $k$ non compare in $\psi$, in $\phi(x)$ e in nessuna ipotesi da cui dipende $\psi$ tranne $\phi(k)$
                \end{center}
            \end{minipage}}
    \end{center}

    \begin{example}[$\forall x (Px \to Qx), \exists x Px \vdash \exists x Qx$]
        \begin{prooftree}
            \AXC{$\exists x Px$}
            \AXC{$\forall x (Px \to Qx)$}
            \RightLabel{($\elim\forall$)}
            \UIC{$Pa \to Qa$}
            \AXC{$[Pa]_1$}
            \RightLabel{($\elim\to$)}
            \BIC{$Qa$}
            \RightLabel{($\intro\exists$)}
            \UIC{$\exists x Qx$}
            \BIC{$\exists x Qx$}
        \end{prooftree}
    \end{example}

\end{frame}

\begin{frame}{Teoremi di correttezza e completezza}
    Ovviamente, anche per la logica dei predicati valgono i teoremi di correttezza e completezza forte.

    \begin{theorem}[Correttezza (forte) della deduzione naturale]
        Se $\phi_1, \ldots, \phi_n \vdash \phi$, allora $\phi_1, \ldots, \phi_n \models \phi$.
    \end{theorem}

    \begin{theorem}[Completezza (forte) della deduzione naturale]
        Se $\phi_1, \ldots, \phi_n \models \phi$, allora $\phi_1, \ldots, \phi_n \vdash \phi$.
    \end{theorem}

    \medskip
    Le dimostrazioni di questi teoremi esulano dal programma del corso.
\end{frame}

% -----------------------------------------------------------------------------
% Materiale aggiuntivo (non usato in questa versione del documento)
% -----------------------------------------------------------------------------
\iffalse
    \section{Combinare le regole di inferenza}

    \begin{frame}{Combinare le regole di inferenza}
        C'è una differenza enorme di complessità tra logica delle proposizioni e dei predicati:
        \begin{itemize}
            \item nella logica delle proposizioni, c'è un metodo sistematico per capire se una regola di inferenza è corretta o no (le tavole di verità);
            \item nella logica dei predicati non esiste un metodo simile;
            \item nella logica dei predicati anche capire se una formula è vera in una data interpretazione è difficile.
        \end{itemize}

        \medskip
        Un modo per determinare se una regola di inferenza è corretta consiste nel capire se è possibile ottenerla \alert{combinando regole di inferenza note}.
    \end{frame}

    \begin{frame}{Dimostrazioni formali (1)}
        Verifichiamo che la seguente regola di inferenza è corretta:
        \begin{center}
            \begin{inference}
                $\forall x (Px \to Qx)$\\
                $Pa$\\
                \hline
                $Qa$
            \end{inference}
        \end{center}
        Si procede come segue:
        \begin{itemize}
            \item diamo dei numeri alle formule delle premesse della regola;
            \item ad ogni passo applichiamo una regola di inferenza nota a due formule che abbiamo numerato, ottenendo una nuova formula che numeriamo con un numero più grande;
            \item se riusciamo ad ottenere in questo modo la conseguenza della regola, vuol dire che la regola è corretta.
        \end{itemize}
        Quella che otteniamo è una \alert{dimostrazione formale} della correttezza della regola.
    \end{frame}

    \begin{frame}{Dimostrazioni formali (2)}
        Verifichiamo che la seguente regola di inferenza è corretta:
        \begin{center}
            \begin{inference}
                $\forall x (Px \to Qx)$\\
                $Pa$\\
                \hline
                $Qa$
            \end{inference}
        \end{center}
        Dimostrazione:
        \begin{enumerate}
            \item $\forall x (Px \to Qx)$\\
            \item $Pa$\\
            \item $Pa \to Qa$ \hfill (1, eliminazione del quantificatore universale)\\
            \item $Qa$ \hfill (2, 3, modus ponens)
        \end{enumerate}
        Dunque la regola di inferenza è corretta.
    \end{frame}

    \begin{frame}{Sistemi di prova}
        Questo metodo di dimostrazione delle regole di inferenza, così com'è, non è sufficiente a verificare tutte le regole di inferenza corrette. Tuttavia, ne esistono vari miglioramenti, chiamati \alert{sistemi di prova}, che consentono di dimostrare qualunque inferenza corretta.

        \medskip
        Questi sistemi di prova sono almeno in parte meccanizzabili, per cui è possibile scrivere programmi che, data una regola di inferenza, provano a dimostrarne la correttezza. Si tratta dei \alert{dimostratori automatici di teoremi}, come ad esempio \href{https://vprover.github.io/}{Vampire}.

        \medskip
        L'argomento è molto complesso ed esula completamente dagli obiettivi di questo insegnamento.
    \end{frame}

\fi

\end{document}