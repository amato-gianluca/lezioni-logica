\documentclass[aspectratio=169,10pt,dvipsnames,xcolor=table,handout]{beamer}

\usepackage{bussproofs}
\EnableBpAbbreviations
%\setbeameroption{show notes}

\input{preamble.inc}

\title{Deduzione naturale}

\begin{document}

\begin{frame}
    \titlepage
\end{frame}

\begin{frame}{Conseguenza logica (1)}
    Nella lezione precedente abbiamo introdotto  il concetto di \alert{conseguenza logica} che ci permette di stabilire se una inferenza è corretta o no.


    \medskip
    Verificare se $Y$ è conseguenza logica di $X_1, \ldots, X_n$ con le tavole di verità è un metodo:
    \begin{itemize}
        \item meccanico (non richiede intuizione);
        \item generale (funziona per qualsiasi formula);
        \item molto lontano da quello che farebbe normalmente un essere umano per verificare la correttezza del ragionamento.
    \end{itemize}
\end{frame}

\begin{frame}{Conseguenza logica (2)}
    Ad esempio, consideriamo la seguente inferenza:
    \begin{center}
        \begin{inference}
            \textit{Carlo è laureato in matematica o in economia}\\
            \textit{Se Carlo è laureato in economia, allora fa il commercialista}\\
            \textit{Carlo non fa il commercialista}\\
            \hline
            \textit{Carlo è laureato in matematica}
        \end{inference}
    \end{center}
    o la sua equivalente forma logica
    \begin{center}
        \begin{inference}
            $A \vee B$\\
            $B \to C$\\
            $\neg C$\\
            \hline
            $A$
        \end{inference}
    \end{center}
\end{frame}

\begin{frame}{Conseguenza logica (3)}
    Si verifica facilmente con le tavole di verità che l'inferenza è corretta.

    \medskip
    Usare le tavole di verità corrisponde a considerare tutti i possibili ``universi alternativi'' in cui Carlo è o non è laureato in matematica, laureato in economia, commercialista, e verificare che in tutti gli ``universi alternativi'' nei quali le premesse sono vere, anche la conclusione è vera.

    \medskip
    Ma un persona non ragiona normalmente in questo modo. Piuttosto, per convincersi che l'inferenza è corretta, seguirebbe un argomento di questo tipo:

    \smallskip
    \begin{quote}
        So che se Carlo è laureato in economia allora fa il commercialista. Ma siccome non fa il commercialista, ne segue che non può essere laureato in economia (\textit{regola del modus tollens}). Tuttavia, sappiamo anche che Carlo è laureato o in matematica o in economia, e non essendo laureato in economia non può che essere laureato in matematica (\textit{regola del sillogismo disgiuntivo}).
    \end{quote}
\end{frame}

\begin{frame}{Calcoli logici}
    Vogliamo adesso introdurre un metodo di verifica della correttezza di una inferenza che sia più vicino a quello che farebbe un essere umano per verificare la validità di un ragionamento.

    \medskip
    Dall'esempio di prima, si vede che il suo argomento si basa sulla combinazione di regole di inferenza di base (come modus ponens e sillogismo disgiuntivo) che si assumono come note.

    \medskip
    I dettagli però di quali regole di inferenza siano valide e come si possano combinare per ottenere nuove inferenze non sono ovvi. Scelte diverse danno luogo a \alert{calcoli logici} diversi.

    \begin{definition}[Calcolo logico]
        Un \alert{calcolo logico} è un insieme di regole che permettono di derivare nuove formule logiche a partire da formule date.

        \smallskip
        Altri termini che sono più o meno sinonimi di calcolo logico sono \alert{sistema deduttivo} o \alert{sistema di prova}.
    \end{definition}
\end{frame}

\begin{frame}{Deduzione naturale}
    Il calcolo logico che ci interessa in questo corso è chiamato \alert{deduzione naturale}.
    \begin{itemize}
        \item Introdotta dal logico Gerhard Gentzen nella prima metà degli anni '30.
        \item È un calcolo molto vicino a quello che farebbe una persona per dimostrare la correttezza di una inferenza.
        \item È molto usato in logica e in informatica teorica.
    \end{itemize}
\end{frame}

\begin{frame}{Un sottoinsieme dei connettivi}
    Non sarebbe necessario, ma per semplicità presentiamo la deduzione naturale per un sottoinsieme dei connettivi logici:
    \[
    \land, \lor, \to, \bot
    \]
    Sappiamo che tutti gli altri sono equivalenti a combinazioni di questi:
    \begin{align*}
        \neg X &\equiv X \to \bot\\
        X \leftrightarrow Y &\equiv (X \to Y) \land (Y \to X)\\
        X \xor Y &\equiv (X \land \neg Y) \lor (\neg X \land Y)\\
        & \equiv (X \land (Y \to \bot)) \lor ((X \to \bot) \land Y)\\
        \top & \equiv X \lor \neg X
    \end{align*}
\end{frame}

\begin{frame}{Regole di inferenza}
    Le regole di inferenza della deduzione naturale hanno alcune differenze rispetto a quelle viste finora:
    \begin{itemize}
    \item le premesse si scrivono in una unica riga sopra la linea orizzontale, invece che su più righe. Ad esempio, si scrive
    \begin{center}
        \begin{inference}
            $A$ \qquad $B$\\
            \hline
            $A \land B$
        \end{inference}
        \qquad invece di \qquad
        \begin{inference}
            $A$ \\
            $B$\\
            \hline
            $A \land B$
        \end{inference}
    \end{center}

    Questo al solo scopo di permettere di combinare graficamente più regole di inferenza in una unica \alert{dimostrazione}.
    \item in alcune regole di inferenza è consentito \alert{scaricare} delle premesse (vedremo più avanti cosa significa).
    \end{itemize}
\end{frame}

\begin{frame}{Regole di introduzione ed eliminazione}
    (Quasi per) ogni connettivo logico, ci sono due tipi di regole di inferenza:
    \begin{itemize}
        \item le regole di \alert{introduzione} (che permettono di introdurre il connettivo nella conclusione);
        \item le regole di  \alert{eliminazione} (che permettono di eliminare il connettivo da una premessa).
    \end{itemize}

    \medskip
    Ogni regola ha un nome come $\elim\wedge$ o $\intro_1\vee$, composto da:
    \begin{itemize}
        \item una lettera che identifica se si tratta di una regola di introduzione ($\intro$) o di eliminazione ($\elim$);
        \item un eventuale numero che identifica una variante della regola, se ce n'è più d'una;
        \item il simbolo del connettivo logico.
    \end{itemize}
\end{frame}

\begin{frame}{Regole per la congiunzione}
    Come sempre la congiunzione è il connettivo più semplice.

    \medskip
    \begin{itemize}
        \item \textbf{introduzione}:
        \begin{center}
            \AxiomC{$X$}
            \AxiomC{$Y$}
            \RightLabel{($\intro\land$)}
            \BinaryInfC{$X \land Y$}
            \DisplayProof
        \end{center}
        \item \textbf{eliminazione}:
        \begin{center}
            \AxiomC{$X \land Y$}
            \RightLabel{($\elim_1\land$)}
            \UnaryInfC{$X$}
            \DisplayProof
            \qquad
            \AxiomC{$X \land Y$}
            \RightLabel{($\elim_2\land$)}
            \UnaryInfC{$Y$}
            \DisplayProof
       \end{center}
    \end{itemize}
\end{frame}

\begin{frame}{Dimostrazione}
    Le regole di inferenza si possono combinare per ottenere una dimostrazione:
    \begin{itemize}
            \item Una \alert{dimostrazione} (o \alert{derivazione}) si ottiene combinando tra loro le regole di inferenza in un unica \emph{struttura ad albero}, dove la conclusione di una inferenza può essere la premessa di un'altra.
        \item Le formule che non sono conclusioni di alcuna inferenza, ovvero che non hanno nessuna linea orizzontale sopra di loro, si chiamano \alert{premesse della dimostrazione}
        \item Le formule che non sono premesse di alcuna inferenza, ovvero che non hanno nessuna linea orizzontale sotto di loro, si chiamano \alert{conclusioni della dimostrazione}.
        \item In una dimostrazione può esserci una sola conclusione.
    \end{itemize}

    Se esiste una dimostrazione che ha
    \begin{itemize}
        \item $Y$ come conclusione;
        \item $X_1, \ldots, X_n$ come premesse (nessuna esclusa, tranne quelle scaricate, di cui parleremo tra un po')
    \end{itemize}
    si dice che $Y$ è dimostrabile da $X_1, \ldots, X_n$ e si scrive
    \[
    X_1, \ldots, X_n \vdash Y
    \]
\end{frame}

\begin{frame}{Esempio}
    Consideriamo la seguente dimostrazione
    \begin{prooftree}
        \AXC{$\color{red}(A \land B) \land C$}
        \RightLabel{($\elim_1\land$)}
        \UIC{$A \land B$}
        \RightLabel{($\elim_2\land$)}
        \UIC{$B$}
        \AXC{$\color{red}(A \land B) \land C$}
        \RightLabel{($\elim_2\land$)}
        \UIC{$C$}
        \RightLabel{($\intro\land$)}
        \BIC{$\color{blue}B \land C$}
    \end{prooftree}
    Le premesse della dimostrazione sono evidenziate in \textcolor{red}{rosso}, la conclusione in \textcolor{blue}{blu}.

    \medskip
    Dunque questa è una dimostrazione di $B \land C$ a partire da $(A \land B) \land C$. In simboli:
    \[
    (A \land B) \land C \vdash B \land C
    \]
\end{frame}

\begin{frame}{Regole per l'implicazione}
    \begin{itemize}
        \item \textbf{introduzione}:
        \begin{center}
            \AXC{$[X]$}
            \dottedLine
            \UIC{$Y$}
            \RightLabel{($\intro{\to}$)}
            \UIC{$X \to Y$}
            \DP
        \end{center}
        \smallskip
        La $[X]$ indica che è possibile \alert{scaricare} la premessa $X$ nella dimostrazione che porta a $Y$. Questo significa che anche se $X$ compare come premessa della dimostrazione, possiamo far finta che non ci sia. La linea tratteggiata indica che lì in mezzo c'è una dimostrazione anche molto complessa, ma non compare effettivamente in una dimostrazione.

        \smallskip
        Questa regola corrisponde alla classica modalità con cui si dimostra un'implicazione: si assume l'antecedente e si dimostra il conseguente.

        \item \textbf{eliminazione}:
        \begin{center}
            \AXC{$X \to Y$}
            \AXC{$X$}
            \RightLabel{($\elim{\to}$)}
            \BIC{$Y$}
            \DP
        \end{center}

        \smallskip
        Questa è la stessa regola che abbiamo chiamato in precedenza \emph{modus ponens}.
    \end{itemize}
\end{frame}

\begin{frame}{example}
    Verifichiamo che
    \[
    A \to B, B \to C \vdash A \to C
    \]
    \begin{prooftree}
        \AXC{$\color{red}B \to C$}
        \AXC{$\color{red}A \to B$}
        \AXC{$\color{green}[A]$}
        \RightLabel{($\elim{\to}$)}
        \BIC{$B$}
        \RightLabel{($\elim{\to}$)}
        \BIC{$C$}
        \RightLabel{($\intro{\to}$)}
        \UIC{$\color{blue}A \to C$}
    \end{prooftree}

    \smallskip
    Le premesse della dimostrazione \textbf{non scaricate} sono evidenziate in \textcolor{red}{rosso}, la conclusione in \textcolor{blue}{blu}. La formula in \textcolor{green}{verde} $A$ è stata \alert{scaricata}, quindi è come se non ci fosse ai fini di determinare quali sono le premesse della dimostrazione.
\end{frame}

\begin{frame}{Regole per la disgiunzione}
\begin{itemize}
    \item \textbf{introduzione}:
    \begin{center}
        \AXC{$X$}
        \RightLabel{($\intro_1\lor$)}
        \UIC{$X \lor Y$}
        \DP
        \qquad
        \AXC{$Y$}
        \RightLabel{($\intro_2\lor$)}
        \UIC{$X \lor Y$}
        \DP
    \end{center}
    \item \textbf{eliminazione}:
    \begin{center}
        \AXC{$X \lor Y$}
        \AXC{$[X]$}
        \dottedLine
        \UIC{$Z$}
        \AXC{$[Y]$}
        \dottedLine
        \UIC{$Z$}
        \RightLabel{($\elim\lor$)}
        \TIC{$Z$}
        \DP
    \end{center}

    \smallskip
    Questa regola corrisponde al \alert{ragionamento per casi}. Se so che $X \vee Y$, mi basta considerare separatamente i casi per cui vale $X$ ed per cui vale $Y$ (sono le assunzioni da scaricare). Se in entrambi i casi riesco a dimostrare $Z$, allora $Z$ segue da $X \vee Y$.
\end{itemize}
\end{frame}

\begin{frame}{Esempio}
    Consideriamo la dimostrazione di
    \[
    A \vee B, A \to C, B \to C \vdash C
    \]

    \begin{prooftree}
        \AXC{$\color{red}A \vee B$}
            \AXC{$\color{red}A \to C$}
            \AXC{$[A]$}
            \RightLabel{($\elim{\to}$)}
        \BIC{$C$}
            \AXC{$\color{red}B \to C$}
            \AXC{$[B]$}
            \RightLabel{($\elim{\to}$)}
        \BIC{$C$}
        \RightLabel{($\elim\lor$)}
        \TIC{$\color{blue}C$}
    \end{prooftree}

    \smallskip
    Le premesse \textbf{non scaricate} sono evidenziate in \textcolor{red}{rosso}, la conclusione in \textcolor{blue}{blu}. Le formule $A$ e $B$ sono state scaricate, quindi è come se non ci fossero ai fini di determinare quali sono le premesse della dimostrazione.
\end{frame}

\begin{frame}{Regole per il falso}
    Per il simbolo $\bot$ perdiamo la simmetria tra introduzione ed eliminazione, perché abbiamo solo la regola di eliminazione.
    \begin{itemize}
        \item \textbf{eliminazione}:
        \begin{center}
            \AXC{$\bot$}
            \RightLabel{($\elim\bot$)}
            \UIC{$X$}
            \DP
        \end{center}

        \smallskip
        Questa è una variante della regola che abbiamo chiamato ``\emph{ex falso quodlibet}'', ovvero il fatto che da una contraddizione si può dedurre qualsiasi cosa.
    \end{itemize}
\end{frame}

\begin{frame}{Esempio}
    Consideriamo la dimostrazione vista all'inizio di queste slide, ovvero
    \[
    A \vee B, B \to C, \neg C \vdash A
    \]
    Siccome non trattiamo direttamente la negazione, riscriviamola nel suo equivalente $C \to \bot$.
    \begin{prooftree}
            \AXC{$\color{red}A \vee B$}
            \AXC{$[A]$}
                    \AXC{$\color{red}C \to \bot$}
                        \AXC{$\color{red}B \to C$}
                        \AXC{$[B]$}
                        \RightLabel{($\elim{\to}$)}
                    \BIC{$C$}
                    \RightLabel{($\elim{\to}$)}
                \BIC{$\bot$}
                \RightLabel{($\elim\bot$)}
            \UIC{$A$}
        \RightLabel{($\elim\vee$)}
        \TIC{A}
    \end{prooftree}

    \smallskip
    Notare che qui l'uso di ($\elim\bot$) è essenziale. Nel ragionamento per casi, quando assumiamo che $B$ è vero arriviamo ad una contraddizione ($\bot$). Noi diremmo informalmente che il caso $B$ non è valido e dobbiamo considerare solo il caso $A$, ma nella deduzione naturale il ragionamento si esplicita in maniera leggermente diversa: diciamo che siccome siamo arrivati ad una contraddizione, allora anche nel caso $B$ possiamo comunque concludere $A$.
\end{frame}

\begin{frame}{Terzo escluso}
    Queste regole corrispondo alla \alert{logica intuizionista}, un tipo di logica che non accetta il principio del terzo escluso. Ed infatti, se ci provate, non c'è modo di dimostrare
    \[
    \vdash A \vee \neg A
    \]
    ovvero che $A \vee \neg A$ è vero senza assumere ipotesi aggiuntive.

    \medskip
    Per rendere la deduzione naturale equivalente alla logica delle proposizioni \alert{classica}, si aggiunge una regola di inferenza specifica che non ricade nella classificazione in regole di introduzione ed eliminazione.

    \medskip
    Ci sono vari modi per farlo. Uno di questi è aggiungere direttamente il terzo escluso come regola di inferenza:
    \begin{center}
        \AXC{}
        \RightLabel{(TE)}
        \UIC{$X \vee (X \to \bot)$}
        \DP
    \end{center}

    \medskip
    Si noti che è una regola di inferenza senza premessa, talvolta chiamata \alert{assioma}.
\end{frame}

\begin{frame}{Esempio: doppia negazione}
    Verifichiamo adesso che $\neg \neg A \vdash A$.
    Ricordiamo che $\neg \neg A \equiv (A \to \bot) \to \bot$.
    \begin{prooftree}
            \AXC{}
            \RightLabel{(EM)}
        \UIC{$A \vee (A \to \bot)$}
        \AXC{$[A]$}
                \AXC{$\color{red}(A \to \bot) \to \bot$}
                \AXC{$[A \to \bot]$}
                \RightLabel{($\elim{\to}$)}
            \BIC{$\bot$}
            \RightLabel{($\elim\bot$)}
        \UIC{$A$}
        \RightLabel{($\elim\vee$)}
        \TIC{$A$}
    \end{prooftree}

    \medskip
    Notare che il contrario $A \vdash \neg \neg A$ si può dimostrare anche senza il principio del terzo escluso.
    \begin{prooftree}
        \AXC{$\color{red}A$}
            \AXC{$[A \to \bot]$}
            \RightLabel{($\elim{\to}$)}
        \BIC{$\bot$}
        \RightLabel{($\intro{\to}$)}
        \UIC{$(A \to \bot) \to \bot$}
    \end{prooftree}
\end{frame}


\newcolumntype{M}[1]{>{\centering\arraybackslash}m{#1}}
\newcolumntype{N}{@{}m{0pt}@{}}

\begin{frame}{Tutte le regole}
    \begin{center}
    \begin{tabular}{r|M{6cm}|M{6cm}}
        & introduzione & eliminazione\\
        \hline
        $\land$ &
        \smallskip
        \AXC{$X$}
        \AXC{$Y$}
        \RightLabel{($\intro\land$)}
        \BIC{{$X \land Y$}}
        \DP \smallskip &
        \smallskip
        \AXC{$X \wedge Y$}
        \RightLabel{($\elim_1\land$)}
        \UIC{{$X$}}
        \DP \qquad
        \AXC{$X \wedge Y$}
        \RightLabel{($\elim_2\land$)}
        \UIC{{$Y$}}
        \DP \smallskip \\
        \hline
        $\lor$ &
        \smallskip
        \AXC{$X$}
        \RightLabel{($\intro_1\lor$)}
        \UIC{$X \lor Y$}
        \DP \qquad
        \AXC{$Y$}
        \RightLabel{($\intro_2\lor$)}
        \UIC{$X \lor Y$}
        \DP\smallskip &
        \smallskip
        \AXC{$X \lor Y$}
        \AXC{$[X]$}
        \dottedLine
        \UIC{$Z$}
        \AXC{$[Y]$}
        \dottedLine
        \UIC{$Z$}
        \RightLabel{($\elim\lor$)}
        \TIC{$Z$}
        \DP
        \smallskip
        \\
        \hline
        $\to$ &
        \smallskip
        \AXC{$[X]$}
        \dottedLine
        \UIC{$Y$}
        \RightLabel{($\intro{\to}$)}
        \UIC{$Y$}
        \DP
        \smallskip &
        \smallskip
        \AXC{$X \to Y$}
        \AXC{$X$}
        \RightLabel{($\elim{\to}$)}
        \BIC{$Y$}
        \DP
        \smallskip
        \\
        \hline
        $\bot$ &
        &
        \smallskip
        \AXC{$\bot$}
        \RightLabel{($\elim\bot$)}
        \UIC{$X$}
        \DP
        \smallskip
        \\
        \hline
        EM &
        \multicolumn{2}{c}{
            \AXC{\phantom{$Y^2$}}
            \RightLabel{(EM)}
            \UIC{$X \vee (X \to \bot)$}
            \DP
            \smallskip
        }\\
    \end{tabular}
    \end{center}
\end{frame}

\begin{frame}{Deduzione naturale e conseguenza logica}
    Cosa lega le dimostrazioni della deduzione naturale con la conseguenza logica? Noi l'abbiamo presentata come un modo di stabilire se una inferenza è corretta o no, ma per far ciò ci intereresserebbe sapere se:
    \begin{enumerate}
        \item \textbf{(essenziale, altrimenti la deduzione naturale non serve a nulla)}:\\
        è vero che se c'è una dimostrazione di $Y$ a partire da $X_1, \ldots, X_n$ (ovvero $X_1, \ldots, X_n \vdash Y$) allora $Y$ è conseguenza logica di $X_1, \ldots, X_n$ (ovvero $X_1, \ldots, X_n \models Y$) ?
        \item \textbf{(opzionale, ma sarebbe comodo)}:\\
        è vero che se $Y$ è conseguenza logica di $X_1, \ldots, X_n$ allora esiste una dimostrazione di $Y$ a partire da $X_1, \ldots, X_n$?
    \end{enumerate}
\end{frame}

\begin{frame}{Teoremi di correttezza e completezza}
    Le due proprietà di sopra sono entrambe vere, e prendono il nome di \alert{correttezza} e \alert{completezza} della deduzione naturale.
    Si parla in realtà di correttezza e completezza \alert{forti}, perché ci sono anche versioni più deboli di questi teoremi.

    \begin{theorem}[Correttezza (forte) della deduzione naturale]
        Se $X_1, \ldots, X_n \vdash Y$, allora $X_1, \ldots, X_n \models Y$.
    \end{theorem}

    \begin{theorem}[Completezza (forte) della deduzione naturale]
        Se $X_1, \ldots, X_n \models Y$, allora $X_1, \ldots, X_n \vdash Y$.
    \end{theorem}

    \medskip
    Le dimostrazioni di questo teoremi esulano dal programma del corso. Accenno solo al fatto che dimostrare il teorema di correttezza è relativamente semplice, mentre dimostrare il teorema di completezza è abbastanza difficile.
\end{frame}

\end{document}
