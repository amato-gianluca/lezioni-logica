\documentclass[10pt]{beamer}
\usepackage[italian]{babel}
\usepackage[utf8]{inputenc}
\usepackage[T1]{fontenc}
\usepackage{qrcode}

\usetheme{Madrid}
\hypersetup{colorlinks=true, linkcolor=cyan, urlcolor=cyan}

% !TeX spellcheck = it_IT

\title{Fondamenti di Informatica}
\subtitle{Presentazione del corso}
\author[Gianluca Amato]{Gianluca Amato}
\institute[CLEII]{Corso di Laurea in Economia e Informatica per l'Impresa\\A.A. 2023/24}
\date{\today}

\setbeamertemplate{part page}{
        \begin{beamercolorbox}[sep=8pt,center,wd=\textwidth]{part title}
            \usebeamerfont{part title}\insertpart\par
        \end{beamercolorbox}
}

\begin{document}

\begin{frame}
	\titlepage
\end{frame}

\begin{frame}{Presentazione del corso}
	L'insegnamento di \alert{Fondamenti di Informatica} (6 CFU) è diviso in due moduli:
	\medskip
	\begin{enumerate}
		\item \alert{Architettura e Sistemi Operativi} (3 CFU)

		      \medskip Si studiano le componenti hardware e software di un calcolatore. Le lezioni si tengono in aula informatica (aula 3), e sono divise in due canali a seconda della prima lettera del vostro cognome:

		      \medskip
		      \begin{itemize}
			      \item prof. Francesca Scozzari (A--L)
			      \item prof. Gianpiero Monaco (M--Z)
		      \end{itemize}
		      \vspace{0.3cm}

		\item \alert{Logica} (3 CFU)

		      \medskip Si studiano i fondamenti della logica. Le lezioni si tengono nelle aule tradizionali.

		      \medskip
		      \begin{itemize}
			      \item prof. Gianluca Amato
		      \end{itemize}
	\end{enumerate}
\end{frame}

\part{Il modulo di logica}

\begin{frame}
	\partpage
\end{frame}

\begin{frame}{Il docente}
	\begin{block}{Informazioni docente}
		\begin{minipage}{0.7\linewidth}
			\textbf{Gianluca Amato}\\
			Dipartimento di Economia\\
			viale Pindaro 42 (qui dove ci troviamo ora)\\
			1° piano scala verde, ingresso lato economia\\
			ufficio n. 28\\
			\medskip
			ricevimento studenti: venerdì 9:00-11:00\\
			email: \href{mailto:gianluca.amato@unich.it}{gianluca.amato@unich.it}
		\end{minipage}%
		\begin{minipage}{0.3\linewidth}
			\qrcode*{https://fad.unich.it/user/profile.php?id=1827}
		\end{minipage}
	\end{block}

	\begin{alertblock}{Attenzione!}
		Queste informazioni potrebbero essere soggette a cambiamenti (numero stanza e orario di ricevimento in particolare). Consultare il \href{https://fad.unich.it/user/profile.php?id=1827}{profilo docente su FAD} per le informazioni aggiornate (vedi anche QR code sopra).
	\end{alertblock}
\end{frame}

\begin{frame}{Materiale didattico}
	\textbf{Il libro di testo}\\

	\medskip
	Dario Palladino\\
	\href{https://www.carocci.it/prodotto/corso-di-logica-3}{Corso di logica: introduzione elementare al calcolo dei predicati (nuova edizione)}\\
	Carocci editore\\

	\bigskip
	\centering
	\qrcode*{https://www.carocci.it/prodotto/corso-di-logica-3}
\end{frame}

\begin{frame}{Sito web del corso}
	Sul sito web del corso trovate ulteriore materiale didattico quali: il programma dettagliato delle lezioni, slide, compiti degli anni precedenti, ecc\ldots

	\medskip Il sito è raggiungibile dalla piattaforma FAD (\url{https://fad.unich.it/}), oppure direttamente all'indirizzo:

	\begin{center}
		\url{https://fad.unich.it/course/view.php?id=1326} \qquad
		\qrcode*{https://fad.unich.it/course/view.php?id=1326}
	\end{center}
\end{frame}


\begin{frame}{Lezioni on-line}
	Per lo svolgimento delle lezioni on-line e la loro conservazione per visione in modalità asincrona utilizziamo la piattaforma Microsoft Teams.

	\bigskip
	\begin{center}
		\qrcode*{https://teams.microsoft.com/l/team/19\%3aNsUZdrVfV4D3arWpp18o5tqjomiCdFC4Luead6jlVb41\%40thread.tacv2/conversations?groupId=5cae59bc-99b4-4bbd-b8ae-56c953194b9f&tenantId=41f8b7d0-9a21-415c-9c69-a67984f3d0de}

		\medskip
		\href{https://teams.microsoft.com/l/team/19\%3aNsUZdrVfV4D3arWpp18o5tqjomiCdFC4Luead6jlVb41\%40thread.tacv2/conversations?groupId=5cae59bc-99b4-4bbd-b8ae-56c953194b9f&tenantId=41f8b7d0-9a21-415c-9c69-a67984f3d0de}{Logica}
	\end{center}
	\bigskip L'accesso è consentito solo agli utenti autorizzati (studenti iscritti con il programma ``PA 110 e lode'' e altre categorie individuate dal Senato Accademico nelle \href{https://www.unich.it/avvisi/disposizioni-sulle-modalita-di-erogazione-delle-attivita-didattiche-aa-20232024}{Disposizioni sulle modalità di erogazione delle attività didattiche a.a. 2023/2024}).
\end{frame}


\end{document}