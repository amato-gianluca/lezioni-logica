\documentclass[aspectratio=169,10pt,dvipsnames,handout]{beamer}

%\setbeameroption{show notes}

\newcommand{\myfbox}[2]{\tikz[baseline=(n.base)]\node(n)[alt=<1>{fill=#1!50}]{#2};}

\input{preamble.inc}

\title{Inferenze nella logica dei predicati}

\newcommand{\mcI}{\mathcal{I}}

\begin{document}

\begin{frame}
    \titlepage
\end{frame}

\begin{frame}
    \copyrightpage
\end{frame}

\begin{frame}
    \bookpage{sul Capitolo 5}
\end{frame}

\section{Dalle proposizioni ai predicati}

\begin{frame}{Forma logica per la logica dei predicati}
    \tikzstyle{every node} = [rounded corners, outer sep=0, inner sep=0.1cm]
    A livello predicativo, la forma logica si ottiene in questo modo:
    \begin{itemize}
        \item \myfbox{purple}{esiste} o \myfbox{purple}{per ogni} al posto dei quantificatori (tutti, alcuni, \ldots);
        \item \myfbox{red}{costanti individuali} al posto di individui (Napoleone, Socrate);
        \item \myfbox{green}{costanti predicative} al posto di proprietà (essere corso, essere mortale);
        \item simboli al posto dei connettivi e quantificatori in italiano.
    \end{itemize}
    \begin{center}
        \begin{inference}
            \myfbox{red}{Napoleone} \myfbox{green}{è corso}\\
            \myfbox{purple}{Tutti} \myfbox{green}{i corsi}  \myfbox{gray}{sono francesi}\\
            \hline
            \myfbox{red}{Napoleone} \myfbox{gray}{è francese}
        \end{inference}
        \qquad
        \begin{inference}
            \myfbox{red}{Socrate} \myfbox{green}{è un uomo}\\
            \myfbox{purple}{Tutti} \myfbox{green}{gli uomini}  \myfbox{gray}{sono mortali}\\
            \hline
            \myfbox{red}{Socrate} \myfbox{gray}{è mortale}
        \end{inference}

        \medskip
        {\Huge$\Downarrow$}\\[0.2cm]

        \begin{inference}
            $\myfbox{green}{P} \myfbox{red}{a}$\\
            $\myfbox{purple}{per ogni} x, \text{se } \myfbox{green}{P} x \text{ allora } \myfbox{gray}{Q}x$\\
            \hline
            $\myfbox{gray}{Q}  \myfbox{red}{a}$
        \end{inference}
        {\Huge$\Rightarrow$}
        \begin{inference}
            $P a$\\
            $\forall x (P x \to Q x)$\\
            \hline
            $Q a$
        \end{inference}
    \end{center}
\end{frame}


\begin{frame}{Costanti individuali, simboli di connettivi e quantificatori}
	Per i \alert{simboli dei connettivi} usiamo gli stessi della logica proposizionale: $\land$, $\lor$, $\lnot$, $\to$, $\leftrightarrow$, $\xor$.

	\pause\medskip
	Per i \alert{simboli dei quantificatori}:
	\begin{itemize}
		\item ``per ogni x,'' si scrive $\forall x$
		\item ``esiste x tale che,'' si scrive $\exists x$
	\end{itemize}

	\pause\medskip
	Le \alert{costanti individuali} rappresentano un individuo. Si scrivono di solito in lettere minuscole. In particolare noi useremo $a$, $b$, $c$, \ldots Per ora non tratteremo il caso di enunciati che contengono descrizioni definite.

    \pause\medskip
    Notare che anche le \alert{variabili individuali} rappresentano degli individui, e per esse utilizzeremo le lettere finali dell'alfabeto latino ($x$, $y$, etc\ldots). Di solito le variabili saranno vincolate, perché compaiono come conseguenza della formalizzazione dei quantificatori in italiano in ``per ogni'' ed  ``esiste \ldots tale che''.
\end{frame}

\begin{frame}{Costanti predicative}
	Le \alert{costanti predicative} rappresentano un predicato. Si scrivono di solito in lettere maiuscole, in particolare noi useremo $P$, $Q$, $R$, $S$, etc\ldots

	\pause\medskip
	Ad ogni costante predicativa sarà associatà un numero naturale chiamato \alert{arità} che dipende dal tipo di relazione che essa rappresenta.
	\begin{itemize}
		\item Ad esempio se $P$ rappresenta il predicato ``essere francese'' (che si applica ad un solo individuo) allora $P$ sarà di arità uno.
		\item Se invece $P$ rappresenta il predicato ``essere più grande di'' (che si applica a due individui) allora $P$ sarà di arità due.
	\end{itemize}

	\pause\medskip
	A seconda dell'arità della costante predicativa, essa accetterà un certo numero di argomenti. Ad esempio, se $P$ è una costante di arità 2, vuol dire che $P$ accetta due argomenti, per cui nelle formule comparità come $Pxy$, $Pab$, \ldots Gli argomenti sono o costanti o variabili individuali.

	\pause\medskip
	Tavolta si usa per le costanti predicative la stessa notazione che si usa per le funzioni Python, con gli argomenti tra parentesi e separate da virgole: $P(x,y)$, $P(a,b)$, \ldots
\end{frame}

\begin{frame}{Formule e inferenze}
    Le formule che è possibile scrivere nella logica dei predicati sono dette \alert{formule ben formate} (fbf) o semplicemente \alert{formule logiche}. Sono l'equivalente delle \emph{forme proposizionali} viste fin'ora.

    \medskip
    Il nostro obiettivo come sempre è capire quando una regola di inferenza è corretta. E la definizione informale di regola di inferenza corretta è la stessa della logica proposizionale:
    \begin{definition}[Regola di inferenza corretta]
        Una regola di inferenza è \alert{corretta} se e solo se ogni volta che le premesse sono vere, anche la conclusione è vera.
    \end{definition}

    \medskip
    Nella logica delle proposizioni la frase ``ogni volta che \ldots'' veniva tradotta in maniera più precisamente con ``in ogni riga della tabella tabella di verità'' o ``in ogni assegnamento di valori di verità a lettere proposizionali''. Ma cosa significa ``ogni volta che \ldots'' nella logica dei predicati?
\end{frame}

\begin{frame}{Inferenze nella logica proposizionale (1)}
    Facciamo un passo indietro. Più in generale, nella logica proposizionale, ``ogni volta che \ldots'' vuol dire ``in tutti i possibili modi di interpretare le lettere proposizionali''.

    \medskip Un formula del tipo $A \to B$ non è né vera né falsa. Ma se fissiamo una proposizione per ogni lettera proposizionale, allora possiamo tornare indietro, dalla formula tornare ad avere una proposizione, e possiamo discutere del suo valore di verità.

    \medskip
    Ad esempio, data la formula $A \to B \land C$, se fissiamo:
    \begin{itemize}
        \item $A$ = Parigi è la capitale della Francia
        \item $B$ = Roma è la capitale d'Italia
        \item $C$ = Berlino è la capitale della Finlandia
    \end{itemize}
    allora la formula diventa

    \smallskip
    \begin{proposition}
        Se Parigi è la capitale della Francia, allora Roma è la capitale d'Italia e Berlino è la capitale della Finlandia.
    \end{proposition}
    che è falsa perché l'antecedente è vero ma il conseguente è falso.
\end{frame}

\begin{frame}{Inferenze nella logica proposizionale (2)}
    Dunque, potremmo riscrivere la definizione di regola di inferenza corretta in questo modo:
    \begin{definition}[Regola di inferenza corretta]
        Una regola di inferenza è corretta se e solo se \alert{per ogni possibile interpretazione delle lettere proposizionali come proposizioni}, ogni volta che le premesse sono vere, anche la conclusione è vera.
    \end{definition}

    \medskip
    Tuttavia, ci rendiamo subito conto che la definizione ha dei problemi:
    \begin{itemize}
        \item i possibili modi di interpretare le lettere proposizionali sono infiniti: la definizione ci da quindi un modo di verificare se una regola di inferenza è corretta;
        \item rimpiazzare le lettere proposizionali con proposizioni ci manda nel reame del linguaggio naturale dove tutto è ambiguo e non è detto che siamo in grado di dire quali proposizioni sono vere e quali false.
    \end{itemize}
\end{frame}

\begin{frame}{Inferenze nella logica proposizionale (3)}
    Fortunatamente, non è veramente necessario rimpiazzare le lettere proposizionali con proposizioni, perché l'unica cosa che ci interessa di queste proposizioni è se sono vere o false.

    \medskip
    Quindi possiamo, in maniera più semplice, rimpiazzare le lettere con dei valori di verità, e otteniamo la definizione che abbiamo visto fin'ora:
    \begin{definition}[Regola di inferenza corretta]
        Una regola di inferenza è corretta se e solo se \alert{per ogni assegnamento di valori di verità alle lettere proposizionali}, ogni volta che le premesse sono vere, anche la conclusione è vera.
    \end{definition}

    \medskip
    Nella logica dei predicati, seguiremo questo percorso:
    \begin{itemize}
        \item vedremo dapprima una definizione più informale di correttezza di regola di inferenza, simile a quella della slide precedente;
        \item poi vedremo (in altre lezioni) una definizione più formale, che purtroppo non sarà così semplice come quella della logica proposizionale.
    \end{itemize}
\end{frame}

\section{Interpretazioni}

\begin{frame}{Interpretazioni}
    Per trasformare una forma proposizionale in una proposizione, è sufficiente rimpiazzare ogni lettera con una proposizione. La cosa non è così semplice nella logica dei predicati. Chiamiamo \alert{interpretazione} il processo con cui passiamo dalle formule ben formate alle proposizioni.

    \begin{definition}[Interpretazione --- informale]
        Una \alert{interpretazione} consiste di:
        \begin{itemize}
            \item un insieme di individui, chiamato \alert{dominio di quantificazione} o \alert{universo del discorso}, da intendersi come i valori su cui le costanti individuali possono essere interpretate e i quantificatori possono essere valutati;
            \item un assegnamento di un individuo del dominio per ogni costante individuale;
            \item un assegnamento di un predicato per ogni costante predicativa, tale che:
                  \begin{itemize}
                      \item se la costante predicativa è di arità 1 allora il predicato dovra essere una proprietà (``essere pari'');
                      \item se la costante predicativa è di arità 2 allora il predicato dovra essere una relazione binaria (``essere più grande di'');
                      \item in maniera simile per costanti proposizionali di arità maggiore.
                  \end{itemize}
        \end{itemize}
    \end{definition}
\end{frame}

\begin{frame}{Esempi di interpretazione (1)}
    Consideriamo le formule
    \begin{enumerate}
        \item $Pa$
        \item $\forall x (P x \to R x)$
        \item $\forall x \exists y Qxy$
    \end{enumerate}
    e vediamo un paio di interpretazioni, completamente diverse, per queste formule.

    \medskip
    Si noti come il dominio influenzi il tipo di predicati che possiamo scegliere per $P$, $Q$ ed $R$. Se il dominio sono le regioni italiane, il predicato $P$ non può essere ``essere pari'' perché non ha senso chiedersi se una regione è pari o dispari.
\end{frame}

\begin{frame}{Esempi di interpretazione (2)}
    \begin{example}[Una interpretazione]
        Consideriamo la seguente interpretazione:
        \begin{itemize}
            \item dominio = i numeri interi relativi
            \item $a$ = 5
            \item $P$ = essere pari
            \item $R$ = essere divisibile per 4
            \item $Q$ = essere maggiore di
        \end{itemize}
        Le formule diventano:
        \begin{enumerate}
            \item ``\prop{5 è pari}\/'' (falso)
            \item ``\prop{per ogni numero intero $x$, se $x$ è pari allora è divisibile per 4}\/'' (falso), che si può rendere in maniera più naturale in italiano con ``\prop{tutti i numeri pari sono divisibili per 4}\/''
            \item ``\prop{per ogni numero intero $x$ esiste un numero intero $y$ tale che $x$ è maggiore di $y$}\/'' (vero)
        \end{enumerate}
    \end{example}
\end{frame}

\begin{frame}{Esempi di interpretazione (3)}
    \begin{example}[Importanza del dominio]
        Consideriamo una nuova interpretazione che differisce dalla precedente solo per il dominio: invece dei numeri interi relativi, consideriamo i numeri naturali (interi positivi).

        \medskip
        Le formule diventano:
        \begin{enumerate}
            \item ``\prop{5 è pari}\/'' (falso)
            \item ``\prop{per ogni numero naturale $x$, se $x$ è pari allora $x$ è divisibile per 4}\/'' (falso), che si può rendere in maniera più naturale in italiano con ``\prop{tutti i numeri naturali pari sono divisibili per 4}\/''
            \item ``\prop{per ogni numero naturale $x$ esiste un numero naturale $y$ tale che $x$ è maggiore di $y$}\/'' (falso)
        \end{enumerate}
        Si noti che l'ultima formula è adesso falsa, mentre prima era vera, perché non esiste un numero naturale minore di 0.
    \end{example}
\end{frame}

\begin{frame}{Esempi di interpretazione (4)}
    \begin{example}[Un'altra interpretazione]
        Consideriamo ora una interpretazione completamente diversa:
        \begin{itemize}
            \item dominio = le regioni italiane
            \item $a$ = Lazio
            \item $P$ = essere a statuto speciale
            \item $R$ = essere bagnata dal mare
            \item $Q$ = essere confinante via terra con
        \end{itemize}
        e otteniamo
        \begin{enumerate}
            \item ``\prop{il Lazio è a statuto speciale}'' (falso);
            \item ``\prop{per ogni regione italiana $x$, se $x$ è a statuto speciale allora $x$ è bagnata dal mare}'', che si può rendere in maniera più naturale in italiano con ``\prop{tutte le regioni italiane a statuto speciale sono bagnate dal mare}'' (falso, perché la Val d'Aosta non è bagnata dal mare);
            \item ``\prop{per ogni regione italiana $x$ esiste una regione italiana $y$ tale che $x$ è confinante con $y$}\/'' (falso, perché le isole non confinano con altre regioni).
        \end{enumerate}
    \end{example}
\end{frame}

\section{Inferenze e formule valide}

\begin{frame}{Regole di inferenza corrette}
    Possiamo quindi definire la correttezza di una regola di inferenza come:
    \begin{definition}[Regola di inferenza corretta]
        Una regola di inferenza è corretta se e solo se \alert{tutte le interpretazioni che rendono vere le premesse rendono vere anche la conclusione}.
    \end{definition}

    \medskip
    Notare che questa definizione non ci consente veramente di stabilire in maniera meccanima se una regola di inferenza è corretta, perché le possibili interpretazioni sono infinite e non possiamo certo provarle tutte!

    \medskip
    Tuttavia, ci consente quantomeno di verificare facilmente che una regola di inferenza non è corretta: basta trovare una singola interpretazione che rende vere le premesse e falsa la conclusione.
\end{frame}

\begin{frame}{Falsificare una regola di inferenza (1)}
    Consideriamo la seguente regola:
    \begin{center}
        \begin{inference}
            $\forall x (Px \to Qx)$\\
            $Qa$\\
            \hline
            $Pa$
        \end{inference}
    \end{center}
    e verifichiamo che non è corretta. Notare che anche ad occhio è sospetta, perché assomiglia alla ``fallacia dell'affermazione del conseguente''
    \begin{center}
        \begin{inference}
            $A \to B$\\
            $B$\\
            \hline
            $A$
        \end{inference}
    \end{center}
    ma con un quantificatore.
\end{frame}

\begin{frame}{Falsificare una regola di inferenza (2)}
    Consideriamo la seguente interpretazione:
    \begin{itemize}
        \item dominio = numeri naturali
        \item $a$ = 2
        \item $P$ = essere divisibile per 4
        \item $Q$ = essere pari
    \end{itemize}
    Le premesse sono:
    \begin{itemize}
        \item ``\prop{per ogni numero naturale $x$, se $x$ è divisibile per 4 allora $x$ è pari}\/'', ovvero ``\prop{tutti i numero divisibili per 4 sono pari}\/'' (vero)
        \item ``\prop{2 è pari}\/'' (vero)
    \end{itemize}
    e la conclusione è
    \begin{itemize}
        \item ``\prop{2 è divisibile per 4}\/'' (falso)
    \end{itemize}
    Quindi la regola non è corretta.
\end{frame}

\begin{frame}{Falsificare una regola di inferenza (3)}
    Notare che esistono anche interpretazioni per cui la conclusione è vera. Ad esempio, basta rimpiazzare $a$ con 8 nella precedente interpretazione. Questo però non rende la regola di inferenza corretta!!!

    \medskip
    Perché sia corretta, la conclusione deve essere vera in tutte le interpretazioni che rendono vere le premesse: basta trovarne una che non è così (come quella della slide precedente) e la regola non è corretta.
\end{frame}

\begin{frame}{Esempio di regola corretta}
    Un esempio di regola corretta che non deriva da quelle proposizionali è:
    \begin{example}[Regola di eliminazione del quantificatore universale]
            \begin{center}
            \begin{inference}
                $\forall x Px$\\
                \hline
                $Pa$
            \end{inference}
        \end{center}
    \end{example}

    Per verificare che è corretta, dovremmo far vedere che, in tutte le possibili interpretazioni, se la premessa è corretta lo è anche la conseguenza. Non possiamo veramente considerare tutte le interpretazioni, ma possiamo scrivere una  \alert{dimostrazione}.

    \begin{proof}
        Consideriamo una qualunque interpretazione in cui $\forall x Px$ è vera. Vuol dire che qualunque oggetto del dominio di quantificazione gode della proprietà $x$. Ma anche $a$ corrisponde ad un ogetto del dominio di interpretazione, e quindi anche $Pa$ è vera.
    \end{proof}
\end{frame}

\begin{frame}{Regole di inferenza derivate dalla logica proposizionale}
    Consideriamo ora:
    \begin{center}
        \begin{inference}
            $(\forall x Px) \to (\exists y Qy)$\\
            $\forall x Px$\\
            \hline
            $\exists y Qy$
        \end{inference}
    \end{center}

    La regola è corretta perché è semplicemente un caso particolare (\alert{istanza}) del \emph{modus ponens}
    \begin{center}
        \begin{inference}
            $A \to B$\\
            $A$\\
            \hline
            $B$
        \end{inference}
    \end{center}
    ottenuto con le sostituzioni $A \mapsto \forall x Px$ a $B \mapsto \exists y Qy$.

    \begin{theorem}
        Tutte le regole della logica dei predicati che sono istanze di regole proposizionali corrette sono anch'esse corrette.
    \end{theorem}
\end{frame}


\begin{frame}{Formule valide (1)}
    Non è solo il concetto di regola di inferenza corretta che si estende immediatamente al caso predicativo. Anche il concetto di \alert{tautologia} si estende in maniera naturale, sebbene con un altro nome.

    \begin{definition}
        Una formula è detta \alert{valida} se e solo se è vera in tutte le interpretazioni.
    \end{definition}

    \begin{example}
    La formula $Pa \lor \neg Pa$ è valida, perché è vera in tutte le interpretazioni. Infatti data una qualunque interpretazione:
        \begin{itemize}
            \item se $Pa$ è vera, allora la formula di sopra è $V \lor \neg V = V$;
            \item se $Pa$ è falsa, allora la formula di sopra è $F \lor \neg F = F \lor V = V$
        \end{itemize}
        Notare che a questa conclusione siamo giunti con una dimostrazione matematica, non riempiendo un tabella in maniera meccanica come per la logica proposizionale.
    \end{example}
\end{frame}

\begin{frame}{Formule valide (2)}
    Notare che $Pa \lor \neg Pa$ è una istanza del terzo escluso ($A \lor \neg A$) con la sostituzione $A \mapsto Pa$. Questo è un metodo standard per ottenere formule valide:
    \begin{theorem}
        Tutte le formule ben formate ottenute come istanza di tautologie sono formule valide.
    \end{theorem}
    \begin{example}
        Le seguenti formule sono valide:
        \begin{itemize}
            \item $(\forall x Px) \lor \neg (\forall x Px)$ --- istanza di $A \lor \neg A$
            \item $(\forall x \exists y Qxy) \to (\forall x \exists y Qxy)$ --- istanza di $A \to A$
            \item  $(\forall x Px) \lor Qa \iff Qa \lor (\forall x Px)$ --- istanza di $A \iff A$
        \end{itemize}
    \end{example}
\end{frame}

\begin{frame}{Formule valide (3)}
    Esistono anche formule valide che non sono istanze di tautologie.
    \begin{example}
        La formula $(\forall x Px) \to Pa$ è una formula valida. Infatti:
        \begin{itemize}
            \item se $\forall x Px$ è vera, $P$ è vero per qualunque elemento del dominio, quindi a maggior ragione è vero per l'elemento $a$, chiunque esso sia;
            \item se $\forall x Px$ è falsa, allora l'implicazione è sicuramente vera per definizione.
        \end{itemize}

        \medskip
        D'altronde è ovvio perché sappiamo che abbiamo già visto che
        \begin{inference}
            $\forall x Px$\\
            \hline
            $Pa$
        \end{inference}
        è una inferenza corretta.
    \end{example}
\end{frame}

\begin{frame}{Equivalenza e conseguenza logica}
    Infine, il concetto di equivalenza e di conseguenza logica si trasporta in maniera naturale alla logica dei predicati.
    \begin{definition}[Equivalenza logica]
        Due formule si dicono \alert{equivalenti} se e solo se sono vere nelle stesse interpretazioni.
    \end{definition}

    \begin{definition}[Conseguenza logica]
        Una formula $\phi$ è \alert{conseguenza logica} delle formule $\phi_1, \ldots, \phi_n$ se e solo se in ogni interpretazione in cui sono vere $\phi_1, \ldots, \phi_n$ è vera anche $\phi$. Si scrive:
    \end{definition}

    \medskip
    Notare che indichiamo una generica formula con le lettere greche $\phi$, $\psi$ e simili. Non usiamo $X$, $Y$ e $Z$ come nella logica proposizionale perché si confonderebbero con le variabili individuali $x$, $y$ e $z$.
\end{frame}


\begin{frame}{Equivalenze derivate dalla logica proposizionale}

    Come abbiamo visto con altri concetti introdotti nelle slide precedenti, ogni istanza di una equivalenza logica proposizionale è anche una equivalenza logica predicativa.
    \begin{example}
    Data l'equivalenza logica $A \land B \equiv B \land A$, otteniamo l'equivalenza logica predicativa
        \[
            (\forall x Px) \land \exists x(Qx \land \forall y Rxy) \equiv \exists x(Qx \land \forall y Rxy) \land (\forall x Px)
        \]
    tramite la sostituzione $A \mapsto \forall x Px$ e  $B \mapsto \exists x(Qx \land \forall y Rxy)$.
    \end{example}

    Se indichiamo con le lettere $\phi_1$ ed $\phi_2$ delle formule generiche, possiamo scrivere l'equivalenza logica predicativa come
    \[
    \phi_1 \land \phi_2 \equiv \phi_2 \land \phi_1 \enspace.
    \]
\end{frame}

\section{Ancora sui quantificatori limitati}

\begin{frame}{Quantificatore universale limitato (1)}
    Consideriamo le proposizioni:
    \[
        \text{\prop{Tutti sono mortali}} \qquad \text{\prop{Tutti gli uomini sono mortali}}
    \]
    La prima afferma che qualunque individuo è mortale, mentre nella seconda la proprietà di essere mortali è ristretta solo ad alcuni individui (gli uomini) perché magari altri (gli dei?) non lo sono.

    \medskip
    Sappiamo che il second tipo prende il nome di \emph{quantificatore limitato} e si può tranquillamente formalizzare usando il quantificatore standard. Se:
    \begin{itemize}
        \item $Mx$ sta per ``$x$ è mortale''
        \item $Ux$ sta per ``$x$ è un essere umano''
    \end{itemize}
    la forma logica di ``\prop{Tutti gli uomini sono mortali}\/'' diventa
    \[
        \forall x (Ux \to Mx)
    \]
    Letteralmente: ``\prop{qualunque individuo $x$ prendiamo, se $x$ è un uomo allora $x$ è mortale\/}''.
\end{frame}

\begin{frame}{Quantificatore universale limitato (2)}
    Per comodità, però, in certi ambiti si usa una notazione più compatta per il quantificatore universale limitato. Ad esempio, in matematica.
    \begin{example}[Quantificatore limitato da un insieme]
        Se vogliamo dire
        \begin{proposition}
            Tutti gli elementi dell'insieme $A$ sono positivi
        \end{proposition}
        possiamo scrivere
        \[
            \forall x (x \in A \to x > 0)
        \]
        ma è più comodo scrivere
        \[
            \forall x \in A \, (x > 0)
        \]
        La seconda formula è solo un modo compatto di scrivere la prima.
    \end{example}
\end{frame}

\begin{frame}{Quantificatore esistenziale limitato (1)}
    Anche il quantificatore esistenziale ha una variante limitata. La proposizione:
    \begin{proposition}
        Esiste un uomo che sa volare
    \end{proposition}
    vuol dire non solo che esiste un qualche individuo che vola, ma che possiamo scegliere quell'individuo in modo che sia un uomo.

    \medskip
    Il quantificatore limitato, anche in questo caso, lo si può ottenere da quello standard. Se
    \begin{itemize}
        \item $Mx$ sta per ``$x$ è mortale''
        \item $Vx$ sta per ``$x$ sa volare''
    \end{itemize}
    allora la forma logica della proposizione è
    \[
    \exists x (Mx \land Vx)
    \]
    \textcolor{Blue}{Attenzione!} La formula corretta $\exists x (Mx \land Vx)$ con l'uso della \alert{congiunzione} e non  $\exists x (Mx \to Vx)$ che vorrebbe dire tutt'altro!
\end{frame}

\begin{frame}{Quantificatore esistenziale limitato (2)}
    Per comodità, in certi ambiti si usa una notazione più compatta per il quantificatore esistenziale limitato. Ad esempio, in matematica.
    \begin{example}[Quantificatore limitato da un insieme]
        Se vogliamo dire
        \begin{proposition}
            Esiste un elemento dell'insieme $A$ positivo
        \end{proposition}
        possiamo scrivere
        \[
            \exists x (x \in A \land x > 0)
        \]
        ma è più comodo scrivere
        \[
            \exists x \in A \, (x > 0)
        \]
        La seconda formula è solo un modo compatto di scrivere la prima.
    \end{example}
\end{frame}


\note{Forse sarebbe meglio fare degli esempi del quantificatore limitato con le tessere, come nella sezione sottostante. In questo modo si potrebbe spiegare meglio il perché l'implicazione è il connettivo giusto da usare nel $\forall$ ma non nell'$\exists$.}

\section{Equivalenze e conseguente logiche notevoli}


\begin{frame}{Negazione e quantificatori (1)}
    Due equivalenze logiche molto importanti (e molto ricorrenti nei test di logica) sono le seguenti:
    \[
        \neg \exists x \phi \equiv \forall x \neg \phi \qquad \neg \forall x \phi \equiv \exists x \neg \phi
    \]
    \begin{example}
        Supponiamo che l'universo del discorso sia costituito dagli esseri umani. Allora, se $\phi$ è la funzione proposizionale ``\prop{$x$ sa volare}\/'' :
        \begin{itemize}
            \item $\neg \exists x \phi$ è ``\prop{non esiste un essere umano che sa volare}\/''
            \item $\forall x \neg \phi$ è ``\prop{tutti gli esseri umani non sanno volare}\/''
        \end{itemize}
        ed entrambi vogliono dire la stessa cosa (sebbene il secondo in italiano sia innaturale)
    \end{example}
    Questa equivalenza  ci consente quindi di spostare la negazione dentro o fuori un quantificatore, pur di cambiare il tipo del quantificatore.
\end{frame}

\begin{frame}{Negazione e quantificatori (2)}
    Notare che questa equivalenza ci consente anche di dimostrare che i due quantificatori sono ridondanti. Ne basterebbe solo uno. Infatti:
    \begin{align*}
        \exists x \phi  & \equiv \neg \neg \exists x \phi \tag{doppia negazione}\\
                        & \equiv \neg \forall x \neg \phi \tag{equivalenza precedente}
    \end{align*}
    \begin{example}
        ``\prop{Esiste un gatto nero}\/'' è equivalente a ``\prop{non è vero che tutti i gatti non sono neri}''.
    \end{example}
    Ovviamente si preferisce continuare ad usare due quantificatori perché rendono il discorso più chiaro.
\end{frame}

\begin{frame}{Negazione e quantificatori (3)}
    Naturalmente, anche il $\forall$ si può riscrivere usando negazione ed $\exists$:
    \begin{align*}
        \forall x X  & \equiv \neg \neg \forall x X \tag{doppia negazione}\\
                        & \equiv \neg \exists x \neg X \tag{equivalenza precedente}
    \end{align*}
    \begin{example}
        ``\prop{Tutti i gatti sono neri}\/'' è equivalente a ``\prop{non esiste un gatto che non è nero}\/''.
    \end{example}
\end{frame}

\begin{frame}{Negazione e quantificatori (4)}
    Queste equivalenze assomigliano un po' alla legge di De Morgan, che riportiamo qui sotto:
    \[
        \neg (A \land B) \equiv \neg A \lor \neg B \qquad \qquad \neg (A \lor B) \equiv \neg A \land \neg B
    \]

    \medskip
    Effettivamente, il quantificatore $\forall$ è una specie di ``congiunzione'' di formule, mentre il quantificatore $\exists$ è una specie di ``disgiunzione'' di formule. Ad esempio, se l'universo del discorso sono i numeri naturali, in maniera non del tutto rigorosa possiamo scrivere:
    \begin{gather*}
        \forall x Px \equiv  P0 \land P1 \land P2 \land P3 \ldots\\
        \exists x Px \equiv  P0 \lor P1 \lor P2 \lor P3 \ldots
    \end{gather*}
    Applicando De Morgan:
    \begin{align*}
    \neg(\forall x Px) & \equiv \neg (P0 \land P1 \land P2 \land P3 \ldots) \tag{equivalenza vista prima}\\
    & \equiv (\neg P0) \lor (\neg P1) \lor (\neg P2) \lor \ldots \tag{legge di De Morgan}\\
    & \equiv \exists x (\neg Px) \tag{equivalenza vista prima}
    \end{align*}
\end{frame}

\begin{frame}{Negazione e quantificatori limitati}
    Le equivalenze precedenti valgono anche per i quantificatori limitati. Ad esempio:
    \begin{align*}
        \neg \forall x \in A \, (x > 0) & \equiv \neg \forall x (x \in A \to x > 0) \tag{rimozione del quantificatore limitato}\\
       & \equiv \exists x \neg (x \in A \to x > 0) \tag{equivalenza precedente}\\
    & \equiv \exists x \neg \neg(x \in A \land \neg  (x > 0)) \tag{poiché $A \to B \equiv \neg(A \land \neg B)$}\\
    & \equiv \exists x (x \in A  \land \neg ( x > 0)) \tag{doppia negazione}\\
        & \equiv \exists x \in A \, \neg (x > 0) \tag{reintroduzione del quantificatore limitato}
    \end{align*}
    In generale:
    \begin{gather*}
    \neg \forall x (Y \to X) \equiv \exists x (Y \land \neg X)\\
    \neg \exists x (Y \land X) \equiv \forall x (Y \to \neg X)
    \end{gather*}
\end{frame}

\begin{frame}{Interpretazione dei prossimi esempi}
    Nei prossimi esempi utilizzeremo la seguente interpretazione:
    \begin{itemize}
        \item il dominio sarà un insieme di tessere che possono avere varie forme e colori;
        \item il predicato unario $R$ sta per ``essere rotondo'';
    \item il predicato unario $G$ sta per ``essere giallo''.
    \end{itemize}
    Il numero di tessere, la loro forma e colore verrà rappresentato graficamente. Ad esempio
    \[
    \begin{tikzpicture}[minimum size=1cm]
        \node (1) [draw, circle, fill=yellow] at (0,0) {1};
        \node (2) [draw, rectangle, fill=yellow, right=of 1]  {2};
        \node (3) [draw, rectangle, fill=red, right=of 2]  {3};
    \end{tikzpicture}
    \]
    è una interpretazione con tre tessere, la prima gialla e tonda, la seconda gialla e rettangolare, la terza rossa e rettangolare.
\end{frame}

\begin{frame}{Quantificatore universale e congiunzione (1)}
    Consideriamo la formula  $\forall x (Rx \land Gx)$, che in italiano si può tradurre con
    \begin{proposition}
        tutte le tessere sono rotonde e gialle
    \end{proposition}
    Dovrebbe essere evidente che questa è equivalente a $(\forall x Rx) \land (\forall x Gx)$, ovvero
    \begin{proposition}
        tutte le tessere sono rotonde e tutte le tessere sono gialle
    \end{proposition}
    La seguente interpretazione soddisfa sia $\forall x (Rx \land Gx)$ che $(\forall x Rx) \land (\forall x Gx)$:
    \[
    \begin{tikzpicture}[minimum size=1cm]
        \node (1) [draw, circle, fill=yellow] at (0,0) {1};
        \node (2) [draw, circle, fill=yellow, right=of 1]  {2};
        \node (3) [draw, circle, fill=yellow, right=of 2]  {3};
    \end{tikzpicture}
    \]
    mentre la seguente non soddisfa nessuna delle due
    \[
        \begin{tikzpicture}[minimum size=1cm]
            \node (1) [draw, circle, fill=red] at (0,0) {1};
            \node (2) [draw, circle, fill=yellow, right=of 1]  {2};
            \node (3) [draw, circle, fill=yellow, right=of 2]  {3};
        \end{tikzpicture}
    \]
\end{frame}

\begin{frame}{Quantificatore universale e congiunzione (2)}
    Esiste una configurazione di tessere che soddisfi $\forall x (Rx \land Gx)$  ma non $(\forall x Rx) \land (\forall x Gx)$, o viceversa ?
    \pause

    \medskip
    Vi renderete presto conto che non è possibile, perché le due formule sono equivalenti e quindi sono vere esattamente nelle stesse interpretazioni !
    \pause

    \medskip
    Più in generale, è vera la seguente equivalenza, che è una sorta di proprietà distributiva del quantificatore universale rispetto alla congiunzione:
    \[
        \forall x (\phi_1 \land \phi_2) \equiv (\forall x \phi_1) \land (\forall x \phi_2)
    \]
\end{frame}

\begin{frame}{Quantificatore universale e disgiunzione (1)}
    Ci chiediamo adesso se valga anche l'equivalenza
    \[
    \forall x (Rx \lor Gx) \equiv (\forall x Rx) \lor (\forall x Gx)
    \]
    ovvero
    \begin{proposition}
        tutte le tessere sono rotonde o gialle\\
        $\equiv$\\
        tutte le tessere sono rotonde o tutte le tessere sono gialle
    \end{proposition}

    \pause
    Tuttavia, la seguente interpretazione soddisfa $\forall x (Rx \lor Gx)$ ma non $(\forall x Rx) \lor (\forall x Gx)$:
    \[
    \begin{tikzpicture}[minimum size=1cm]
        \node (1) [draw, circle, fill=red] at (0,0) {1};
        \node (2) [draw, rectangle, fill=yellow, right=of 1]  {2};
        \node (3) [draw, circle, fill=yellow, right=of 2]  {3};
    \end{tikzpicture}
    \]
\end{frame}

\begin{frame}{Quantificatore universale e disgiunzione (2)}
    È vero però (vi ricordo che $\models$ è il simbolo di conseguenza logica) che
    \[
    (\forall x Rx) \lor (\forall x Gx) \models \forall x (Rx \lor Gx)
    \]
    Infatti se $(\forall x Rx) \land (\forall x Gx)$, allora o tutte le tessere sono rotonde, oppure sono tutte gialle (o anche entrambe le cose). In ogni caso, ogni tessara è o rotonda o gialla.

    \medskip
    Una interpretazione che:
    \begin{itemize}
    \item rende vera $(\forall x Rx) \lor (\forall x Gx)$
    \item rende falsa $\forall x (Rx \lor Gx)$
    \end{itemize}
    non esiste!

    \medskip In generale è vero che:
    \[
    (\forall x \phi_1) \lor (\forall x \phi_2) \models \forall x (\phi_1 \lor \phi_2)
    \]
\end{frame}

\begin{frame}{Quantificatore esistenziale e congiunzione}
    Se rimpiazziamo il quantificatore universale con l'esistenziale, otteniamo proprietà simili. Intanto
    \[
        \exists x (Rx \lor Gx) \equiv (\exists x Rx) \lor (\exists x Gx)
    \]
    ovvero
    \begin{proposition}
        esiste una tessera rotonda o gialla\\
        $\equiv$\\
        esiste una tessera rotonda o esiste una tessera gialla
    \end{proposition}
    In generale è vero
    \[
        \exists x (\phi_1 \lor \phi_2) \equiv (\exists x \phi_1) \lor (\exists x \phi_2)
    \]
\end{frame}

\begin{frame}{Quantificatore esistenziale e disgiunzione (1)}
    Consideriamo adesso
    \begin{itemize}
        \item $\exists x (Rx \land Gx)$, ovvero ``\prop{esiste una tessera rotonda e gialla}\/''
        \item $(\exists x Rx) \land (\exists x Gx)$, ovvero ``\prop{esiste una tessera rotonda ed esiste una tessera gialla}\/''
    \end{itemize}
    Sono equivalenti ?

    \medskip La risposta è no. Questa interpretazione rende vera la seconda ma non la prima:
    \[
    \begin{tikzpicture}[minimum size=1cm]
        \node (1) [draw, circle, fill=red] at (0,0) {1};
        \node (2) [draw, rectangle, fill=yellow, right=of 1]  {2};
        \node (3) [draw, circle, fill=green, right=of 2]  {3};
    \end{tikzpicture}
    \]
    Infatti è vero che esiste una tessera rotonda (la 1 e la 3) e che esiste una tessera gialla (2), ma non esiste una tessera che è rotonda e gialla contemporaneamente.
\end{frame}

\begin{frame}{Quantificatore esistenziale e disgiunzione (2)}
    È vero però che
    \[
        \exists x (Rx \land Gx) \models (\exists x Rx) \land (\exists x Gx)
    \]
    Infatti se $\exists x (Rx \land Gx)$, vuol dire che possiamo trovare una tessera che è rotonda e gialla allo stesso tempo. Ovviamente, grazie a questa tessera, saranno vere sia $\exists x Rx$ che $\exists x Gx$.

    \medskip
    Una interpretazione che:
    \begin{itemize}
        \item rende vera $\exists x (Rx \land Gx)$
        \item rende false $(\exists x Rx) \land (\exists x Gx)$
    \end{itemize}
    non esiste!

    \medskip In generale è vero che:
    \[
        \exists x (\phi_1 \land \phi_2) \models (\exists x \phi_1) \land (\exists x \phi_2)
    \]
\end{frame}

\begin{frame}{Esercizi consigliati}
	Esercizi da 5 in poi del Capitolo 5.
\end{frame}

\end{document}
