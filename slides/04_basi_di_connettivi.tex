\documentclass[aspectratio=169,10pt,dvipsnames,xcolor=table,handout]{beamer}

%\setbeameroption{show notes}

\input{preamble.inc}

\title{Basi di connettivi}

\begin{document}

\begin{frame}
    \titlepage
\end{frame}

\section{Determinare la fp di un tavola di verità}

\begin{frame}{Determinare la fp di un tavola di verità (1)}
    Consideriamo la seguente tavola di verità con una colonna non definita:

    \medskip
    \begin{quote}
        \begin{tabular}{c|c|c||c}
            $A$ & $B$ & $C$ & ? \\
            \hline
            F   & F   & F   & F \\
            F   & F   & V   & F \\
            F   & V   & F   & F \\
            F   & V   & V   & F \\
            V   & F   & F   & V \\
            V   & F   & V   & V \\
            V   & V   & F   & V \\
            V   & V   & V   & F
        \end{tabular}
    \end{quote}

    \medskip
    Ci chiediamo se esiste una forma proposizionale che la genera.
\end{frame}

\begin{frame}[t]{Determinare la fp di un tavola di verità (2)}
        Effettivamente, esiste una procedura sistematica per ottenere la fp cercata:
        \begin{columns}[t]
            \column{0.45\textwidth}
            \begin{center}
                \small
                \begin{tabular}{c|c|c||c|l}
                    $A$ & $B$ & $C$ & ? \\
                    \hline
                    F   & F   & F   & F \\
                    F   & F   & V   & F \\
                    F   & V   & F   & F \\
                    F   & V   & V   & F \\
                    \only<2>{\cellcolor{red}}V & \only<2>{\cellcolor{red}}F & \only<2>{\cellcolor{red}}F & \only<2>{\cellcolor{red}}V & \uncover<3->{$A \wedge \neg B \wedge \neg C$} \\
                    \only<2>{\cellcolor{red}}V & \only<2>{\cellcolor{red}}F & \only<2>{\cellcolor{red}}V & \only<2>{\cellcolor{red}}V & \uncover<3->{$A \wedge \neg B \wedge C$}\\
                    \only<2>{\cellcolor{red}}V & \only<2>{\cellcolor{red}}V & \only<2>{\cellcolor{red}}F & \only<2>{\cellcolor{red}}V & \uncover<3->{$A \wedge B \wedge \neg C$}\\
                    V   & V   & V   & F
                \end{tabular}
            \end{center}%
            \column{0.55\textwidth}
            \begin{enumerate}
                \item<2-> individuare le righe il cui risultato è V
                \item<3-> per ogni riga, scrivere la congiunzione di tutte le variabili: quelle cha hanno valore V appaiono direttamente, quelle cha hanno valore F appaiono negate
                \item<4-> scrivere la disgiunzione di tutte le congiunzioni ottenute al passo precedente, ottenendo
                \small
                \[
                    (A \wedge \neg B \wedge \neg C) \vee (A \wedge \neg B \wedge C) \vee (A \wedge  B \wedge \neg C)
                \]
            \end{enumerate}
        \end{columns}
    \uncover<5>{
    Una formula di questo tipo è detta in \emph{forma normale disgiuntiva} (FND). Questo esempio ci mostra anche che tutte le fp ne hanno una equivalente in FND.}
\end{frame}

\begin{frame}{Perché funziona?}
    Ognuna delle tre congiunzioni è vera solo in corrispondenza della riga che l'ha generate (nell'esempio qui sotto, quella dello stesso colore).

    \medskip
    L'or di queste congiunzioni è vera in corrispondenza di tutte le righe in cui almeno una delle congiunzioni è vera, cioè in corrispondenza di tutte le righe in cui la colonna ? è V.
    \begin{center}
        \small
        \begin{tabular}{c|c|c||c|c|c|c|c}
            $A$ & $B$ & $C$ & ? & \cellcolor{red}{$P=A \wedge \neg B \wedge \neg C$} &  \cellcolor{blue}{$Q=A \wedge \neg B \wedge  C$} &  \cellcolor{green}{$R=A \wedge  B \wedge \neg C$} &  $P \vee Q \vee R$\\
            \hline
            F   & F   & F   & F & F & F & F & F \\
            F   & F   & V   & F & F & F & F & F \\
            F   & V   & F   & F & F & F & F & F \\
            F   & V   & V   & F & F & F & F & F \\
            \cellcolor{red}V  & \cellcolor{red}F   & \cellcolor{red}F   & \cellcolor{red}V & V & F & F & V \\
            \cellcolor{blue}V   & \cellcolor{blue}F   & \cellcolor{blue}V   & \cellcolor{blue}V & F & V & F & V \\
            \cellcolor{green}V   & \cellcolor{green}V   & \cellcolor{green}F   & \cellcolor{green}V & F & F & V & V \\
            V   & V   & V   & F & F & F & F & F
        \end{tabular}
    \end{center}
\end{frame}

\begin{frame}{Semplificare la FND (1)}
    La formula ottenuta in questo modo può essere notevolmente più complessa del necessario. Consideriamo la seguente tavola di verità.

    \medskip
    \begin{quote}
        \begin{tabular}{c|c||c}
            $A$ & $B$ & ? \\
            \hline
            F   & F  & V \\
            F   & V  & V \\
            V   & F  & V \\
            V   & V  & F \\
        \end{tabular}
    \end{quote}

    \pause
    \medskip Guardando la tavola, è possibile vedere che la formula cercata è \pause $\neg (A \wedge B)$.

    \pause
    \medskip Ma applicando il meccanismo visto prima, otteniamo $(\neg A \wedge \neg B) \vee (\neg A \wedge B) \vee (A \wedge \neg B)$

    \pause
    \medskip Tuttavia, applicando le equivalenze logiche, possiamo semplificarla.
\end{frame}

\begin{frame}{Semplificare la FND (2)}
    \[
        \begin{array}{rl@{\hspace{0.8cm}}l}
                          & (\neg A \wedge \neg B) \vee (\neg A \wedge B) \vee (A \wedge \neg B) & \pause \text{(raccoglimento)} \\
            \pause \equiv & \left(\neg A \wedge (\neg B \vee B)\right) \vee (A \wedge \neg B)           & \pause \text{(terzo escluso)}            \\
            \pause \equiv & \left(\neg A \wedge \top\right) \vee (A \wedge \neg B)           & \pause \text{(elemento neutro di $\wedge$)}            \\
            \pause \equiv & \neg A \vee (A \wedge \neg B)                     & \pause \text{(prop. distributiva)}          \\
            \pause \equiv & (\neg A \vee A) \wedge (\neg A \vee \neg B)        & \pause \text{(terzo escluso)}      \\
            \pause \equiv & \top \wedge (\neg A \vee \neg B)        & \pause \text{(elemento neutro di $\wedge$)}      \\
            \pause \equiv & \neg A \vee \neg B       & \pause \text{(De Morgan*)}      \\
            \pause \equiv & \neg (A \wedge B)
        \end{array}
    \]
    Nel passaggio marcato da $*$ è stata usata la legge di De Morgan $\neg (A \wedge B) \equiv \neg A \vee \neg B$ al contrario, da destra verso sinistra.
\end{frame}

\begin{frame}{Revisitare le equivalente notevoli}
    Usando la provedura vista, possiamo facilmente di ottenere le equivalenze che ci consentono di riscrivere i connettivi $\to$, $\xor$  in termini di $\wedge$, $\vee$ a $\neg$. Si tratta di:
    \begin{gather*}
        A \to B \equiv \neg A \vee B \\
        A \xor B \equiv (\neg A \wedge B) \vee (A \wedge \neg B)
    \end{gather*}
    che abbiamo già visto nella lezione sulle equivalenze notevoli.
\end{frame}

\section{Basi di connettivi}

\begin{frame}{Basi di connettivi}
    Abbiamo visto come qualunque fp abbia una equivalente forma normale disgiuntiva. Le FND usano solo i connettivi $\neg$, $\wedge$ e $\vee$. Dunque:
    \begin{theorem}
        Tutte le forme proposizionali ne hanno una equivalente che usa solo $\neg$, $\wedge$ e $\vee$.
    \end{theorem}

    \begin{definition}
        Un insieme di connettvi che consentono di esprimere tutte le fp è detto \emph{base di connettivi}.
    \end{definition}

    Il teorema di sopra si può quindi riformulare nel modo seguente:
    \begin{theorem}
        $\{ \neg, \wedge, \vee\}$ è una base di connettivi.
    \end{theorem}
\end{frame}

\begin{frame}{$\wedge$ e $\neg$ costituiscono una base di connettivi (1)}
    In realtà, $\neg$ e $\wedge$ da soli sono sufficienti per avere una base di connettivi. Infatti è facile verificare che
    \begin{theorem}
    \[
        A \vee B \equiv \neg(\neg A \wedge \neg B)
    \]
    \end{theorem}
    \begin{proof}
    Potremmo verificarlo con le tavole di verità, ma proviamo a farlo per via algebrica.
    Dalla legge di De Morgan sappiamo che
    \[
        \neg(A \vee B) \equiv \neg A \wedge \neg B
    \]
    Applicando una negazione ad ambo i membri, si ottiene
    \[
        \neg \neg (A \vee B) \equiv \neg(\neg A \wedge \neg B)
    \]
    e per la legge di doppia negazione, si ottiene quanto voluto.
    \end{proof}
\end{frame}

\begin{frame}{$\wedge$ e $\neg$ costituiscono una base di connettivi (2)}
    Poiché
    \[A \vee B \equiv \neg(\neg A \wedge \neg B)\]
    quando abbiamo un \fp che contiene $\neg$, $\wedge$ e $\vee$, possiamo rimpiazzare tutte le $\vee$ usando la formula di sopra, ottenendo una formula che usa solo $\neg$ e $\wedge$.

    \begin{example}
        Sia data la \fp
        \[
            \neg A \vee (B \wedge C)
        \]
        Applicando l'equivalenza di prima, otteniamo:
        \[
            \neg (\neg \neg A \wedge \neg (B \wedge C))
        \]
        che volendo possiamo semplificare in
        \[
            \neg (A \wedge \neg (B \wedge C))
        \]
    \end{example}
\end{frame}

\begin{frame}{$\wedge$ e $\neg$ costituiscono una base di connettivi (2)}
    Possiamo concludere con il seguente:
    \begin{theorem}
        $\{ \neg, \wedge\}$ è una una base di connettivi.
    \end{theorem}
    \begin{proof}
        Quanlunque formula si può riscrivere in una equivalente che usa solo $\neg$, $\wedge$ e $\vee$. Ma poiché $A \vee B \equiv \neg(\neg A \wedge \neg B)$, possiamo rimpiazzare tutte le disgiunzioni con formule che usano solo $\neg$ e $\wedge$. Quindi $\{ \neg, \wedge\}$ è una base di connettivi.
    \end{proof}
    Ovviamente, in maniera analoga, usando il fatto che $A \vee B = \neg(\neg A \wedge \neg B)$, si dimostra anche
    \begin{theorem}
        $\{ \neg, \vee\}$ è una una base di connettivi.
    \end{theorem}
\end{frame}

\begin{frame}{$\wedge$ e $\neg$ non costituiscono una base di connettivi}
    Ci si potrebbe chiedere se $\wedge$ e $\vee$ costituiscono una base di connettivi, facendo a meno della negazione. La risposta è negativa.

    \begin{theorem}
        $\{ \wedge, \vee\}$ non è una base di connettivi.
    \end{theorem}
    \begin{proof}
        Sia $X$ una \fp che usa solo i connettivi $\wedge$ e $\vee$. Poiché $F \vee F = F$ e $F \wedge F= F$, segue che se tutte le lettere proposizionali hanno valore di verità falso, la \fp X ha anch'essa valore di verità falso. Quindi $X$ non consente di esprimere tutte le \fp.
    \end{proof}

    Come conseguenza, ovviamente, neanche $\wedge$  o $\vee$ da soli costituiscono una base.
\end{frame}

\begin{frame}{$\to$ e $\bot$ costituiscono una base di connettivi}
    \begin{theorem}
        $\{ \to, \bot\}$ è una base di connettivi.
    \end{theorem}
    \begin{proof}
    Basta trovare formule equivalenti a $\neg A$ ed a $A \vee B$ che usano solo $\to$ e $\bot$. È facile verificare che
    \begin{gather*}
        \neg A \equiv A \to \bot\\
        A \vee B \equiv \neg A \to B \equiv (A \to \bot) \to B
    \end{gather*}
    \end{proof}

    Invece $\to$ da sola non è sufficiente. Siccome $V \to V = V$, qualunque formula che usa solo $\to$ ha valore di verità V quando tutte le lettere proposizionali hanno valore di verità V. Quindi $\to$ da solo non è sufficiente per esprimere tutte le \fp.
\end{frame}

\begin{frame}{Il connettivo nand}
    Un connettivo nuovo è il connettivo ``\alert{nand}'', abbreviazione di ``not and''. In italiano corrisponde al cosiddetto ``\alert{o incompatibile}''. Non esiste un simbolo universalmente accettto per il nand, ma spesso si usa $\uparrow$.

    \medskip
    La sua tavola di verità è la negazione della congiunzione:
    \begin{center}
        \begin{tabular}{c|c||c}
            $A$ & $B$ & $A \uparrow B$ \\
            \hline
            F   & F   & V         \\
            F   & V   & V         \\
            V   & F   & V         \\
            V   & V   & F
        \end{tabular}
    \end{center}

    \begin{example}
        A tavola vale la regola: ``o si mangia o si parla'', vuol dire che non si possono fare entrambe le cose, ma qualunque altra combinazione è ammessa (anche non fare nessuna delle due).
    \end{example}
\end{frame}

\begin{frame}{$\uparrow$ costituisce da solo una base di connettivi}
    \begin{theorem}
        $\{ \uparrow \}$ è una base di connettivi.
    \end{theorem}
    \begin{proof}
     Siccome sappiamo che $\{\neg, \wedge\}$ è una base, basa mostrare che sia $\neg$ che $\wedge$ si possono esprimere usando solo $\uparrow$.

    \medskip
    Si ottiene facilmente, con le tavole di verità:
    \begin{gather*}
        \neg A \equiv A \uparrow A\\
        A \wedge B \equiv \neg(A \uparrow B) \equiv (A \uparrow B) \uparrow (A \uparrow B)
    \end{gather*}
    \end{proof}
\end{frame}

\begin{frame}{Il connettivo nor}
    Un altro connettivo nuovo è il connettivo ``\alert{nor}'', abbreviazione di ``not or''. In italiano corrisponde a ``\alert{non è né \ldots né}'' (in inglese, appunto, ``neither \ldots nor''). Anche in questo caso non esiste un simbolo universalmente accettto per il nor, ma spesso si usa $\downarrow$. La sua tavola di verità è la negazione della disgiunzione:

    \medskip
    \begin{center}
        \begin{tabular}{c|c||c}
            $A$ & $B$ & $A \downarrow B$ \\
            \hline
            F   & F   & V         \\
            F   & V   & F        \\
            V   & F   & F         \\
            V   & V   & F
        \end{tabular}
    \end{center}

    \begin{example}
        La proposizione ``Non è né sabato né domenica'' è vera solo se le proposizioni ``è sabato'' ed ``è domenica'' sono false.
    \end{example}
\end{frame}

\begin{frame}{$\downarrow$ costituisce da solo una base di connettivi}
    \begin{theorem}
        $\{ \downarrow \}$ è una base di connettivi.
    \end{theorem}
    \begin{proof}
     Siccome sappiamo che $\{\neg, \vee\}$ è una base, basa mostrare che sia $\neg$ che $\vee$ si possono esprimere usando solo $\downarrow$.

    \medskip
    Si ottiene facilmente, con le tavole di verità:
    \begin{gather*}
        \neg A \equiv A \downarrow A\\
        A \vee B \equiv \neg(A \vee B) \equiv (A \downarrow B) \downarrow (A \downarrow B)
    \end{gather*}
    \end{proof}
\end{frame}

\end{document}
