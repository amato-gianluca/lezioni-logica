\documentclass[aspectratio=169,10pt,dvipsnames,handout]{beamer}

%\setbeameroption{show notes}

\input{preamble.inc}

\title{Le forme proposizionali}

\begin{document}

\begin{frame}
    \titlepage
\end{frame}

\section{Forme proposizionali}

\begin{frame}{Forme proposizionali}
    Usando le lettere $A$, $B$, $C$ etc..., i connettivi $\neg$, $\wedge$, $\vee$, $\xor$, $\to$, $\iff$ e le parentesi tonde, è possibile costruire delle espressioni chiamate \alert{forme proposizionali} (abbreviata in fp) o anche \alert{formule proposizionali}  Indicheremo le fp con le lettere $X$, $Y$, etc\ldots
    \begin{example}[Forme proposizionali]
        \centering
        $\neg(A \wedge \neg B)$ \qquad \qquad
        $(A \to B) \wedge (B \to A)$ \qquad \qquad
        $A \vee \neg (B \wedge C)$
    \end{example}
    \pause
    È possibile ottenere delle proposizioni rimpiazzando le lettere proposizionali:
    \begin{example}[dalle forme proposizionali alle proposizioni]
        Considerate la forma proposizionale $(A \vee B) \to C$. Allora se
        \begin{itemize}
            \item $A={}$ Carlo è laureato in economia
            \item $B={}$ Carlo è laureato in informatica
            \item $C={}$ Carlo è ammesso al concorso
        \end{itemize}
        otteniamo
        \begin{itemize}
            \item Se Carlo è laureato in economia o in informatica, allora è ammesso al concorso.
        \end{itemize}
    \end{example}
\end{frame}

\begin{frame}{Calcolo del valore di verità di una forma proposizionale}
    Possiamo pensare alle forme proposizionali come espressioni algebriche che usano i valori vero e falso invece dei numeri. Se assegnamo un valore di verità alle lettere, è possibile calcolare il valore di verità della forma proposizionale.

    \begin{example}[Calcolo valore di verità]
        Consideriamo la forma proposizionale $(A \vee B) \to C$. Se
        \begin{itemize}
            \item $A={}$ vero
            \item $B={}$ falso
            \item $C={}$ vero
        \end{itemize}
        allora
        \begin{itemize}
            \item $(A \vee B) \to C = (\text{vero} \vee \text{falso}) \to \text{vero} = \pause \text{vero} \to \text{vero} = \pause \text{vero}$
        \end{itemize}
    \end{example}
\end{frame}

\begin{frame}{Parentesi e precedenze (1)}
    Le parentesi, come nelle espressioni algebriche, servono ad evitare le ambiguità:
    \begin{example}[Ambiguità]
        Consideriamo:
        \begin{itemize}
            \item $A \to B \wedge C$
            \item $A=\text{falso}$, $B=\text{vero}$ $C=\text{falso}$
        \end{itemize}.
        La \fp ha due interpretazioni, con due risultati diversi:
        \begin{itemize}
            \item $(A \to B) \wedge C  = (\text{falso} \to \text{vero}) \wedge \text{falso} = \pause \text{vero} \wedge \text{falso} = \pause \text{falso}$
            \item  \pause $A \to (B \wedge C)  = \text{falso} \to (\text{vero} \wedge \text{falso}) = \pause \text{falso} \to \text{falso} = \pause \text{vero}$
        \end{itemize}
    \end{example}
\end{frame}

\begin{frame}{Precedenze}
    Per evitare di mettere troppe parentesi, si usano le precedenze:
    \begin{itemize}
        \item $\neg$ ha la precedenza più alta
        \item $\wedge$, $\vee$ e $\xor$ hanno precedenza intermedia
        \item $\to$ e $\iff$ hanno precedenza più bassa
    \end{itemize}
    \begin{example}[Precedenze]
        \begin{itemize}
            \item $A \to B \wedge C$  si legge $A \to (B \wedge C)$
            \item $\neg A \wedge B \to C$ si legge $((\neg A) \wedge B) \to C$.
        \end{itemize}
    \end{example}

    \medskip
    Alcune parentesi sono comunque necessarie:
    \begin{itemize}
        \item $A \vee B \wedge C$ è ambigua, occorre scrivere $(A \vee B) \wedge C$ oppure $A \vee (B \wedge C)$.
    \end{itemize}

    \medskip
    Inoltre, spesso qualche parentesi in più non guasta:
    \begin{itemize}
        \item Invece di $\neg A \wedge B \to C$ meglio scrivere $(\neg A \wedge B) \to C$.
    \end{itemize}
\end{frame}

\begin{frame}{Tavole di verità per \fp (1)}
    Quando si calcola il valore verità di una \fp per tutti i possibili assegnamenti di valori di verità alle variabili proposizionali, si ottiene una \alert{tavola di verità}.

    \begin{example}[Tavola di verità di $\neg A \vee B$]
        \begin{center}
            \begin{tabular}{c|c||c}
                $A$ & $B$ & $\neg A \vee B$ \\
                \hline
                F   & F   & \only<2->V      \\
                F   & V   & \only<3->V      \\
                V   & F   & \only<4->F      \\
                V   & V   & \only<5->V
            \end{tabular}
        \end{center}
    \end{example}
\end{frame}

\tikzstyle{inline}=[remember picture, baseline, anchor=base, thick, inner sep=0, outer sep=0]


\begin{frame}{Tavole di verità per \fp (2)}

    Se la \fp è complessa, è possibile inserire eventuali colonne aggiuntive per formule intermedie.

    \begin{example}[Tavola di verità di $(A \to B) \to (\neg A \to B)$]

        \begin{center}
            \begin{tabular}{c|c||c|c|c|c}
                $A$ & \tikzmark{varb}$B$
                    & \tikz[inline]{\node (atob) {$A \to B$}}
                    & \tikz[inline]{\node (nega) {$\neg A$}}
                    & \tikz[inline]{\node (negatob1) {$\neg A$}} $\to$ \tikz[inline]{\node (negatob2) {$B$}}
                    & (\tikz[inline]{\node (final1) {$(A \to B)$}} $\to$
                \tikz[inline]{\node (final2) {$(\neg A \to B)$}}                                                                                                 \\
                \hline
                F   & F                                                                                      & \only<2->V & \only<3->V & \only<4->F & \only<5->F \\
                F   & V                                                                                      & \only<2->V & \only<3->V & \only<4->V & \only<5->V \\
                V   & F                                                                                      & \only<2->F & \only<3->F & \only<4->V & \only<5->V \\
                V   & V                                                                                      & \only<2->V & \only<3->F & \only<4->V & \only<5->V
            \end{tabular}
        \end{center}
        \onslide<4>{
            \begin{tikzpicture}[overlay, remember picture, thick]
                \node[fit=(negatob1), rectangle, rounded corners, draw, red, outer sep=0.2] {} ;
                \node[fit=(negatob2), rectangle, rounded corners, draw, red, outer sep=0.2] {} ;
                \draw[red] (nega) edge [->,bend left] (negatob1);
                \draw[red] (pic cs:varb) edge [->,bend right] (negatob2);
            \end{tikzpicture}
        }%
        \onslide<5>{
            \begin{tikzpicture}[overlay, remember picture, thick]
                \node[fit=(final1), rectangle, rounded corners, draw, red, outer sep=0.2] {} ;
                \node[fit=(final2), rectangle, rounded corners, draw, red, outer sep=0.2] {} ;
                \draw[red] (atob) edge [->,bend left] (final1);
                \draw[red] ($(negatob1)!0.5!(negatob2)$) edge [->,bend right] (final2);
            \end{tikzpicture}
        }%
    \end{example}
    \onslide<6->{Nella tabella di sopra ho voluto creare una colonna per ogni sottoformula, ma ovviamente ciò non è necessario}
\end{frame}

\begin{frame}{Tavole di verità per \fp (2)}
    È possibile disegnare la tavola di verità anche di \fp con tre o più lettere proposizionali: basta non sbagliare nel considerare tutte le possibili combinazioni di valori di verità.

    \begin{example}[Tavola di verità di $(A \to \neg C) \to (B \to \neg C)$]
        \begin{center}
            \begin{tabular}{c|c|c||c|c|c|c}
                $A$ & $B$ & $C$ & $\neg C$   & $A \to \neg C$ & $B \to \neg C$ & $(A \to \neg C) \to (B \to \neg C)$ \\
                \hline
                F   & F   & F   & \only<2->V & \only<3->V     & \only<4->V     & \only<5->V                          \\
                F   & F   & V   & \only<2->F & \only<3->V     & \only<4->V     & \only<5->V                          \\
                F   & V   & F   & \only<2->V & \only<3->V     & \only<4->V     & \only<5->V                          \\
                F   & V   & V   & \only<2->F & \only<3->V     & \only<4->F     & \only<5->F                          \\
                V   & F   & F   & \only<2->V & \only<3->V     & \only<4->V     & \only<5->V                          \\
                V   & F   & V   & \only<2->F & \only<3->F     & \only<4->V     & \only<5->V                          \\
                V   & V   & F   & \only<2->V & \only<3->V     & \only<4->V     & \only<5->V                          \\
                V   & V   & V   & \only<2->F & \only<3->F     & \only<4->F     & \only<5->V
            \end{tabular}
        \end{center}
    \end{example}
\end{frame}

\begin{frame}{Enumerare tutte le possibili combinazioni (1)}
    Come si fa a non dimenticarsi nessuna delle possibili combinazioni di valori di verità per le lettere proposizionali?
    \begin{itemize}
        \item con $n$ lettere proposizionali, ci sono $2^n$ combinazioni possibili;
        \item partendo dalla colonna più a destra, si alternano una ($2^0$) F ed una V vino ad arrivare a $2^n$ righe;
        \item quindi si passa alla colonna immediatamente a sinistra, alternando due ($2^1$) V e due F;
        \item quindi, passando alle colonne sempre più a sinistra si alternato 4 ($2^2$), 8 ($2^3$), \ldots V ed altrettante F fino a riempire tutte le colonne delle lettere proposizionali.
    \end{itemize}
\end{frame}

\begin{frame}{Enumerare tutte le possibili combinazioni (2)}
    \begin{example}
        \begin{columns}
            \column{0.5\textwidth}
            \small
            \centering
            \begin{tabular}{c|c|c|c}
                A          & B          & C          & D          \\
                \hline
                \only<5->F & \only<4->F & \only<3->F & \only<2->F \\
                \only<5->F & \only<4->F & \only<3->F & \only<2->V \\
                \only<5->F & \only<4->F & \only<3->V & \only<2->F \\
                \only<5->F & \only<4->F & \only<3->V & \only<2->V \\
                \only<5->F & \only<4->V & \only<3->F & \only<2->F \\
                \only<5->F & \only<4->V & \only<3->F & \only<2->V \\
                \only<5->F & \only<4->V & \only<3->V & \only<2->F \\
                \only<5->F & \only<4->V & \only<3->V & \only<2->V \\
                \only<5->V & \only<4->F & \only<3->F & \only<2->F \\
                \only<5->V & \only<4->F & \only<3->F & \only<2->V \\
                \only<5->V & \only<4->F & \only<3->V & \only<2->F \\
                \only<5->V & \only<4->F & \only<3->V & \only<2->V \\
                \only<5->V & \only<4->V & \only<3->F & \only<2->F \\
                \only<5->V & \only<4->V & \only<3->F & \only<2->V \\
                \only<5->V & \only<4->V & \only<3->V & \only<2->F \\
                \only<5->V & \only<4->V & \only<3->V & \only<2->V
            \end{tabular}
            \column{0.5\textwidth}
            \begin{itemize}
                \item<2-> colonna D: una F ed una V
                \item<3-> colonna C: due F e due V
                \item<4-> colonna B: quattro F e quattro V
                \item<5-> colonna A: otto F e otto V
            \end{itemize}
        \end{columns}
    \end{example}
\end{frame}

\begin{frame}{Enumerare tutte le possibili combinazioni (3)}
    Un metodo alternativo per non dimenticare nessuna combinazione è quella di riempiere le righe una alla volta, contando in binario, ma con F al posto di 0 e V al posto di 1.
    \begin{example}
        \begin{columns}
            \column{0.5\textwidth}
            \small
            \centering
            \begin{tabular}{c|c|c|c}
                A          & B          & C          & D          \\
                \hline
                \only<2->F & \only<2->F & \only<2->F & \only<2->F \\
                \only<3->F & \only<3->F & \only<3->F & \only<3->V \\
                \only<4->F & \only<4->F & \only<4->V & \only<4->F \\
                \multicolumn{4}{c}{\ldots}
            \end{tabular}
            \column[b]{0.5\textwidth}
            \begin{itemize}
                \item<2-> prima riga: $0 = 0000_2$
                \item<3-> seconda riga: $1 = 0001_2$
                \item<4-> terza riga: $2 = 0010_2$
            \end{itemize}
        \end{columns}
    \end{example}
\end{frame}

\note{Esercizi 2.20, 2.23, 2.26, 3.2}

\section{Tautologie ed equivalenze}

\begin{frame}{Tautologie}
    \begin{definition}[Tautologia]
        Si chiama \alert{tautologia} una forma proposizionale che è sempre vera, indipendentemente dal valore di verità delle lettere proposizionali che in essa compaiono.
    \end{definition}

    \begin{example}[Legge di Pierce]
        $((A \to B) \to A) \to A$ è una tautologia.
        \begin{center}
            \begin{tabular}{c|c||c|c|c}
                $A$ & $B$ & $A \to B$  & $(A \to B) \to A$ & $((A \to B) \to A) \to A$ \\
                \hline
                F   & F   & \only<2->V & \only<3->F        & \only<4->V                \\
                F   & V   & \only<2->V & \only<3->F        & \only<4->V                \\
                V   & F   & \only<2->F & \only<3->V        & \only<4->V                \\
                V   & V   & \only<2->V & \only<3->V        & \only<4->V
            \end{tabular}
        \end{center}
    \end{example}
\end{frame}

\begin{frame}{Tautologie notevoli: non contraddizione}
    Il \alert{principio di non contraddizione} afferma che non è possibile che una proposizione sia contemporaneamente vera e falsa.

    \begin{theorem}[Principio di non contraddizione]
        $\neg (A \wedge \neg A)$ è una tautologia.
        \begin{center}
            \begin{tabular}{c||c|c}
                $A$ & $A \wedge \neg A$ & $\neg (A \wedge \neg A)$ \\
                \hline
                F   & \only<2->F        & \only<3->V               \\
                V   & \only<2->F        & \only<3->V               \\
            \end{tabular}
        \end{center}
    \end{theorem}
    \begin{example}[in italiano]
        ``\textit{Carlo è alto e non è alto}'' è sicuramente falsa, anche se non sappiamo chi è Carlo.
    \end{example}


    \medskip
    \onslide<4->{
        Il principio di non contraddizione è un principio fondante di praticamente qualunque tipo di logica esistente.
    }
\end{frame}

\begin{frame}{Tautologie notevoli: terzo escluso}
    Il \alert{principio del terzo escluso}, noto anche col nome latino \emph{tertium non datur}, afferma sostanzialmente che ogni proposizione è vera o falsa.

    \begin{theorem}[Principio del terzo escluso]
        La \fp $A \vee \neg A$ è una tautologia.
        \begin{center}
            \begin{tabular}{c||c|c}
                $A$ & $A \vee \neg A$ \\
                \hline
                F   & \only<2->V      \\
                V   & \only<2->V      \\
            \end{tabular}
        \end{center}
    \end{theorem}
    \begin{example}[in italiano]
        ``\textit{Carlo è alto oppure non è alto}'' è sicuramente vera, anche se non sappiamo chi è Carlo.
    \end{example}

    \medskip
    \onslide<3->{
        Il principio del terzo escluso, a differenze di quello di non contraddizione, non è universalmente accettato: esistono molti sistemi logici in cui non vale, anche se noi non ce ne occupiamo.
    }
\end{frame}

\begin{frame}{Contraddizione}
    \begin{definition}[Contraddizione]
        Si chiama \alert{contraddizione} una forma proposizionale che è sempre falsa, indipendentemente dal valore di verità delle lettere proposizionali che in essa compaiono.
    \end{definition}
    \begin{example}[Contraddizioni]
        Sono contraddizioni:
        \[ A \wedge \neg A \qquad \qquad A \iff \neg A \qquad \qquad A \xor A \]
    \end{example}
    \begin{theorem}
        $X$ è una tautologia se e solo se $\neg X$ è una contraddizione, e viceversa
    \end{theorem}
\end{frame}

\begin{frame}{Equivalenza logica}
    \begin{definition}[Equivalenza logica]
        Due fp si dicono \alert{logicamente equivalenti} se hanno lo stesso valore di verità per ogni possibile assegnamento di valori di verità alle lettere proposizionali.

        \medskip
        Detto in altri termini, due \fp sono equivalenti se hanno la stessa tavola di verità.
    \end{definition}


    \begin{example}
        Consideriamo le \fp $\neg A \iff B$ e $A \iff \neg B$. Mostriamo una unica tavola di verità che mostra entrambe le \fp assieme
        \begin{center}
            \begin{tabular}{c|c||c|c|c|c}
                $A$ & $B$ & $\neg A$ & $\neg B$ & $\neg A \iff B$ & $A \iff \neg B$ \\
                \hline
                F   & F   & V        & V        & F               & F               \\
                F   & V   & V        & F        & V               & V               \\
                V   & F   & F        & V        & V               & V               \\
                V   & V   & F        & F        & F               & F
            \end{tabular}
        \end{center}
        Poiché le colonne $\neg A \iff B$ e $A \iff \neg B$ sono uguali, le due \fp sono logicamente equivalenti.
    \end{example}
\end{frame}

\begin{frame}{Notazione per l'equivalenza logica}
    Nel seguito, useremo molto il concetto di equivalenza logica, per cui introduciamo il simbolo \alert{$\equiv$} per indicarla.

    \medskip
    Per dire che $\neg A \iff B$ e $A \iff \neg B$ sono equivalenti, scriveremo $\neg A \iff B \equiv A \iff \neg B$

    \begin{block}{Attenzione!}
        Il simbolo $\equiv$ \textbf{non è un connettivo} e $\neg A \iff B \equiv A \iff \neg B$ non è una forma proposizionale.  Il simbolo $\equiv$ è solo una notazione abbreviata per dire che due \fp sono equivalenti.
    \end{block}
\end{frame}

\begin{frame}{Tautologie ed equivalenza logica (1)}
    \begin{theorem}
        Due \fp $P$ e $Q$ sono logicamente equivalenti se e solo se la \fp $P \iff Q$ è una tautologia.
    \end{theorem}

    \begin{example}
        Le \fp $P = \neg A \vee B$ e $Q = A \to B$ sono equivalenti. Infatti
        \begin{center}
            \begin{tabular}{c|c||c|c|c}
                $A$ & $B$ & $\neg A$ & $\neg A \vee B$ & $A \to B$ \\
                \hline
                F   & F   & V        & V               & V         \\
                F   & V   & V        & V               & V         \\
                V   & F   & F        & F               & F         \\
                V   & V   & F        & V               & V
            \end{tabular}
        \end{center}
        Le due colonne $X = \neg A \vee B$ e $Y = A \to B$ sono uguali, quindi $X \equiv Y$.

        \smallskip Cosa possiamo dire di $X \iff Y$ ?
    \end{example}
\end{frame}

\begin{frame}{Tautologie ed equivalenza logica (2)}
    \begin{theorem}
        Due \fp $X$ e $Y$ sono logicamente equivalenti se e solo se la \fp $X \iff Y$ è una tautologia.
    \end{theorem}

    \begin{example}
        Aggiungiamo una nuova colonna alla tavola.
        \begin{center}
            \begin{tabular}{c|c||c|c|c|c}
                $A$ & $B$ & $\neg A$ & $\neg A \vee B$ & $A \to B$ & $(\neg A \vee B) \iff (A \to B)$ \\
                \hline
                F   & F   & V        & V               & V         & V                                \\
                F   & V   & V        & V               & V         & V                                \\
                V   & F   & F        & F               & F         & V                                \\
                V   & V   & F        & V               & V         & V
            \end{tabular}
        \end{center}
        La colonna $X \iff Y = (\neg A \vee B) \iff (A \to B)$ è sempre vera, perché i due valori di verità che la doppia implicazione collega sono uguali.

        \medskip Dunque $X \iff Y$ è una tautologia.
    \end{example}
\end{frame}

\note{Esercizio 3.6}

\section{Equivalenze logiche notevoli}

\begin{frame}{Equivalenze logiche notevoli}
    L'equivalenza logica è assimilabile all'uguaglianza tra espressioni algebriche:
    \begin{itemize}
        \item siccome $x + y = y + x$, dovunque c'è una somma possiamo invertire gli addendi senza cambiare il risultato.
        \item allo stesso modo, siccome $A \vee B \equiv B \vee A$, possiamo invertire l'ordine delle proposizioni a cui si applica l'or senza cambiare il risultato (ovvero la tavola di verità)
    \end{itemize}

    % \medskip
    % Le equivalenze logiche ci danno quindi un modo di operare sulle \fp come fossero espressioni algebriche. Vediamo adesso alcune equivalenze notevoli, confrontandole con analoghe proprietà delle espressioni algebriche.

    \medskip
    Da ora in poi, \textbf{utilizzeremo nelle \fp anche i due simboli $\top$ e $\bot$}, che stanno per una formula sempre vera (tautologia) ed una sempre falsa (contraddizione). Non usiamo $V$ ed $F$ per evitare confusione con le lettere proposizionali.

    \medskip
    Vediamo quindi alcune equivalenze logiche notevoli. Quando possibile, faremo un parallelo con equivalenze notevoli delle espressioni algebriche, secondo la corrispondenza:
    \[
        \vee \Rightarrow + \qquad \wedge \Rightarrow \cdot \qquad \neg \Rightarrow 1 - {} \qquad \top \Rightarrow 1 \qquad \bot \Rightarrow 0
    \]
\end{frame}

\begin{frame}{Proprietà di congiunzione e disgiunzione}
    \centering
    \begin{tabular}{c|c|c}
        \textbf{logica}                                             & \textbf{nome proprietà} & \textbf{algebra}                               \\
        \hline
        $A \wedge B \equiv B \wedge A$                              & commutativa             & $x \cdot y = y \cdot x$                        \\
        $A \wedge (B \wedge C) \equiv (A \wedge B) \wedge C$        & associativa             & $x \cdot (y \cdot z) = (x \cdot y) \cdot z$    \\
        $A \wedge A \equiv A$                                       & idempotenza             & \sout{$x \cdot x = x$}                         \\
        $A \wedge \top \equiv A$                                    & elemento neutro         & $x \cdot 1 = x$                                \\
        $A \wedge \bot \equiv \bot$                                 & elem. assorbente        & $x \cdot 0 = 0$                                \\
        \hline
        \hline
        $A \vee B \equiv B \vee A$                                  & commutativa             & $x + y = y + x$                                \\
        $A \vee (B \vee C) \equiv (A \vee B) \vee C$                & associativa             & $x + (y + z) = (x + y) + z$                    \\
        $A \vee A \equiv A$                                         & idempotenza             & \sout{$x + x = x$}                             \\
        $A \vee \bot \equiv A$                                      & elemento neutro         & $x + 0 = x$                                    \\
        $A \vee \top \equiv \top$                                   & elem. assorbente        & \sout{$x + 1 = 1$}                             \\
        \hline
        \hline
        $A \wedge (B \vee C) \equiv (A \wedge B) \vee (A \wedge C)$ & distributiva            & $x \cdot (y + z) = x \cdot y + x \cdot z$      \\
        $A \vee (B \wedge C) \equiv (A \vee B) \wedge (A \vee C)$   & distributiva            & \sout{$x + y \cdot z = (x + y) \cdot (x + z)$} \\
        $A \wedge (A \vee B) \equiv A$                              & assorbimento            & \sout{$x \cdot (x + y) = x$}                   \\
        $A \vee (A \wedge B) \equiv A$                              & assorbimento            & \sout{$x + (x \cdot y) = x$}                   \\
    \end{tabular}
\end{frame}

\begin{frame}{Proprietà che coinvolgono la negazione}
    \begin{center}
        \begin{tabular}{c|c|c}
            \textbf{logica}               & \textbf{nome proprietà} & \textbf{algebra}             \\
            \hline
            $\neg \neg A \equiv A$      & doppia negazione        & $ 1 - (1-x) = x$             \\
            $A \vee \neg A \equiv \top$   & terzo escluso           & $x + (1 - x) = 1$            \\
            $A \wedge \neg A \equiv \bot$ & non contraddizione      & \sout{$x \cdot (1 - x) = 0$} \\
            $\neg \top = \bot$            &                         & $1 - 1 = 0$                  \\
            $\neg \bot = \top$            &                         & $1- 0 = 1$
        \end{tabular}
    \end{center}

    \medskip
    Vediamo quindi che per alcune equivalenze logiche vale una equivalenza algebrica corrispondente, ma non per tutte. Una corrispondenza totale si potrebbe avere tra le equivalenze logiche e le equivalenze insiemistiche (con le operazioni di unione, intersezione e complemento), ma non approfondiamo questo argomento.
\end{frame}

\begin{frame}{Alcuni esempi in linguaggio naturale (1)}
    Il vatto che quelle viste prime siano tatutologie segue dal calcolo della tavole di verità. Cerchiamo comunque di dare una qualche intuizione nel linguaggio naturale.
    \begin{itemize}
        \item $A \wedge B \equiv B \wedge A$: cambiare l'ordine delle affermazioni, non cambia il significato.
        ``Carlo è alto e Mario è basso'' $\equiv$ ``Mario è basso e Carlo è alto''.
        \item $A \wedge (B \wedge C) \equiv (A \wedge B) \wedge C$: non è facile fare un esempio in italiano, dove non ci sono le parentesi. Tuttavia, possiamo ripetere il soggetto in modo diverso per dare una idea del raggruppamento. `Carlo è alto e biondo e Carlo è affabile'' $\equiv$ ``Carlo è alto e Carlo è biondo e affabile''.
        \item $A \wedge A \equiv A$: ripetere due voleta la stessa proposizione non ne cambia il significato.`Carlo è alto e Carlo è alto'' $\equiv$ ``Carlo è alto''.
        \item $A \wedge \top \equiv A$:  ``Michele è o non è alto'' è sicuramente vero, quindi non ha alcuna rilevanza. ``Carlo è alto e Michele è o non è alto'' $\equiv$ ``Carlo è alto''
        \item $A \wedge \bot \equiv \bot$:``Carlo è alto e Michele è alto e non è alto'' è sicuramente falso.
    \end{itemize}
\end{frame}

\begin{frame}{Alcuni esempi in linguaggio naturale (2)}
    \begin{itemize}
        \item $A \wedge (B \vee C) \equiv (A \wedge B) \vee (A \wedge C)$:
        ``Carlo studia informatica e Carlo studia economia o filosofia'' $\equiv$ ``Carlo studia informatica ed economia oppure Carlo studia informatica e filosofia''.
        \item $A \vee (B \wedge C) \equiv (A \vee B) \wedge (A \vee C)$:
        ``Carlo studia informatica oppure Carlo studia economia e filosofia'' $\equiv$ ``Carlo studia informatica o economia e Carlo studio informatica o filosofia''. Tuttavia,  l'equivalenza in italiano non è molto intuitiva.
        \item $A \wedge (A \vee B)$: ``Carlo studia informatica e Carlo studia informatica o economia'' $\equiv$ ``Carlo studia informatica''. Infatti, perché la proposizione sia vera, è necessario che Carlo studi informatica, mentre se studia anche economia è ininfluente.
        \item $A \vee (A \wedge B)$: ``Carlo studia informatica oppure Carlo studia informatica ed economia'' $\equiv$ ``Carlo studia informatica''. Come prima, perché la proposizione sia vera è sufficiente che Carlo studi informatica, mentre se studia anche economia è ininfluente.
        \item $\neg \neg A \equiv A$: due negazioni affermano. ``Non è vero che Carlo non studia informatica'' $\equiv$ ``Carlo studia informatica ''.
    \end{itemize}
\end{frame}

% \begin{frame}{Equivalenza logica e insiemistica (1)}
%     Vediamo quindi che per alcune equivalenze logiche vale una equivalenza algebrica corrispondente, ma non per tutte.

%     \medskip
%     In realtà, una totale corrispondenza si ha tra le equivalenze logiche e le equivalenze insiemistiche. Si consideri un insieme fissato $U$ (detto \emph{insieme universo}) e tutti i suoi sottinsiemi. La corrispondenza tra logica e insiemistica è la seguente:
%     \[
%         \vee \Rightarrow \cup \qquad \wedge \Rightarrow \cap \qquad \neg \Rightarrow  U \setminus \qquad \top = U \qquad \bot = \emptyset
%     \]
% \end{frame}

% \begin{frame}{Equivalenza logica e insiemistica (2)}
%     \centering
%     \begin{tabular}{c|c|c}
%         \textbf{logica}                                      & \textbf{nome}         & \textbf{insiemi}                          \\
%         \hline
%         $A \vee B \equiv B \vee A$                           & prop. commutativa     & $A \cup B = B \cup A$                     \\
%         $A \vee (B \vee C) \equiv (A \vee B) \vee C$         & prop. associativa     & $A \cup (B \cup C) = (A \cup B) \vee C$   \\
%         $A \vee A \equiv A$                                  & prop. di idempotenza  & $A \cup A = A$                            \\
%         $A \vee \bot \equiv A$                               & elemento neutro       & $A \cup \emptyset = A$                    \\
%         $A \vee \top \equiv \top$                            & elemento assorbente   & $A \cup U = U$                            \\
%         \hline
%         \hline
%         $A \wedge B \equiv B \wedge A$                       & prop. commutativa     & $A \cap B = B \cap A$                     \\
%         $A \wedge (B \wedge C) \equiv (A \wedge B) \wedge C$ & prop. associativa     & $A \cap (B \wedge C) = (A \cap B) \cap C$ \\
%         $A \wedge A \equiv A$                                & prop. di idempotenza  & $A \cap A = A$                            \\
%         $A \wedge \top \equiv A$                             & elemento neutro       & $A \cap U = A$                            \\
%         $A \wedge \bot \equiv \bot$                          & elemento assorbente   & $A \cap \emptyset = \emptyset$            \\
%         \hline
%         \hline
%         $A \wedge (B \vee C)$                                & prop. distributiva    & $A \cap (B \cup C)$                       \\
%         $A \vee (B \wedge C)$                                & prop. distributiva    & $A \cup (B \cap C)$                       \\
%         $A \wedge (A \vee B) \equiv A$                       & prop. di assorbimento & $A \cap (A \cup B) =A$                    \\
%         $A \vee (A \wedge B) \equiv A$                       & prop. di assorbimento & $A \cup (A \cap B) = A$                   \\
%     \end{tabular}
% \end{frame}


\begin{frame}{Leggi di De Morgan (1)}
    Le leggi di De Morgan correlano congiunzione e disgiunzione con la negazione. Sono leggi molto importanti che consentono di semplificare le \fp complesse.
    \begin{gather*}
        \neg (A \wedge B) = \neg A \vee \neg B\\
        \neg (A \vee B) = \neg A \wedge \neg B
    \end{gather*}

    \begin{example}
        Consideriamo la proposizione:
        \begin{quote}
            Non è vero che Carlo studia informatica ed economia
        \end{quote}
        abbreviazione di
        \begin{quote}
            Non è vero che Carlo studia informatica ed economia
        \end{quote}
        È equivalente a dire
        \begin{quote}
            Carlo non studia informatica o Carlo non studia economia
        \end{quote}
    \end{example}
\end{frame}

\begin{frame}[fragile]{Leggi di De Morgan (2)}
    \begin{example}
        Consideriamo un frammento di  programma per calcolare le spese di spedizione.

        \begin{minted}[autogobble,xleftmargin=0.5cm]{python}
        shipping_charge = 10
        if not (state == "Italy" and region != "Sardegna"):
            shipping_charge = 20
    \end{minted}

        C'è una tariffa base di 10 €, ma sotto certe condizioni la tariffa sale a 20 € Ma quali sono queste codizioni ? Non si capisce molto bene.

        \pause \medskip
        Applicando De Morgan, possiamo riscrivere la condizione come:
        \begin{minted}[autogobble,xleftmargin=0.5cm]{python}
       (not state == "Italy") or (not regione != "Sardegna")
    \end{minted}

        \pause \medskip
        E quindi
        \begin{minted}[autogobble,xleftmargin=0.5cm]{python}
        state != "Italy" or regione == "Sardegna"
    \end{minted}
        Le spese aggiuntive si pagane se la spedizione e all'estero, oppure in Sicilia o in Sardegna.
    \end{example}
\end{frame}

\begin{frame}{Il principio di dualità}
    Abbiamo visto che molte equivalenze logiche si possono ottenere scambiando $\wedge$ con $\vee$ e $\top$ con $\bot$.

    \begin{example}
        $A \wedge \top \equiv A$ diventa $A \vee \bot \equiv A$.
    \end{example}

    Questo fatto non è un caso, e si può formalizzare come segue:

    \begin{theorem}[Principio di dualità]
        Sia $X$ ed $Y$ due \fp logicamente equivalente. Se $X'$ e $Y'$ sono ottenute da $X$e ed $Y$ scambiando $\wedge$ con $\vee$ e $\top$ con $\bot$, allora $X' \equiv Y'$.
    \end{theorem}
\end{frame}

\begin{frame}{Proprietà associativa e parentesi}
    Il fatto che $\wedge$ e $\vee$ godano della proprietà associativa, ci consente di risparmiare qualche parentesi.
    \begin{itemize}
    \item Consideriamo la formula $A \wedge B \wedge C$. La formula non va bene perché è ambigua, la si può interpretare come $(A \wedge B) \wedge C$ oppure $A \wedge (B \wedge C)$.
    \item Le due possibile interpretazioni sono però equivalenti, per la proprietà associativa, quindi non ci crea nessun problema!
    \item La stessa cosa vale anche per $\vee$, ma non se ci c'è una mescolanza di $\wedge$ e $\vee$:
    \begin{itemize}
        \item possiamo scrivere $A \vee B \vee C$ perché tanto le due alternative $(A \vee B) \vee C$ e $A \vee (B \vee C)$ sono equivalenti;
        \item non possiamo scrivere $A \wedge B \vee C$ perché le due alternative $(A \wedge B) \vee C$ e $A \wedge (B \vee C)$ \textbf{non} sono equivalenti.
    \end{itemize}
    \end{itemize}
\end{frame}

\begin{frame}{Equivalenze con l'or esclusivo}
    L'or esclusivo è definibile a partire da and, or e not, come segue:
    \[
        A \xor B \equiv (A \wedge \neg B) \vee (\neg A \wedge B)
    \]
    La formula è complessa, ci torneremo dopo.

    \pause \medskip Altre equivalenze notevoli:
    \begin{align*}
        A \xor B          & \equiv B \xor A          \\
        A \xor (B \xor C) & \equiv (A \xor B) \xor C \\
        A \xor \bot       & \equiv A                 \\
        A \xor \top       & \equiv \neg A            \\
        A \xor A          & \equiv \bot              \\
        A \xor \neg A     & \equiv \top              \\
    \end{align*}
\end{frame}

\begin{frame}{L'implicazione (1)}
    L'implicazione è definibile a partire da and, or, e not:
    \[
        A \to B \equiv \neg(A \wedge \neg B)  \equiv \neg A \vee B
    \]
    Detto in altro modo, affermare $A \to B$ è equivalente ad affermare:
    \begin{itemize}
        \item ``non è vero che $A$ è vero e contemporaneamente $B$ è falso''
        \item ``$A$ è falso oppure $B$ è vero''.
    \end{itemize}

    \smallskip
    Ancora:
    \begin{align*}
        A \to A      & \equiv \top   & (A \wedge B) \to C & \equiv A \to (B \to C)            \\
        A \to \neg A & \equiv \neg A & A \to (B \wedge C) & \equiv (A \to B) \wedge (A \to C) \\
        \bot \to A   & \equiv \top   & (A \vee B) \to C   & \equiv (A \to C) \wedge (B \to C) \\
        \top \to A   & \equiv A      & A \to (B \vee C)   & \equiv (A \to B) \vee (A \to C)   \\
        A \to \bot   & \equiv \neg A                                                          \\
        A \to \top   & \equiv \top
    \end{align*}
\end{frame}

\begin{frame}{L'implicazione (2)}
    Data una implicazione $A \to B$, si possono definire altre tre implicazioni correlate:
    \begin{itemize}
        \item l'implicazione \alert{inversa}: $B \to A$
        \item l'implicazione \alert{contraria}: $\neg A \to \neg B$
        \item l'implicazione \alert{contronominale}: $\neg B \to \neg A$
    \end{itemize}
    L'inversa e la contraria non sono direttamente correlate con l'implicazione originale, invece la contronominale è equivalente all'originale:
    \[
        A \to B \equiv \neg B \to \neg A
    \]
    \begin{example}
        Data la funzione proposizionale ``se $x$ è divisibile per 4 allora x è pari'',  vera per tutti gli $x$:
        \begin{itemize}
            \item l'inversa ``se $x$ è pari allora x è divisibile per 4'' è falsa per $x=2$;
            \item la contraria ``se $x$ non è divisibile per 4 allora x non è pari'' è falsa per $x=2$;
            \item la contronominale ``se $x$ non è pari allora x non è divisibile per 4'' è vera per tutti i possibili $x$.
        \end{itemize}
    \end{example}
\end{frame}

\begin{frame}{Contronomiale, modus ponens e modus tollens}
    Nella parte introduttiva del corso abbiamo visto due regole di inferenza che abbiamo chiamato \emph{modus ponens} e \emph{modus tollens}:

    \medskip
    \begin{center}
    \begin{inference}
        $A \to B$\\
        $A$\\
        \hline
        $B$
    \end{inference}
    (modus ponens)
    \hspace{2cm}
    \begin{inference}
        $A \to B$\\
        $\neg B$\\
        \hline
        $\neg A$
    \end{inference}
    (modus tollens)
    \end{center}
    In virtù della equivalenza logica tra $A \to B$ e $\neg B \to \neg A$, possiamo pensare al modus tollens come segue:
    \begin{center}
    \begin{inference}
        $\neg B \to \neg A$\\
        $\neg B$\\
        \hline
        $\neg A$
    \end{inference}
    \end{center}
    che quindi non è altro che il modus ponens ma applicato alla contronominale della proposizione $A \to B$.
\end{frame}

\begin{frame}{La doppia implicazione (1)}
    Due importanti equivalenze riguardanti la doppia implicazione sono
    \begin{gather*}
        A \iff B \equiv (A \to B) \wedge (B \to A)\\
        A \iff B \equiv (A \to B) \wedge (\neg A \to \neg B)
    \end{gather*}
    che mostrano come $A \iff B$ non è altro che la congiunzione di due implicazioni nei due versi opposti.

    \pause \medskip
    La doppia implicazione è inoltre definibile a partire dall'or esclusivo (e viceversa).
    \begin{gather*}
        A \iff B \equiv \neg (A \xor B)\\
        A \xor B \equiv \neg (A \iff B)
    \end{gather*}

    \pause \medskip
    Infine, la doppia implicazione è anche equivalente ad una formula che usa solo and, or e not:
    \[
        A \iff B \equiv (A \wedge B) \vee (\neg A \wedge \neg B)
    \]
\end{frame}

\begin{frame}{La doppia implicazione (2)}
    Altre equivalenze che coinvogono la doppia implicazione:
    \begin{align*}
        A \iff A      & \equiv \top     \\
        A \iff \neg A & \equiv \bot     \\
        A \iff \bot   & \equiv \neg A   \\
        A \iff \top   & \equiv A        \\
        A \iff B      & \equiv B \iff A \\
    \end{align*}
\end{frame}


\section{Semplificare le \fp}

\begin{frame}{Applicare una equivalenza ad una formula (1)}
    Il fatto che $A \wedge B \equiv B \wedge A$ vuol dire che, ovunque in una fp compaia $A \wedge B$, è possibile rimpiazzarlo con $B \wedge A$ senza cambiare il significato della formula (e viceversa).
    \begin{itemize}
        \item $A \wedge B \to C \equiv$ \pause $B \wedge A \to C$ \pause\
        \item $(B \wedge A) \xor (C \iff A) \equiv$ \pause $(A \wedge B) \xor (C \iff A)$
    \end{itemize}

    \pause\medskip
    Questo fatto si può formalizzare come segue:
    \begin{theorem}
        Sia $P$, $Q$ ed $R$ forme proposizionali, con $P \equiv Q$. Se $R'$ è ottenuta da $R$ rimpiazzando tute le occorrenze di $P$ con $Q$ o viceversa, allora $R \equiv R'$.
    \end{theorem}
\end{frame}

\begin{frame}{Applicare una equivalenza ad una formula (2)}
    Inoltre, in una equivalenza com $A \wedge B \equiv B \wedge A$, le lettere $A$ e $B$ possono essere rimpiazzate con qualsiasi altra formula proposizionale, ottenendo sempre una equivalenza:
    \begin{itemize}
        \item $C \wedge D \equiv$ \pause $D \wedge C$ \\
                (abbiamo rimpiazzato $A$ con $C$ e $B$ con $D$)
        \item $(A \to B) \wedge (A \vee C) \equiv$ \pause $(A \vee C) \wedge (A \to B)$\\
            (abbiamo rimpiazzato $A$ con $A \to B$, e $B$ con $A \vee C$)
    \end{itemize}

    \pause\medskip
    Questo fatto si può formalizzare come segue:
    \begin{theorem}
        Sia $P \equiv Q$. Se $P'$ e $Q'$ sono altre forme proposizionali ottenute da $P$ e $Q$ rimpiazzando le lettere proposizionali con forme proposizionali in maniera consistente (ovvero, la stessa lettera è sempre sostituita dalla stessa formula), allora $P' \equiv Q'$.
    \end{theorem}
\end{frame}

% \begin{frame}{Applicare una equivalenza ad una formula (2)}
%     Quando rimpiazzo una lettera proposizionale con una formula, devo sempre aggiungere delle parentesi attorno alla formula, altrimenti rischio di ottenere forme proposizionali ambigue o equivalenze errate.
%     \begin{example}
%         Dato $A \wedge B \equiv B \wedge A$, rimpiazziamo
%         \begin{itemize}
%             \item $A \Rightarrow A \to B$
%             \item $B \Rightarrow A \vee C$
%         \end{itemize}
%         Senza parentesi, si ottiene \pause $A \to B \wedge A \vee C \equiv A \vee C \wedge A \to B$, che è ambiguo per l'alternarsi di $\wedge$ e $\vee$.

%         \pause
%         \medskip
%         Peggio ancora, se rimpiazzo
%         \begin{itemize}
%             \item $A \Rightarrow A$
%             \item $B \Rightarrow B \to C$
%         \end{itemize}
%         ottengo \pause $A \wedge B \to C \equiv B \to C \wedge A$ che si legge
%         $(A \wedge B) \to C \equiv B \to (C \wedge A)$ che non è una equivalenza corretta!
%     \end{example}
% \end{frame}

\begin{frame}{Applicare una equivalenza ad una formula (3)}
    Ovviamente i due teoremi si possono combinare tra di loro. Siccome $A \wedge B \equiv B \wedge A$, abbiamo:
    \begin{itemize}
        \item $A \to (B \wedge (C \iff D))  \equiv$ \pause $A \to ((C \iff D) \wedge B)$ \pause
        \item $((A \vee B) \wedge (C \iff D)) \to D \equiv$ \pause $( (C \iff D) \wedge (A \vee B)) \to D$
    \end{itemize}

    \pause\medskip
    Nella prima, ad esempio, considere la sottoformula $B \wedge (C \iff D)$ e applico la commutatività di $\wedge$ per scambiare $B$ con $(C \iff D)$.

    \pause\medskip
    In conclusione, il fatto che $A \wedge B \equiv B \wedge A$ vuol dire che, in una forma proposizionale, ogna volta che c'è un connettivo ``and'' è possibile cambiare l'ordine dei congiunti e ottenere una forma equivalente. Analogamente per le altre equivalenze notevoli.

    \medskip
    È possibile usare queste manipolazioni di tipo algebrico per
    \begin{itemize}
        \item verificare nuove equivalenze logiche in maniera più veloce non costruendo la tavola di verità;
        \item semplificare delle forme proposizionali complesse.
    \end{itemize}
\end{frame}

\begin{frame}{Verificare una equivalenza logica}
    Vogliamo verificare l'equivalenza $P \equiv Q$.  Si parte dalla proposizione sul lato sinistro o destro (quella che sembra più comoda) e si applicano le equivalenze note fino ad ottenere l'altra proposizione.

    \begin{example}
        Vogliamo verificare l'equivalenza $A \vee B \equiv \neg(\neg A \wedge \neg B)$. Partendo da destra:
        \[
            \begin{array}{rl@{\hspace{0.8cm}}l}
                              & \neg(\neg A \wedge \neg B) & \pause \text{(legge di De Morgan)} \\
                \pause \equiv & \neg\neg A \vee \neg\neg B & \pause \text{(doppia negazione)}   \\
                \pause \equiv & A \vee \neg\neg B          & \pause \text{(doppia negazione)}   \\
                \pause \equiv & A \vee B
            \end{array}
        \]
    \end{example}
\end{frame}

\note{
    Credo che serva dare una spiegazione un po' più dettagliata di come $\neg(\neg A \wedge \neg B)$ diventa $(\neg \neg A \vee \neg \neg B)$. Si potrebbe probabilmente scrivere la formula di De Morgan originaria come $\neg (X \wedge Y) = \neg X \vee \neg Y$ e far vedere graficamente che rimpiazziamo $X$ con $\neg A$ e $Y$ con $\neg B$.
}

\begin{frame}{Semplificazione di una fp}
    Vogliamo semplificare la forma proposizionale $P$. La tecnica è la stessa: si parte da $P$ e si applicano le equivalenze note fino ad ottenere una forma più semplice, ricordando che talvolta, per semplificare le cose, occorre prima complicarle !

    \begin{example}
        Vogliamo semplificare la fp. $(A \wedge B) \vee (A \wedge \neg B)$.
        \[
            \begin{array}{rl@{\hspace{0.8cm}}l}
                              & (A \wedge B) \vee (A \wedge \neg B) & \pause \text{(prop. distributiva${}^*$)} \\
                \pause \equiv & A \wedge (B \vee \neg B)            & \pause \text{(terzo escluso)}            \\
                \pause \equiv & A \wedge \top                       & \pause \text{(elemento neutro di $\wedge$)}          \\
                \pause \equiv & A
            \end{array}
        \]
        Quello indicato con ${}^*$ è l'uso della proprietà distributiva $A \wedge (B \vee C) \equiv (A \wedge B) \vee (A \wedge C)$ ma da destra a sinistra. È l'operazione che nelle espressioni algebriche si chiama \emph{mettere in evidenza} o \emph{raccogliere}.
    \end{example}
\end{frame}

\note{Esercizio 4.5, $(A \to B) \wedge (A \to C) \equiv A \to (B \wedge C)$}

\end{document}
