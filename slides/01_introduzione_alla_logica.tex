\documentclass[aspectratio=169,10pt,dvipsnames,handout]{beamer}

%\setbeameroption{show notes}

\input{preamble.inc}

\title{Introduzione alla logica}

\begin{document}

\begin{frame}
	\titlepage
\end{frame}

\begin{frame}
	\copyrightpage
\end{frame}

\begin{frame}
	\bookpage{sul capitolo \emph{Introduzione}}
\end{frame}

\section{Logica, inferenze e proposizioni}

\begin{frame}{Cos'è la logica}
	\begin{definition}[Logica]
		La \alert{logica} (dal greco \textit{logos}, ovvero ``parola'', ``pensiero'') è lo studio del ragionamento.
	\end{definition}
	\begin{example}[Esempi di ragionamento]
		\begin{columns}
			\column{0.5\textwidth}
			\centering\propshape

			Tutti gli uomini sono mortali.\\
			Socrate è un uomo.\\
			Dunque Socrate è mortale.\\[0.6cm]
			Ogni numero intero è o pari o dispari\\
			Il numero intero 3 non è pari\\
			Dunque 3 è dispari.
			\column{0.5\textwidth}
			\centering
			\includegraphics[width=2cm,keepaspectratio]{logica.png}\\
			\alert{Sono ragionamenti corretti?}
		\end{columns}
	\end{example}
	\begin{block}{Nota}
		Esistono varie forme di ragionamento, qui siamo interessati solo al \alert{ragionamento deduttivo}.
	\end{block}
\end{frame}

\begin{frame}{Inferenze}
	La forma principale del ragionamento è quella dell'\alert{inferenza}.
	\begin{definition}[Inferenza]
		Una \alert{inferenza} è una sequenza di \alert{proposizioni} di cui l'ultima è ottenuta come \alert{conclusione} delle rimanenti, che si assumono come \alert{premesse}.
	\end{definition}
	\begin{example}
		\medskip
		\[
			\begin{tikzpicture}
				\tikzset{
					fillstyle/.style={rounded corners, draw=black},
					greenstyle/.style={fillstyle,fill=green!50},
					redstyle/.style={fillstyle,fill=red!50},
					bluestyle/.style={fillstyle,fill=blue!50},
					yellowstyle/.style={fillstyle,fill=yellow!50}
				}
				\clip (-6cm, -3cm) rectangle (6cm, 0.5cm);
				\node (p1) {\prop{Tutti gli uomini sono mortali}.};
				\node (p2) [below of=p1] {\prop{Socrate è un uomo}.};
				\node (c) [below of=p2] {\prop{Socrate è mortale}.};
				\begin{scope}[on background layer]
					\only<2->{\node (inf) [greenstyle, fit={(p1) (p2) (c)}, inner sep=0.5cm] {};}
					\only<4->{\node (premises) [redstyle, fit={(p1) (p2)},  inner sep=0.2cm] {};}
					\only<5->{\node (conclusion) [yellowstyle, fit={(c)},  inner sep=0.2cm] {};}
					\only<3->{\node (prop1) [bluestyle, fit={(p1)},  inner sep=-0.05cm] {};}
					\only<3->{\node (prop2) [bluestyle, fit={(p2)},  inner sep=-0.05cm] {};}
					\only<3->{\node (prop3) [bluestyle, fit={(c)},  inner sep=-0.05cm] {};}
				\end{scope}
				\draw ($(c.west) + (-1cm,0.5cm)$) -- ($(c.east) + (1cm,0.5cm)$);
				\only<2->{
					\draw (inf) ++ (-4.5cm,0cm)  node (inflabel) [greenstyle] {Inferenza};
					\draw [thick, ->] (inflabel) -- (inf.west);
				}
				\only<4->{
					\draw (premises) ++ (-4.5cm,0cm)  node (premiseslabel) [redstyle] {Premesse};
					\draw [thick, ->] (premiseslabel) -- (premises.west);
				}
				\only<5->{
					\draw (conclusion) ++ (-4.5cm,0cm)  node (conclusionlabel) [yellowstyle] {Conclusione};
					\draw [thick, ->] (conclusionlabel) -- (conclusion.west);
				}
				\only<3->{
					\draw (prop2) ++ (+4.5cm,0cm)  node (proplabel) [bluestyle] {Proposizioni};
					\draw [thick, ->] (proplabel) -- (prop1.east);
					\draw [thick, ->] (proplabel) -- (prop2.east);
					\draw [thick, ->] (proplabel) -- (prop3.east);
				}
			\end{tikzpicture}
		\]
	\end{example}
\end{frame}

\begin{frame}{Proposizioni}
	\begin{definition}[Proposizione]
		Una \alert{proposizione} è una qualsiasi espressione linguistica per la quale ha senso chiedersi se sia vera o falsa.
	\end{definition}
	\begin{example}[Proposizioni \checkmark o no? \xmark]
		\begin{itemize}
			\item \prop{Elisa ama Massimo} \pause \checkmark
			\item \prop{Elisa ama Massimo?} \pause \xmark\ (è una \emph{domanda})
			\item \prop{$4 + 2 = 6$} \pause  \checkmark
			\item \prop{$4 + 2 = 5$} \pause \checkmark (è falsa, ma è comunque una proposizione)
			\item \prop{$3 + 5^2$} \pause \ \xmark \ (è una \emph{descrizione})
			\item \prop{Tutti gli uomini sono mortali} \pause \checkmark
			\item \prop{Lazzaro, alzati e cammina!} \pause \xmark \ (è un \emph{ordine})
			\item \prop{Quel ramo del lago di Como che volge a mezzogiorno} \pause \xmark \ (è una \emph{descrizione})
		\end{itemize}
	\end{example}
\end{frame}

% \begin{frame}{Proposizioni}
% 	\begin{definition}[Proposizione]
% 		Una \alert{proposizione} è una qualsiasi espressione linguistica per la quale ha senso chiedersi se sia vera o falsa.
% 	\end{definition}
% 	Non è necessario che sappiamo dire veramente se una proposizione è vera o falsa.

% 	\begin{example}[Mancanza di contesto]
% 		\emph{Luigi è più alto di Michele.}

% 		\medskip
% 		Se non sappiamo a chi ci riferiamo con Luigi e Michele, non sappiamo le proposizione è vera o falsa.
% 	\end{example}

% 	\begin{example}[Congettura di Goldbach]
% 		\emph{Ogni numero pari maggiore di 2 può essere scritto come somma di due numeri primi (eventualmente uguali).}

% 		\medskip
% 		È stata verificata al calcolatore fino a  $4 \cdot 10^{18}$, ma nessuno è mai riuscito a dimostrarla.
% 	\end{example}
% \end{frame}

\begin{frame}{Valori di verità}
	\begin{definition}[Principio di bivalenza]
		Una proposizione può assumere uno e uno solo dei due \alert{valori di verità}:
		\begin{itemize}
			\item vero (\alert{V}), oppure true in inglese (\alert{T})
			\item falso (\alert{F}), oppure false in inglese (\alert{F})
		\end{itemize}
	\end{definition}
	Nella vita quotidiana, questo principio non sempre si applica:
	\begin{itemize}
		\item è quasi certo che \ldots
		\item è probabile che \ldots
		\item sono abbastanza sicuro che \ldots
	\end{itemize}
	La logica si occupa anche dei casi in cui il principio di bivalenza non vale, ma non lo faremo in questo corso.
\end{frame}

\begin{frame}{Proposizioni ed enunciati}
	Quando vorremo essere più precisi, distingueremo tra \alert{proposizioni} ed \alert{enunciati}:
	\begin{itemize}
		\item una proposizione è qualcosa che può essere valutata come vera o falsa;
		\item un enunciato è una sequenza di parole che esprime una proposizione.
	\end{itemize}

	\medskip
	Un esempio dovrebbe rendere evidente la differenza. Consideriamo le tre frasi:
	\begin{itemize}
		\item \prop{il gatto non è nero}
		\item \prop{non è vero che il gatto è nero}
		\item \prop{the cat is not black}
	\end{itemize}
	Questi sono tre enunciati che però vogliono dire la stessa cosa, ovvero esprimono la stessa proposizione.
\end{frame}

%
%\begin{frame}{Introduciamo qualche simbolo}
%	\textbf{Variabili per proposizioni}\\
%	Quando vogliamo indicare una proposizione senza specificarla esattamente, useremo le lettere maiuscole: $A$, $B$, $C$, \ldots, $P$, \ldots.
%
%	\medskip
%	\textbf{Indicatore di conclusione}\\
%	Invece di usare parole come ``dunque'' o ``allora'' per indicare la conclusione di una  inferenza, utilizzeremo una linea orizzontale che separi premesse da conclusione.
%
%	\begin{center}
%	\begin{inference}
%	Tutti gli uomini sono mortali\\
%	Socrate è un uomo\\
%	\hline
%	Socrate è mortale
%	\end{inference}
%	\end{center}
%\end{frame}

% \begin{frame}{Conseguenza logica}
% 	\begin{definition}[Conseguenza logica]
% 		Una proposizione $C$ è \alert{conseguenza logica} delle (o \alert{segue logicamente} dalle) proposizioni $P_1, \ldots, P_n$ qualora ogni volta che le proposizioni $P_1, \ldots, P_n$ sono vere, anche la conclusione $C$ sarà necessariamente vera.
% 	\end{definition}
% 	\begin{example}[Conseguenza logica]
% 		Sia
% 		\begin{itemize}
% 			\item $P_1 = $ ``Carlo è ligure o Stefano è piemontese''
% 			\item $P_2 = $ ``Stefano non è piemontese''
% 			\item $C = $ ``Carlo è ligure''
% 		\end{itemize}
% 		Allora \alert{$C$ à conseguenza logica di $P_1$ e $P_2$}.
% 	\end{example}
% 	\begin{block}{Nota}
% 		In alternativa, possiamo dire che $C$ è \alert{conseguenza logica} delle proposizioni $P_1, \ldots, P_n$ qualora è impossibile che $P_1, \ldots, P_n$ siano tutte vere e nello stesso tempo $C$ sia falsa.
% 	\end{block}
% \end{frame}

\begin{frame}{Inferenze corrette}
	\begin{definition}[Inferenza  corretta]
		Un'inferenza è \alert{corretta} (o \alert{valida}) se ogni qualvolta le premesse sono vere, allora anche la conclusione è necessariamente vera.
	\end{definition}
	\begin{example}[Inferenza corretta]
		\begin{center}
			\begin{inference}
				Carlo è ligure o Stefano è piemontese\\
				Stefano non è piemontese\\
				\hline
				Carlo è ligure
			\end{inference}
		\end{center}
	\end{example}
	\begin{block}{Nota}
		In alternativa, possiamo dire che l'inferenza è corretta quando è impossibile che le premesse siano vere e la conclusione falsa.
	\end{block}
\end{frame}


\begin{frame}{Inferenze corrette e verità delle conclusioni}
	Il fatto che un'inferenza sia corretta non vuol dire che la conclusione sia vera.
	\begin{itemize}
		\item \alert{Se le premesse sono tutte vere}, un'inferenza corretta garantisce che la conclusione è vera.
		\item \alert{Se le premesse non sono tutte vere}, la conclusione può essere vera o falsa.
	\end{itemize}

	\medskip
	Consideriamo ancora questa inferenza:

	\begin{center}
		\begin{inference}
			Carlo è ligure o Stefano è piemontese\\
			Stefano non è piemontese\\
			\hline
			Carlo è ligure
		\end{inference}
	\end{center}

	\medskip
	L'inferenza è corretta, ma se un giorno dovessimo scoprire che Carlo è nato a Roma e Stefano è di Pescara, la conclusione dell'inferenza sarebbe falsa. Ma ciò è normale, perché le premesse dell'inferenza non erano vere!
\end{frame}

\begin{frame}{Inferenze e contesto}
	Non consideriamo corrette inferenze la cui correttezza dipende da conoscenze di contesto non esplicitate.
	\begin{example}[Inferenza non corretta]
		\begin{columns}
			\begin{column}{0.3\textwidth}
				\begin{center}
					\begin{inference}
						Marco è pescarese\\
						\hline
						Marco è abruzzese
					\end{inference}
				\end{center}
				\vspace{0.1cm}
			\end{column}
			\begin{column}{0.6\textwidth}
				\pause
				dipende dal significato di pescarese e abruzzese
			\end{column}
		\end{columns}
	\end{example}
	\pause
	La si può rendere corretta esplicitando le assunzioni di fondo:
	\begin{example}[Inferenza corretta]
		\begin{center}
			\begin{inference}
				Marco è pescarese\\
				\textcolor{blue}{Tutti i pescaresi sono abruzzesi}\\
				\hline
				Marco è abruzzese
			\end{inference}
		\end{center}
		\vspace{0.1cm}
	\end{example}
\end{frame}

%\begin{frame}{Inferenze e struttura sintattica (1)}
%		\begin{example}[Inferenza corretta]
%			\centering
%		\begin{inference}
%			Carlo è ligure o Stefano è piemontese\\
%			Stefano non è piemontese\\
%			\hline
%			Carlo è ligure
%		\end{inference}
%	\end{example}
%
%	\medskip
%	Per stabilire che l'inferenza di prima è corretta, non è necessario che conosciamo Carlo o Stefano. La correttezza dipende solo dalla \alert{struttura sintattica} delle proposizioni.
%
%	\medskip
%	In questo caso, la prima premessa dice che una tra due affermazioni (Carlo è ligure, oppure Stefano è piemontese) è vera. La seconda premessa dice che la seconda affermazione (Stefano è piemontese) è vera. Quindi, se le premesse sono entrambe vere, non può che essere vera la prima affermazione (Carlo è ligure).
%\end{frame}

\section{Regole di inferenza}

\begin{frame}{Inferenze e struttura sintattica}
	Se rimpiazziamo ``Carlo è ligure'' e ``Stefano è piemontese'' con altre proposizioni, otteniamo altre inferenze corrette: quella che importa è solo la struttura sintattica delle proposizioni, chiamata \alert{forma logica}.
	\begin{example}
		\centering
		\begin{inference}
			Carlo è ligure o Stefano è piemontese\\
			Stefano non è piemontese\\
			\hline
			Carlo è ligure
		\end{inference}

		\medskip
		\begin{inference}
			Il maggiordomo è colpevole o la cameriera è colpevole.\\
			La cameriera non è colpevole\\
			\hline
			Il maggiordomo è colpevole
		\end{inference}

		\medskip
		% \begin{inference}
		% 	Marcello è matematico o fisico\\
		% 	Marcello non è fisico\\
		% 	\hline
		% 	Marcello è matematico
		% \end{inference}
		\begin{inference}
			Due buchi neri si sono scontrati,\hspace{6cm}\hfill\\ \hfill o il rivelatore di onde gravitazionali è rotto.\\
			Il rivelatore di onde gravitazionali non è rotto.\\
			\hline
			Due buchi neri si sono scontrati.
		\end{inference}
	\end{example}
	%\small Per l'ultima inferenza, notate che dire ``Marcello è matematico o fisico'' è solo un modo compatto per dire ``Marcello è matematico o Marcello è fisico''.
\end{frame}

\begin{frame}{Regole di inferenza}
	Mettiamo in evidenza la \alert{forma logica} dell'inferenza.
	\begin{itemize}
		\item usiamo delle \alert{lettere proposizionali}, al posto delle proposizioni;
		\item lasciamo invariante parole chiavi come ``o'' e ``non'', chiamate \alert{connettivi}.
	\end{itemize}
	\begin{example}[Forma logica]
		Se si pone \textit{A=``Carlo è ligure''} e \textit{B=``Stefano è piemontese''}:
		\[
			\begin{tikzpicture}[remember picture]
				\node (carlo) {\begin{inference}
						Carlo è ligure o Stefano è piemontese\\
						Stefano non è piemontese\\
						\hline
						Carlo è ligure
					\end{inference}
				};
				\node (disj) [right=of carlo] {
					\begin{inference}
						A o B\\
						non B\\
						\hline
						A
					\end{inference}
				};
				\draw [-{Latex[width=3mm]},very thick] (carlo) -- (disj);
			\end{tikzpicture}
		\]
	\end{example}
	\pause
	\begin{definition}[Regola di inferenza]
		Uno schema di inferenza in cui usiamo \alert{lettere proposizionali} invece di proposizioni vere e proprie prende il nome di \alert{regola di inferenza}.
		\begin{tikzpicture}[overlay, remember picture]
			\draw [red, thick] (disj) circle (0.8cm);
			\path (disj) ++ (1.6cm,1cm) node [red] {Regola di inferenza};
		\end{tikzpicture}
	\end{definition}
\end{frame}

\begin{frame}{Regole di inferenza corrette}
	Una regola di inferenza rappresenta tutte le inferenze che si possono ottenere rimpiazzando le lettere proposizionali con delle proposizioni.
	\begin{itemize}
		\item le inferenze ottenute in questo modo si chiamano \alert{istanze} della regola;
		\item se una regola è corretta, tutte le istanze lo sono.
	\end{itemize}
	\begin{example}
		La regola di inferenza\\
		\medskip
		\qquad
		\begin{inference}
			A o B\\
			non B\\
			\hline
			A
		\end{inference}\\
		\medskip
		chiamata \alert{regola del sillogismo disgiuntivo} è corretta, quindi la sua istanza\\
		\medskip
		\qquad
		\begin{inference}
			Il maggiordomo è colpevole o la cameriera è colpevole.\\
			La cameriera non è colpevole\\
			\hline
			Il maggiordomo è colpevole
		\end{inference}\\

		\medskip
		è corretta.
	\end{example}
\end{frame}

\begin{frame}{Inferenze corrette e verità delle conclusioni}
	\medskip
	Ribadiamo che bisogna distinguere la correttezza delle inferenze dalla verità della conclusione!

	\medskip
	Consideriamo alcune istanze della regola vista prima:
	\[
		\begin{tikzpicture}
			\node (i2) {\begin{inference}
					A o B\\
					non B\\
					\hline
					A
				\end{inference}
			};
			\uncover<2->{
			\node (i1) [left=of i2] {\begin{inference}
					3 è primo o 3 è pari \only<3->{(\checkmark)}\\
					3 non è pari \only<4->{(\checkmark)}\\
					\hline
					3 è primo \only<5->{(\checkmark)}
				\end{inference}
			};
			\draw[-{Latex[width=3mm]},very thick] (i2) -- (i1);
			}
			\uncover<6->{
			\node (i3) [right=of i2] {\begin{inference}
					3 è pari o 3 è primo  \only<7->{(\checkmark)}\\
					3 non è primo \only<8->{(\xmark)}\\
					\hline
					3 è pari \only<9->{(\xmark)}
				\end{inference}
			};
			\draw[-{Latex[width=3mm]},very thick] (i2) -- (i3);
			}
			\uncover<10->{
			\node (i4) [below= of i3]{\begin{inference}
					3 è dispari o 3 è primo \only<11->{(\checkmark)}\\
					3 non è primo \only<12->{(\xmark)}\\
					\hline
					3 è dispari \only<13->{(\checkmark)}
				\end{inference}
			};
			\draw[-{Latex[width=3mm]},very thick] (i2) -- (i4);
			}
			\uncover<14->{
			\node (i5) [below= of i1]{\begin{inference}
					?????????????????? (\checkmark)\\
					?????????????????? (\checkmark)\\
					\hline
					?????????????????? (\xmark)
				\end{inference}
			};
			\draw[-{Latex[width=3mm]},very thick] (i2) -- (i5);
			}
			\only<15->{
				\draw  [red, line width=2mm] (i5) ++ (-2cm, 0.55cm) --  ++ (4cm, -1.1cm);
				\draw  [red, line width=2mm] (i5) ++ (-2cm, -0.55cm) --  ++ (4cm, 1.1cm);
			}
		\end{tikzpicture}
	\]
\end{frame}

\section{Esempi di regole di inferenza}
\begin{frame}{Un'altra regola corretta: modus ponens}
	Consideriamo l'inferenza
	\begin{center}
		\begin{inference}
			Se Fabio è pescarese, allora Fabio è abruzzese\\
			Fabio è pescarese\\
			\hline
			Fabio è abruzzese
		\end{inference}
	\end{center}
	\pause
	Si generalizza nella regola:
	\begin{center}
		\begin{inference}
			Se A allora B\\
			A\\
			\hline
			B
		\end{inference}
	\end{center}
	detta \alert{regola del modus ponens}. Vedi anche:
	\begin{center}
		\begin{inference}
			Se sono colpevole devo essere punito\\
			Sono colpevole\\
			\hline
			Devo essere punito
		\end{inference}
	\end{center}
\end{frame}

\begin{frame}{Un'altra regola corretta: modus tollens}
	Consideriamo l'inferenza
	\begin{center}
		\begin{inference}
			Se Fabio è pescarese, allora Fabio è abruzzese\\
			Fabio non è abruzzese\\
			\hline
			Fabio non è pescarese
		\end{inference}
	\end{center}
	\pause
	Si generalizza nella regola:
	\begin{center}
		\begin{inference}
			Se A allora B\\
			non B\\
			\hline
			non A
		\end{inference}
	\end{center}
	detta \alert{regola del modus tollens}. Vedi anche:
	\begin{center}
		\begin{inference}
			Se sono colpevole devo essere punito\\
			Non devo essere punito\\
			\hline
			Non sono colpevole
		\end{inference}
	\end{center}
\end{frame}

\begin{frame}{Fallacia della negazione dell'antecedente (1)}
	Consideriamo l'inferenza
	\begin{center}
		\begin{inference}
			Se sono un ladro allora devo essere punito\\
			Non sono un ladro\\
			\hline
			Non devo essere punito
		\end{inference}
	\end{center}
	che si generalizza nella regola
	\begin{center}
		\begin{inference}
			Se A allora B\\
			non A\\
			\hline
			non B
		\end{inference}
	\end{center}
	È corretta?
\end{frame}

\begin{frame}{Fallacia della negazione dell'antecedente (2)}
	All'apparenza potrebbe sembrare corretta, ma consideriamo un'altra istanza della regola.
	\begin{center}
		\begin{inference}
			Se Fabio è pescarese, allora Fabio è abruzzese \only<3->{\checkmark}\\
			Fabio non è pescarese \only<4->{\checkmark}\\
			\hline
			Fabio non è abruzzese \only<5->{\xmark}
		\end{inference}
	\end{center}
	È corretta?

	\pause
	\medskip
	Pensate al caso in cui Fabio sia nato a Chieti.

	\only<6->{
		\medskip
		Le premesse sono vere, ma la conclusione è falsa, quindi l'inferenza \alert{non è corretta}. Abbiamo individuato un \alert{controesempio}, ovvero una situazione che rende evidente il fatto che l'inferenza non è corretta.

		\medskip
		Ed infatti, se ci pensiamo bene, non è corretta neanche la prima inferenza: magari non sono un ladro, ma devo essere punito per qualche altro motivo.
	}
\end{frame}

\begin{frame}{Fallacia dell'affermazione del conseguente (1)}

	Consideriamo questa inferenza\ldots
	\begin{center}
		\begin{inference}
			Se manca la benzina, allora l'auto non parte\\
			L'auto non parte\\
			\hline
			Manca la benzina
		\end{inference}
	\end{center}
	che si generalizza nella regola
	\begin{center}
		\begin{inference}
			Se A allora B\\
			B\\
			\hline
			A
		\end{inference}
	\end{center}
	È corretta?
\end{frame}

\begin{frame}{Fallacia della affermazione del conseguente (2)}
	All'apparenza potrebbe sembrare corretta, ma consideriamo un'altra istanza della regola.
	\begin{center}
		\begin{inference}
			Se Fabio è pescarese, allora Fabio è abruzzese\\
			Fabio è abruzzese\\
			\hline
			Fabio è pescarese
		\end{inference}
	\end{center}
	È corretta?

	\pause
	\medskip
	Anche questa inferenza \alert{non è corretta}: pensate di nuovo al caso in cui Fabio sia nato a Chieti.

	\medskip
	Ed infatti, se ci pensiamo bene, non è corretta neanche la prima inferenza: magari l'auto non parte perché è scarica la batteria.
\end{frame}

\begin{frame}{Esercizi consigliati}
	Esercizio 1 del Capitolo 1.

	\medskip
	Esercizi da 1 a 7 del capitolo \emph{Introduzione}.
\end{frame}

\note{Esercizi 1.1, Intro.1, Intro.4, Intro.7}

\section{Logica dei predicati}

\newcommand{\myfbox}[2]{\tikz[baseline=(n.base)]\node(n)[alt=<1>{fill=#1!50}]{#2};}
\newcommand{\myfboxbis}[2]{\tikz[baseline=(n.base)]\node(n)[alt=<2>{fill=#1!50}]{#2};}

% \begin{frame}{Evidenziare la forma logica}
% 	\tikzstyle{every node} = [rounded corners, outer sep=0, inner sep=0.1cm]
% 	\begin{center}
% 		\begin{inference}
% 			\myfbox{red}{2}\myfbox{green}{è minore di}\myfbox{blue}{5}\\
% 			Se\myfbox{orange}{un numero}\myfbox{green}{è minore di}\myfbox{pink}{un altro}, allora\myfbox{pink}{il secondo}\myfbox{gray}{è maggiore del} \myfbox{orange}{primo}\\
% 			\hline
% 			\myfbox{blue}{5}\myfbox{gray}{è maggiore di}\myfbox{red}{2}
% 		\end{inference}\\[0.2cm]
% 		$\Downarrow$\\[0.2cm]
% 		\begin{inference}
% 			\myfbox{green}{R}\myfbox{red}{a}\myfbox{blue}{b}\\
% 			\myfboxbis{green}{per ogni }\myfbox{orange}{$x$}, \myfboxbis{green}{per ogni } \myfbox{pink}{$y$}, \myfboxbis{red}{se} \myfbox{green}{R}\myfbox{orange}{$x$}\myfbox{pink}{$y$}\myfboxbis{red}{allora}\myfbox{gray}{S}\myfbox{pink}{$y$}\myfbox{orange}{$x$}\\
% 			\hline
% 			\myfbox{gray}{S}\myfbox{blue}{b}\myfbox{red}{a}\\
% 		\end{inference}\\[0.2cm]
% 		\pause
% 		$\Downarrow$\\[0.2cm]
% 		\begin{inference}
% 			Rab\\
% 			$\myfboxbis{green}{$\forall x \forall y$} (Rxy \myfboxbis{red}{$\to$} Syx)$\\
% 			\hline
% 			Sba
% 		\end{inference}
% 	\end{center}
% \end{frame}

\begin{frame}{Inferenze a livello predicativo}
	\tikzstyle{every node} = [rounded corners, outer sep=0, inner sep=0.1cm]
	Negli esempi visti prima, le regole di inferenza sono a livello di \alert{logica proposizionale}: la loro verità dipende dai legami tra le proposizioni.

	\medskip
	Queste inferenze sono più complesse:

	\begin{center}
		\begin{inference}
			Napoleone è corso\\
			Tutti i corsi sono francesi\\
			\hline
			Napoleone è francese
		\end{inference}
		\qquad
		\begin{inference}
			Socrate è un uomo\\
			Tutti gli uomini sono mortali\\
			\hline
			Socrate è mortale
		\end{inference}
	\end{center}
	Se analizzate come fatto finora, corrispondono alla regola di inferenza:
	\begin{center}
		\begin{inference}
			A\\
			B\\
			\hline
			C
		\end{inference}
	\end{center}
	che non è corretta! Bisogna passare alla \alert{logica dei predicati}.
\end{frame}


\begin{frame}{Forma logica per regole di inferenza a livello predicativo}
	\tikzstyle{every node} = [rounded corners, outer sep=0, inner sep=0.1cm]
	A livello predicativo, la forma logica si ottiene in questo modo:
	\begin{itemize}
		\item \myfbox{red}{costanti individuali} al posto di individui (Napoleone, Socrate)
		\item \myfbox{green}{costanti predicative} al posto di proprietà (essere corso, essere mortale);
		\item \myfbox{purple}{esiste} o \myfbox{purple}{per ogni} al posto dei quantificatori (tutti, alcuni, \ldots)
	\end{itemize}
	\begin{center}
		\begin{inference}
			\myfbox{red}{Napoleone} \myfbox{green}{è corso}\\
			\myfbox{purple}{Tutti} \myfbox{green}{i corsi}  \myfbox{gray}{sono francesi}\\
			\hline
			\myfbox{red}{Napoleone} \myfbox{gray}{è francese}
		\end{inference}
		\qquad
		\begin{inference}
			\myfbox{red}{Socrate} \myfbox{green}{è un uomo}\\
			\myfbox{purple}{Tutti} \myfbox{green}{gli uomini}  \myfbox{gray}{sono mortali}\\
			\hline
			\myfbox{red}{Socrate} \myfbox{gray}{è mortale}
		\end{inference}

		\medskip
		{\Huge$\Downarrow$}\\[0.2cm]

		\begin{inference}
			$\myfbox{green}{P} \myfbox{red}{a}$\\
			$\myfbox{purple}{per ogni} x, \text{se } \myfbox{green}{P} x \text{ allora } \myfbox{gray}{Q}x$\\
			\hline
			$\myfbox{gray}{Q}  \myfbox{red}{a}$
		\end{inference}

		% \medskip
		% {\Huge$\Downarrow$}\\[0.2cm]

		% \begin{inference}
		% 	$P a$\\
		% 	$\forall x, P x \to Q x$\\
		% 	\hline
		% 	$Q a$
		% \end{inference}
	\end{center}
\end{frame}

\begin{frame}{Inferenze corrette e verità delle conclusioni}
	Come per il caso proposizionale, bisogna distinguere tra correttezza dell'inferenza e verità della conclusione.
	\begin{center}
		\begin{inference}
			Napoleone è francese \only<2->{\rlap\checkmark}\\
			Tutti i francesi sono abruzzesi \only<2->{\rlap\xmark}\\
			\hline
			Napoleone è abruzzese \only<2->{\rlap\xmark}
		\end{inference}
		\qquad
		\begin{inference}
			Napoleone è francese \only<2->{\rlap\checkmark}\\
			Tutti i francesi sono corsi \only<2->{\rlap\xmark}\\
			\hline
			Napoleone è corso \only<2->{\rlap\checkmark}
		\end{inference}

		\medskip
		\begin{inference}
			Napoleone è genovese \only<3->{\rlap\xmark}\\
			Tutti i genovesi sono cinesi \only<3->{\rlap\xmark}\\
			\hline
			Napoleone è cinese \only<3->{\rlap\xmark}
		\end{inference}
		\qquad
		\begin{inference}
			Napoleone è cinese \only<3->{\rlap\xmark}\\
			Tutti i cinesi sono francesi \only<3->{\rlap\xmark}\\
			\hline
			Napoleone è francese \only<3->{\rlap\checkmark}
		\end{inference}

		\medskip
		\begin{inference}
			Napoleone è londinese \only<4->{\rlap\xmark}\\
			Tutti i londinesi sono inglesi \only<4->{\rlap\checkmark}\\
			\hline
			Napoleone è inglese \only<4->{\rlap\xmark}
		\end{inference}
		\qquad
		\begin{inference}
			Napoleone è parigino \only<4->{\rlap\xmark} \\
			Tutti i parigini sono francesi \only<4->{\rlap\checkmark}\\
			\hline
			Napoleone è francese \only<4->{\rlap\checkmark}
		\end{inference}

		\medskip
		\tikz[overlay, remember picture]{\node (nocase) {};}
		\begin{inference}
			?????????????????? \only<6->{\rlap\checkmark}\\
			?????????????????? \only<6->{\rlap\checkmark}\\
			\hline
			?????????????????? \only<6->{\rlap\xmark}
		\end{inference}
		\qquad
		\begin{inference}
			Napoleone è corso \only<5->{\rlap\checkmark} \\
			Tutti i corsi sono francesi \only<5->{\rlap\checkmark}\\
			\hline
			Napoleone è francese \only<5->{\rlap\checkmark}
		\end{inference}
	\end{center}
	\only<7->{
		\begin{tikzpicture}[overlay, remember picture]
			\draw  [red, line width=2mm] (nocase) ++ (0cm, 0.55cm) --  ++ (4cm, -0.8cm);
			\draw  [red,  line width=2mm] (nocase) ++ (0cm, -0.25cm) --  ++ (4cm, 0.8cm);
		\end{tikzpicture}
	}
\end{frame}

\section{Oltre il ragionamento deduttivo}

% \begin{frame}{Ragionamenti e inferenze}
% 	Sebbene abbiamo detto che la logica è lo studio del ragionamento, noi preferiamo non usare questo termine, perché:
% 	\begin{enumerate}
% 		\item ci aspettiamo che le premesse di un ragionamento contengano delle motivazioni a sostegno di una verità. Considereremmo queste inferenze convincenti?

% 		      \medskip
% 		      \begin{center}
% 			      \begin{inference}
% 				      Napoleone è cinese\\
% 				      Tutti i cinesi sono francesi\\
% 				      \hline
% 				      Napoleone è francese
% 			      \end{inference}
% 			      \qquad
% 			      \begin{inference}
% 				      Carlo è ligure o Stefano è piemontese\\
% 				      Stefano non è piemontese\\
% 				      \hline
% 				      Carlo è ligure
% 			      \end{inference}
% 		      \end{center}
% 		      \medskip

% 		\item anche ignorando il punto di cui sopra, le inferenze colgono solo un tipo di ragionamento, chiamato \alert{ragionamento deduttivo}. Ma ce n'è altri!
% 	\end{enumerate}
% \end{frame}

\begin{frame}{Ragionamento induttivo e abduttivo}
	Esistono altre forme di ``ragionamento'', oltre a quello deduttivo.

	\medskip \textbf{Ragionamento induttivo}: \uncover<2->{\textcolor{blue}{dal caso particolare al generale}}

	\begin{center}
		\begin{inference}
			Tutti i corvi finora osservati sono neri\\
			\hline
			\uncover<2->{Tutti i corvi sono neri}
		\end{inference}
	\end{center}

	\uncover<3->{
		\medskip \textbf{Ragionamento abduttivo}: \uncover<4->{\textcolor{blue}{introduzione di ipotesi esplicative}}

		\begin{center}
			\begin{inference}
				L'assassino ha lasciato tracce di fango\\
				Chiunque fosse entrato dal giardino, avrebbe lasciato tracce di fango\\
				\hline
				\uncover<4->{L'assassino è entrato dal giardino}
			\end{inference}
		\end{center}
	}

	\uncover<4->{
		\begin{block}{Nota}
			Dal punto di vista deduttivo, queste inferenze non sono corrette. L'ultima, in particolare, è proprio quella che abbiamo chiamato ``fallacia dell'affermazione del conseguente''.
		\end{block}
	}
	% \textbf{Ragionamento per default (o per difetto)}
	% \begin{inference}
	% 	Normalmente i polacchi non sanno parlare italiano\\
	% 	Karol è polacco\\
	% 	\hline
	% 	\pause
	% 	Karol non sa parlare italiano
	% \end{inference}
\end{frame}

% \begin{frame}{Inferenze e monotònia}
% 	Tutti i tipi di ragionamento visti pocanzi \alert{non sono monotòni}: nuove premesse possono trasformare la conclusione del ragionamento.
% 	\begin{center}
% 		\begin{inference}
% 			Normalmente i polacchi non sanno parlare italiano\\
% 			Karol è polacco\\
% 			Karol ha vissuto 20 anni in Italia\\
% 			Quasi sicuramente chi vive più di 10 anni in Italia sa parlare italiano\\
% 			\hline
% 			Karol sa parlare italiano
% 		\end{inference}
% 	\end{center}

% 	Invece le inferenze (e i ragionamenti deduttivi) \alert{sono monotòne}. Se aggiungo nuove premesse, l'inferenza rimane corretta.

% 	\medskip
% 	Per questo l'inferenza è il meccanismo principale delle \alert{dimostrazioni matematiche}.
% \end{frame}

\begin{frame}{Inferenze e monotònia}
	\begin{itemize}
		\item Ragionamenti induttivi e abduttivi \alert{non sono monotòni}: nuove premesse possono trasformare la conclusione del ragionamento.
		      \begin{center}
			      \begin{inference}
				      L'assassino ha lasciato tracce di fango\\
				      Chiunque fosse entrato dal giardino, avrebbe lasciato tracce di fango\\
				      \textcolor{blue}{Un uomo è stato visto imbrattare appositamente le scarpe di fango}\\
				      \hline
				      L'assassino non è entrato dal giardino
			      \end{inference}
		      \end{center}

		\item Invece il ragionamento deduttivo \alert{è monotòno}: se aggiungo nuove premesse, un'inferenza rimane corretta. Per questo il ragionamento deduttivo è il meccanismo principale delle \alert{dimostrazioni matematiche}:
		      \begin{itemize}
			      \item si parte da proposizioni assunte vere per principio, chiamate \alert{assiomi};
			            \begin{itemize}
				            \item assiomi di Euclide, assiomi dei numeri reali, etc\ldots
			            \end{itemize}
			      \item si applicano inferenze corrette a premesse vere per ottenere conclusioni vere (\alert{teoremi});
			      \item in questo modo il teorema sarà vero per sempre:
			            \begin{itemize}
				            \item \link{\url{https://youtu.be/LlED5V7EuFY?si=40FmAOQ4491dCbVQ&t=313}}
			            \end{itemize}
		      \end{itemize}
	\end{itemize}
\end{frame}

\begin{frame}{L'esempio di Charles S. Peirce (1839-1914)}
	\begin{inference}
		Tutti i fagioli di questo sacchetto sono bianchi\\
		Questi fagioli vengono da questo sacchetto\\
		\hline
		Questi fagioli sono bianchi
	\end{inference}
	\only<2->{\qquad \textbf{Deduzione}}

	\medskip\medskip
	\begin{inference}
		Questi fagioli vengono da questo sacchetto\\
		Questi fagioli sono bianchi\\
		\hline
		Tutti i fagioli di questo sacchetto sono bianchi\\
	\end{inference}
	\only<3->{\qquad \textbf{Induzione}}

	\medskip\medskip
	\begin{inference}
		Tutti i fagioli di questo sacchetto sono bianchi\\
		Questi fagioli sono bianchi\\
		\hline
		Questi fagioli vengono da questo sacchetto
	\end{inference}
	\only<4->{\qquad \textbf{Abduzione}}
\end{frame}

\end{document}
