\documentclass[aspectratio=169,10pt,dvipsnames,xcolor=table,handout]{beamer}

%\setbeameroption{show notes}

\input{preamble.inc}

\title{Conseguenza logica}

\begin{document}

\begin{frame}
    \titlepage
\end{frame}

\begin{frame}
    Siamo giunti alla fine del nostro viaggio nella logica proposizionale. Abbiamo finalmente tutte le conoscenze necessarie per determinare in maniera precisa quali regole di inferenza sono corrette e quali no.
\end{frame}

\section{Conseguenza logica}

\begin{frame}{Conseguenza logica (1)}
    \begin{block}{Ripasso: equivalenza logica}
        Due fp $X$ ed $Y$ sono \alert{equivalenti} ($X \equiv Y$) quando hanno lo stesso valore di verità per ogni assegnamento di verità alle variabili proposizionali.
    \end{block}

    \pause
    La \alert{conseguenza logica} è una forma più debole dell'equivalenza.

    \begin{definition}[Conseguenza logica]
        Una \fp $Y$ è \alert{conseguenza logica} della \fp $X$ quando tutti gli assegnamenti di lettere a valori di verità che rendono vera $X$ rendono vera anche $Y$.
    \end{definition}

    \pause
    Pertanto, se $X$ è vera deve essere vera anche $Y$, ma non è detto il viceversa!

    \pause
    \begin{definition}[Simbolo per conseguenza logica]
        Scriviamo \alert{$X \models Y$} per indicare $Y$ è conseguenza logica di $X$.
    \end{definition}
\end{frame}

\begin{frame}{Conseguenenza logica (2)}
    Ingrandimento del simbolo di conseguenza logica:

    \smallskip
    \begin{center}
        \resizebox{2cm}{!}{$\models$}
    \end{center}

    \pause
    \medskip
    In generale, il concetto di conseguenza logica si generalizza al caso di premesse multiple come segue:
    \begin{definition}[Conseguenenza logica]
        Una \fp $Y$ è \alert{conseguenza logica} delle \fp $X_1, \ldots, X_n$ quando tutti gli assegnamenti di lettere a valori di verità che rendono vere tutte le  $X_i$ contemporanamente, rendono vera anche $Y$.

        \smallskip
        In simboli: $X_1, \ldots, X_n \models Y$.
    \end{definition}
\end{frame}

\begin{frame}{Verificare una conseguenza logica}
    Vogliamo verificare che \textcolor{blue}{$B$} è conseguenza logica di \textcolor{magenta}{$A \to B$} ed \textcolor{magenta}{$A$}. Scriviamo la tavola di verità delle tre formule $A \to B$, $A$ e $B$:
    \begin{center}
        \begin{tabular}{|c|c||c|c|c|}
            $A$                         & $B$                         & \color{magenta}$A \to B$                   & \color{magenta}$A$                         & \color{blue}$B$                         \\
            \hline
            F                           & F                           & V                           & F                           & F                           \\
            F                           & V                           & V                           & F                           & V                           \\
            V                           & F                           & F                           & V                           & F                           \\
            \only<3->{\cellcolor{red}}V & \only<3->{\cellcolor{red}}V & \only<3->{\cellcolor{red}}V & \only<3->{\cellcolor{red}}V & \only<3->{\cellcolor{red}}V
        \end{tabular}
    \end{center}
    \begin{itemize}
        \item<2-> Evidenziamo quali sono gli assegnamenti che rendono vere entrambe le premesse ($A$ e $A \to B$)
        \item<4-> Constatiamo che in tutte queste righe (che poi è una sola) anche la conclusione $B$ è vera
        \item<5-> Quindi $B$ è conseguenza logica di $A \to B$ ed $A$
    \end{itemize}
\end{frame}

\begin{frame}{Un altro esempio}
    Verifichiamo adesso che \textcolor{blue}{$\neg B$} \textbf{non è} conseguenza logica di \textcolor{magenta}{$A \to B$} e \textcolor{magenta}{$\neg A$}.
    \begin{center}
        \begin{tabular}{|c|c||c|c|c|}
            $A$                         & $B$                         & \color{magenta}$A \to B$                   & \color{magenta}$\neg A$                    & \color{blue}$\neg B$                    \\
            \hline
            \only<3->{\cellcolor{red}}F & \only<3->{\cellcolor{red}}F & \only<3->{\cellcolor{red}}V & \only<3->{\cellcolor{red}}V & \only<3->{\cellcolor{red}}V \\
            \only<3->{\cellcolor{red}}F & \only<3->{\cellcolor{red}}V & \only<3->{\cellcolor{red}}V & \only<3->{\cellcolor{red}}V & \only<3->{\cellcolor{red}}F \\
            V                           & F                           & F                           & F                           & V                           \\
            V                           & V                           & V                           & F                           & F
        \end{tabular}
    \end{center}
    \begin{itemize}
        \item<2-> Evidenziamo quali sono gli assegnamenti che rendono vere entrambe le premesse ($A \to B$ e $\neg A$)
        \item<4-> Constatiamo che nella seconda riga evidenziata, le premesse sono vere ma la conclusione ($\neg B$) è falsa
        \item<5-> Quindi $\neg B$ \textbf{non è} conseguenza logica di $A \to B$ e $\neg A$
    \end{itemize}
\end{frame}

\begin{frame}{Conseguenza logica e implicazione (1)}
    Il concetto di conseguenza logica, ha delle affinità con il connettivo dell'implicazione:
    \begin{itemize}
        \item $Y$ è conseguenza logica di $X$ quando non può mai accadere che $X$ è vera ed $Y$ è falsa
        \item la \fp $X\to Y$ è vera sempre tranne quando $X$ è vera ed $Y$ è falsa.
    \end{itemize}

    \medskip Non deve sorprendere quindi il seguente:
    \begin{theorem}[Conseguenza logica e implicazione]
        $X_1, \ldots, X_n \models Y$ se e solo se $X_1 \land \cdots \land X_n \to Y$ è una tatutologia.
    \end{theorem}
\end{frame}

\begin{frame}{Conseguenza logica e implicazione (2)}
    \begin{example}
        Riprendiamo la tavola di verità usata per verificare che $B$ è conseguenza logica di $A \to B$ e $A$, a cui aggiungo un paio di colonne.
        \begin{center}
            \begin{tabular}{|c|c||c|c|c|c|c|}
                $A$ & $B$ & \color{magenta}$A \to B$ & \color{magenta}$A$ & \color{blue}$B$ & \only<1>{\phantom{$(A \to B) \land A$}}\only<2->{$(A \to B) \land A$} & \only<1-2>{\phantom{$((A \to B) \land A) \to B$}}\only<3->{$((A \to B) \land A) \to B$} \\
                \hline
                F   & F   & V         & F   & F & \only<2->{F} & \only<3->V  \\
                F   & V   & V         & F   & V & \only<2->{F} & \only<3->V  \\
                V   & F   & F         & V   & F & \only<2->{F} & \only<3->V  \\
                \cellcolor{red}V   & \cellcolor{red}V   & \cellcolor{red}V         &\cellcolor{red}V   & \cellcolor{red}V & \only<2->{\cellcolor{red}V} & \only<3->{\cellcolor{red}V}
            \end{tabular}
        \end{center}
        \begin{itemize}
            \item<2-> Aggiungiamo una colonna con la congiunzione delle premesse. Sarà vera solo quando tutte le premesse sono vere, ovvero per la riga precedentemente evidenziata in rosso.
            \item<3-> Aggiungiamo una colonna con l'implicazione tra congiunzione delle premesse e conseguenza.
            \item<4-> Per le righe evidenziate,  $(A \to B) \land A \to B$  è vera perché perché il conseguente è vero.
            \item<4-> Per le righe non evidenziate, $(A \to B) \land A \to B$ è vera perché l'antecedente è falso.
            \item<5-> Dunque $((A \to B) \land A) \to B$ è sempre vera.
        \end{itemize}

    \end{example}
\end{frame}


\section{Inferenze corrette}

\begin{frame}{Inferenze e regole di inferenza}
    \begin{block}{Ripasso: inferenza}
        Ricordiamo la differenza tra inferenza
        \begin{center}
                \begin{inference}
                    Se manca la benzina, allora l'auto non parte\\
                    Manca la benzina\\
                    \hline
                    L'auto non parte
                \end{inference}
        \end{center}
        dove premesse e conclusioni sono proposizioni\ldots
    \end{block}

    \pause
    \begin{block}{Ripasso: regola di inferenza}
        \ldots e regola di inferenza
        \begin{center}
            \begin{inference}
                $A \to \neg B$\\
                $A$\\
                \hline
                $\neg B$
            \end{inference}
        \end{center}
        dove premesse e conclusione sono fore proposizionali.
    \end{block}
\end{frame}

\begin{frame}{Conseguenza logica e correttezza delle inferenze (1)}
    \begin{block}{Ripasso: inferenza corretta}
		Una inferenza è \alert{corretta} (o \alert{valida}) se ogni qualvolta le premesse sono vere, allora è necessariamente vera anche la conclusione.
    \end{block}

    \pause Sappiamo anche che la correttezza di una inferenza dipende solo dalla sua forma logica, ovvero dalla regola di inferenza da cui deriva.

    \pause
    \begin{definition}[Regola di inferenza corretta]
        Una regola di inferenza come
        \begin{center}
            \begin{inference}
                $X_1$\\
                $\vdots$\\
                $X_n$\\
                \hline
                $Y$
            \end{inference}
        \end{center}
        è \alert{corretta} (o \alert{valida}) se ogni qualvolta le premesse sono vere, allora è necessariamente vera anche la conclusione.
    \end{definition}

    \pause
    \begin{theorem}
        Una regola di inferenza è corretta quando la conclusione è conseguenza logica delle premesse.
    \end{theorem}
\end{frame}

\begin{frame}{Conseguenza logica e correttezza delle inferenze (2)}
    \begin{definition}[Regola di inferenza corretta]
        \begin{center}
            \begin{inference}
                $X_1$\\
                $\vdots$\\
                $X_n$\\
                \hline
                $Y$
            \end{inference}
        \end{center}
        è \alert{corretta} se ogni qualvolta le premesse sono vere, allora è necessariamente vera anche la conclusione.
    \end{definition}
    \pause
    Ma questa è esattamente la definizione di conseguenza logica
    \[
    X_1, \ldots, X_n \models Y
    \]
    solo espresso in termini un po' più informali.

    \pause
    \begin{theorem}
        Una regola di inferenza è corretta quando la conclusione è conseguenza logica delle premesse.
    \end{theorem}
\end{frame}

\begin{frame}{Conseguenza logica e correttezza delle inferenze (3)}
    \begin{example}[Modus ponens]
        Possiamo verificare formalmente che il \emph{modus ponens} è una inferenza corretta. Il modus ponens è
        \begin{center}
            \begin{inference}
                $A \to B$\\
                $A$\\
                \hline
                $B$
            \end{inference}
        \end{center}
        e abbiamo già visto nelle slide precedenti che $B$ è conseguenza logica di $A \to B$ e di $A$.
    \end{example}

    \pause
    \begin{example}[Fallacia della negazione dell'antecedente]
        Possiamo verificare formalmente che la seguente regola di inferenza non è corretta:
        \begin{center}
            \begin{inference}
                $A \to B$\\
                $\neg A$\\
                \hline
                $\neg B$
            \end{inference}
        \end{center}
        Abbiamo già vistol che $\neg B$ non è conseguenza logica di $A \to B$ e di $\neg A$.
    \end{example}
\end{frame}

\begin{frame}{Ex falso quodlibet (1)}
    Consideriamo la seguente regola di inferenza:
    \begin{center}
        \begin{inference}
            $A$\\
            $\neg A$\\
            \hline
            $B$
        \end{inference}
    \end{center}

    \pause
    \medskip
    Una istanza di questa regola, ad esempio, è
    \begin{center}
        \begin{inference}
            Carla va alla festa\\
            Carla non va alla festa\\
            \hline
            Michele è laureato
        \end{inference}
    \end{center}
\end{frame}

\begin{frame}{Ex falso quodlibet (2)}
    Cerchiamo di verificare la correttezza della regola con le tavola di verità:
    \smallskip
    \begin{center}
        \begin{tabular}{|c|c||c|c|c|}
            $A$ & $B$ & $A$ & $\neg A$ & $B$ \\
            \hline
            F   & F   & F   & V        & F   \\
            F   & V   & F   & V        & V   \\
            V   & F   & V   & F        & F   \\
            V   & V   & V   & F        & V
        \end{tabular}
    \end{center}

    \only<2->{Constatiamo che:}
    \begin{itemize}
        \item<2-> Non esistono righe in cui le premesse ($A$ e $\neg A$) sono contemporanamente vere!
        \item<3-> In tutte le righe in cui le premesse sono contemporanamente vere, anche la conclusione $B$ è vera
        \item<4-> $B$ è conseguenza logica di $A$ e $\neg A$ e quindi l'inferenza è corretta
    \end{itemize}

    \only<5->{Intuitivamente:}
    \begin{itemize}
        \item<5-> Da una premessa sicuramente falsa si può dedurre qualsiasi cosa;
        \item<6-> Si usa spesso la locuzione latina \alert{ex falso sequitur quodlibet} o la sua ellissi \alert{ex falso quodlibet}.
    \end{itemize}
\end{frame}

\begin{frame}{Impegno esistenziale}

    \begin{itemize}[<+->]
        \item Abbiamo affermato che \textbf{in tutte le righe in cui le premesse sono contemporanamente vere, anche la conclusione è vera}.
        \item Nella logica moderna il quantificatore universale non implica l'esistenza di ciò che si quantifica.
        \item La proposizione ``\textbf{tutti gli elefanti volanti sono rosa}'' è vera.
        \item L'idea è che non si può falsificare la proposizione con un controesempio. Se fosse falsa, dovrebbe esistere un elefante volante che non è rosa. Vi sfido a trovarne uno !
        \item  In altre parola, la proposizione
        \begin{center}
            \emph{tutti gli elefanti volanti sono rosa}
        \end{center}
        equivale a \begin{center}
            \emph{non esiste alcun elefante volante che non è rosa}
        \end{center}
        \item Non la si è sempre pensata così in passato. Nella \alert{logica aristotelica}, un frase del tipo ``\emph{Tutti gli A sono B}'' implica l'esistenza di almeno un $A$.
        \item Si dice che nella logica aristotelica il quantificatore ``tutti'' ha un \alert{impegno esistenziale}. Il vostro libro di testo discute questo fatto a pag.~186.
    \end{itemize}
\end{frame}

\end{document}
